\chapter{Fonctionnalit�s du syst�mes (``System Features'')}

The template in Figure 10-1 is organized by system feature, which is
just one possible way to arrange the functional requirements. Other
organizational options include by use case, mode of operation, user
class, stimulus, response, object class, or functional hierarchy (IEEE
1998b). Combinations of these elements are also possible, such as use
cases within user classes. There is no single right choice; you should
select an organizational approach that makes it easy for readers to
understand the product's intended capabilities. I'll describe the
feature scheme as an example.

    \section{Fonctionnalit� X du syst�me (``System Feature X'')}

    State the name of the feature in just a few words, such as "3.1
    Spell Check." Repeat subsections 3.x.1 through 3.x.3 for each
    system feature.

    \subsection{Description et priorit� (``Description and Priority'')}

    Provide a short description of the feature and indicate whether it
    is of high, medium, or low priority. (See Chapter 14, "Setting
    Requirement Priorities.") Priorities are dynamic, changing over
    the course of the project. If you're using a requirements
    management tool, define a requirement attribute for
    priority. Requirements attributes are discussed in Chapter 18 and
    requirements management tools in Chapter 21.

    \subsection{S�quences stimulis/r�ponses (``Stimulus/Response Sequences'')}

    List the sequences of input stimuli (user actions, signals from
    external devices, or other triggers) and system responses that
    define the behaviors for this feature. These stimuli correspond to
    the initial dialog steps of use cases or to external system
    events.

    \subsection{Requis fonctionnels (``Functional Requirements'')}

    Itemize the detailed functional requirements associated with this
    feature. These are the software capabilities that must be present
    for the user to carry out the feature's services or to perform a
    use case. Describe how the product should respond to anticipated
    error conditions and to invalid inputs and actions. Uniquely label
    each functional requirement. 