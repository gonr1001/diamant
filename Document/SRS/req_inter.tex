\chapter{Requis des interfaces externes (``External Interface Requirements'')}

According to Richard Thayer (2002), "External interface requirements
specify hardware, software, or database elements with which a system
or component must interface...." This section provides information to
ensure that the system will communicate properly with external
components. If different portions of the product have different
external interfaces, incorporate an instance of this section within
the detailed requirements for each such portion.

Reaching agreement on external and internal system interfaces has been
identified as a software industry best practice (Brown 1996). Place
detailed descriptions of the data and control components of the
interfaces in the data dictionary. A complex system with multiple
subcomponents should use a separate interface specification or system
architecture specification (Hooks and Farry 2001). The interface
documentation could incorporate material from other documents by
reference. For instance, it could point to a separate application
programming interface (API) specification or to a hardware device
manual that lists the error codes that the device could send to the
software.

    \section{Interfaces utilisateurs (``User Interfaces'')}

    Describe the logical characteristics of each user interface that
    the system needs. Some possible items to include are

    \begin{itemize}
        \item References to GUI standards or product family style
              guides that are to be followed.
        \item Standards for fonts, icons, button labels, images, color
              schemes, field tabbing sequences, commonly used
              controls, and the like.
        \item Screen layout or resolution constraints.
        \item Standard buttons, functions, or navigation links that
              will appear on every screen, such as a help button.
        \item Shortcut keys.
        \item Message display conventions.
        \item Layout standards to facilitate software localization.
        \item Accommodations for visually impaired users.
    \end{itemize}

    Document the user interface design details, such as specific
    dialog box layouts, in a separate user interface specification,
    not in the SRS. Including screen mock-ups in the SRS to
    communicate another view of the requirements is helpful, but make
    it clear that the mock-ups are not the committed screen
    designs. If the SRS is specifying an enhancement to an existing
    system, it sometimes makes sense to include screen displays
    exactly as they are to be implemented. The developers are already
    constrained by the current reality of the existing system, so it's
    possible to know up front just what the modified, and perhaps the
    new, screens should look like.

    \section{Interfaces mat�riels (``Hardware Interfaces'')}

    Describe the characteristics of each interface between the
    software and hardware components of the system. This description
    might include the supported device types, the data and control
    interactions between the software and the hardware, and the
    communication protocols to be used.

    \section{Interfaces logiciels (``Software Interfaces'')}

    Describe the connections between this product and other software
    components (identified by name and version), including databases,
    operating systems, tools, libraries, and integrated commercial
    components. State the purpose of the messages, data, and control
    items exchanged between the software components. Describe the
    services needed by external software components and the nature of
    the intercomponent communications. Identify data that will be
    shared across software components. If the data-sharing mechanism
    must be implemented in a specific way, such as a global data area,
    specify this as a constraint.

    \section{Interfaces de communication (``Communications Interfaces'')}

    State the requirements for any communication functions the product
    will use, including e-mail, Web browser, network communications
    protocols, and electronic forms. Define any pertinent message
    formatting. Specify communication security or encryption issues,
    data transfer rates, and synchronization mechanisms. 