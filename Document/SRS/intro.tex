\chapter{Introduction}

The introduction presents an overview to help the reader understand
how the SRS is organized and how to use it.

    \section{But (``Purpose'')}

    Identify the product or application whose requirements are
    specified in this document, including the revision or release
    number. If this SRS pertains to only part of an entire system,
    identify that portion or subsystem.

    \section{Conventions propres � ce document (``Document Conventions'')}

    Describe any standards or typographical conventions, including
    text styles, highlighting, or significant notations. For instance,
    state whether the priority shown for a high-level requirement is
    inherited by all its detailed requirements or whether every
    functional requirement statement is to have its own priority
    rating.

    \section{Auditoire cibl� et suggestions de lecture (``Intended Audience and Suggested Reading'')}

    List the different readers to whom the SRS is directed. Describe
    what the rest of the SRS contains and how it is organized. Suggest
    a sequence for reading the document that is most appropriate for
    each type of reader.

    \section{�tendu du projet (``Project Scope'')}

    Provide a short description of the software being specified and
    its purpose. Relate the software to user or corporate goals and to
    business objectives and strategies. If a separate vision and scope
    document is available, refer to it rather than duplicating its
    contents here. An SRS that specifies an incremental release of an
    evolving product should contain its own scope statement as a
    subset of the long-term strategic product vision.

    \section{R�f�rences (``References'')}

    List any documents or other resources to which this SRS refers,
    including hyperlinks to them if possible. These might include user
    interface style guides, contracts, standards, system requirements
    specifications, use-case documents, interface specifications,
    concept-of-operations documents, or the SRS for a related
    product. Provide enough information so that the reader can access
    each reference, including its title, author, version number, date,
    and source or location (such as network folder or URL). 