\chapter{Introduction}


    \section{But}

    \diamant{} est un logiciel servant � la construction d'horaires de
    cours et d'examens sur plusieurs sites � partir d'une interface
    utilisateur.

    \section{Conventions propres � ce document}

    Les anglicismes devront �tre en italique.

    \section{Auditoire cibl� et suggestions de lecture}

    Ce document s'adresse � toutes les personnes impliqu�es dans le
    d�veloppement de \diamant{} tout au long de son cycle de vie: il
    s'agit des utilisateurs, des analystes, des architectes, des
    concepteurs, des testeurs et du chef de projet.

    \section{�tendu du projet}

    \diamant{} est un logiciel de construction d'horaires pr�sent� aux
    utilisateurs sous forme de fen�tre avec une barre de menus,
    contenant des sous-menus. La fen�tre principale pr�sente une
    grille d�crivant l'horaire sur lequel l'utilisateur travaille. En
    dessous de la grille horaire se trouve une barre de t�ches
    \corrpascal{Est-ce une barre de ``t�che'' ou une barre de
    ``statut'' ?}  montrant les ressources en utilisation (�tudiants,
    enseignants, activit�s et locaux) et les conflits d�tect�s.

    Les sous-menus sont de deux types~: ceux qui d�clenchent
    l'ex�cution d'une fonctionnalit� et ceux qui appellent une bo�te
    de dialogue, puis d�clenchent des actions. En g�n�ral ces menus
    entra�nent une mise � jour des donn�es en fonction du traitement
    d�clench�.


    \section{R�f�rences}

    Manuel d'utilisation de \diamant{}. 