\chapter{Description globale (``Overall Description'')}

This section presents a high-level overview of the product and the
environment in which it will be used, the anticipated product users,
and known constraints, assumptions, and dependencies.

    \section{Perspective du produit (``Product Perspective'')}

    Describe the product's context and origin. Is it the next member
    of a growing product family, the next version of a mature system,
    a replacement for an existing application, or an entirely new
    product?  If this SRS defines a component of a larger system,
    state how this software relates to the overall system and identify
    major interfaces between the two.

    \section{Fonctionnalit�s du produit (``Product Features'')}

    List the major features the product contains or the significant
    functions that it performs. Details will be provided in Section 3
    of the SRS, so you need only a high-level summary here. A picture
    of the major groups of requirements and how they are related, such
    as a top-level data flow diagram, a use-case diagram, or a class
    diagram, might be helpful.

    \section{Classes et caract�ristiques d'utilisateurs (``User Classes and Characteristics'')}

    Identify the various user classes that you anticipate will use
    this product and describe their pertinent characteristics. (See
    Chapter 6, "Finding the Voice of the Customer.") Some requirements
    might pertain only to certain user classes. Identify the favored
    user classes. User classes represent a subset of the stakeholders
    described in the vision and scope document.

    \section{Environnement d'op�ration (``Operating Environment'')}

    Describe the environment in which the software will operate,
    including the hardware platform, the operating systems and
    versions, and the geographical locations of users, servers, and
    databases. List any other software components or applications with
    which the system must peacefully coexist. The vision and scope
    document might contain some of this information at a high level.

    \section{Contraintes de design et d'impl�mentation (``Design and Implementation Constraints'')}

    Describe any factors that will restrict the options available to
    the developers and the rationale for each constraint. Constraints
    might include the following:

    \begin{itemize}
        \item Specific technologies, tools, programming languages, and
              databases that must be used or avoided.
        \item Restrictions because of the product's operating
              environment, such as the types and versions of Web
              browsers that will be used.
        \item Required development conventions or standards. (For
              instance, if the customer's organization will be
              maintaining the software, the organization might specify
              design notations and coding standards that a
              subcontractor must follow.)
        \item Backward compatibility with earlier products.
        \item Limitations imposed by business rules (which are
              documented elsewhere, as discussed in Chapter 9).
        \item Hardware limitations such as timing requirements, memory
              or processor restrictions, size, weight, materials, or
              cost.
        \item Existing user interface conventions to be followed when
              enhancing an existing product.
        \item Standard data interchange formats such as XML.
    \end{itemize}

    \section{Documentation utilisateur (``User Documentation'')}

    List the user documentation components that will be delivered
    along with the executable software. These could include user
    manuals, online help, and tutorials. Identify any required
    documentation delivery formats, standards, or tools.

    \section{\og{} Pris pour acquis \fg{} et d�pendances (``Assumptions and Dependencies'')}

    An assumption is a statement that is believed to be true in the
    absence of proof or definitive knowledge. Problems can arise if
    assumptions are incorrect, are not shared, or change, so certain
    assumptions will translate into project risks. One SRS reader
    might assume that the product will conform to a particular user
    interface convention, whereas another assumes something
    different. A developer might assume that a certain set of
    functions will be custom-written for this application, but the
    analyst assumes that they will be reused from a previous project,
    and the project manager expects to procure a commercial function
    library.

    Identify any dependencies the project has on external factors
    outside its control, such as the release date of the next version
    of an operating system or the issuing of an industry standard. If
    you expect to integrate into the system some components that
    another project is developing, you depend upon that project to
    supply the correctly operating components on schedule. If these
    dependencies are already documented elsewhere, such as in the
    project plan, refer to those other documents here. 