\chapter{Autres requis non fonctionnels (``Other Nonfunctional Requirements'')}

This section specifies nonfunctional requirements other than external
interface requirements, which appear in section 4, and constraints,
recorded in section 2.5.

    \section{Requis de performance (``Performance Requirements'')}

    State specific performance requirements for various system
    operations. Explain their rationale to guide the developers in
    making appropriate design choices. For instance, stringent
    database response time demands might lead the designers to mirror
    the database in multiple geographical locations or to denormalize
    relational database tables for faster query responses. Specify the
    number of transactions per second to be supported, response times,
    computational accuracy, and timing relationships for real-time
    systems. You could also specify memory and disk space
    requirements, concurrent user loads, or the maximum number of rows
    stored in database tables. If different functional requirements or
    features have different performance requirements, it's appropriate
    to specify those performance goals right with the corresponding
    functional requirements, rather than collecting them all in this
    one section.

    Quantify the performance requirements as specifically as
    possible--for example, "95 percent of catalog database queries
    shall be completed within 3 seconds on a single-user 1.1-GHz Intel
    Pentium 4 PC running Microsoft Windows XP with at least 60 percent
    of the system resources free." An excellent method for precisely
    specifying performance requirements is Tom Gilb's Planguage,
    described in Chapter 12, "Beyond Functionality: Software Quality
    Attributes."

    \section{Requis face aux dangers et � la sant� (``Safety Requirements'')}

    Safety and security are examples of quality attributes, which are
    more fully addressed in section 5.4. I've called these two
    attributes out in separate sections of the SRS template because if
    they are important at all, they are usually critical. In this
    section, specify those requirements that are concerned with
    possible loss, damage, or harm that could result from the use of
    the product (Leveson 1995). Define any safeguards or actions that
    must be taken, as well as potentially dangerous actions that must
    be prevented. Identify any safety certifications, policies, or
    regulations to which the product must conform. Examples of safety
    requirements are

    \begin{itemize}
        \item{\textbf{SA-1}} The system shall terminate any operation
              within 1 second if the measured tank pressure exceeds 95
              percent of the specified maximum pressure.
        \item{\textbf{SA-2}} The radiation beam shield shall remain
              open only through continuous computer control. The
              shield shall automatically fall into place if computer
              control is lost for any reason.
    \end{itemize}

    \section{Requis au niveau de la s�curit� (``Security Requirements'')}

    Specify any requirements regarding security, integrity, or privacy
    issues that affect access to the product, use of the product, and
    protection of data that the product uses or creates. Security
    requirements normally originate in business rules, so identify any
    security or privacy policies or regulations to which the product
    must conform. Alternatively, you could address these requirements
    through the quality attribute called integrity. Following are
    sample security requirements:

    \begin{itemize}
        \item{\textbf{SE-1}} Every user must change his initially
              assigned login password immediately after his first
              successful login. The initial password may never be
              reused.
        \item{\textbf{SE-2}} A door unlock that results from a
              successful security badge read shall keep the door
              unlocked for 8.0 seconds.
    \end{itemize}

    \section{Attributs de qualit� du logiciel (``Software Quality Attributes'')}

    State any additional product quality characteristics that will be
    important to either customers or developers. (See Chapter 12.)
    These characteristics should be specific, quantitative, and
    verifiable. Indicate the relative priorities of various
    attributes, such as ease of use over ease of learning, or
    portability over efficiency. A rich specification notation such as
    Planguage clarifies the needed levels of each quality much better
    than can simple descriptive statements. 