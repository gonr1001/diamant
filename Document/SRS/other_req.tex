\chapter{Autres requis (``Other Requirements'')}

Define any other requirements that are not covered elsewhere in the
SRS. Examples include internationalization requirements (currency,
date formatting, language, international regulations, and cultural and
political issues) and legal requirements. You could also add sections
on operations, administration, and maintenance to cover requirements
for product installation, configuration, startup and shutdown,
recovery and fault tolerance, and logging and monitoring
operations. Add any new sections to the template that are pertinent to
your project. Omit this section if all your requirements are
accommodated in other sections.

\chapter{Appendice A: Glossaire (``Appendix A: Glossary'')}

Define any specialized terms that a reader needs to know to properly
interpret the SRS, including acronyms and abbreviations. Spell out
each acronym and provide its definition. Consider building an
enterprise-level glossary that spans multiple projects. Each SRS would
then define only those terms that are specific to an individual
project.

\chapter{Appendice B: Mod�les analytiques (``Appendix B: Analysis Models'')}

This optional section includes or points to pertinent analysis models
such as data flow diagrams, class diagrams, state-transition diagrams,
or entity-relationship diagrams. (See Chapter 11, "A Picture Is Worth
1024 Words.")

\chapter{Appendice C: Liste de points litigieux (``Appendix C: Issue List'')}

This is a dynamic list of the open requirements issues that remain to
be resolved. Issues could include items flagged as TBD, pending
decisions, information that is needed, conflicts awaiting resolution,
and the like. This doesn't necessarily have to be part of the SRS, but
some organizations always attach a TBD list to the SRS. Actively
manage these issues to closure so that they don't impede the timely
baselining of a high-quality SRS.
