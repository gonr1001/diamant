\documentclass[francais]{article}
\bibliographystyle{unsrt}
%\usepackage[T1]{fontenc}
%\usepackage[applemac]{inputenc}
%\usepackage{french}
\usepackage{babel}
\usepackage[latin1]{inputenc}

%\usepackage{program}
\usepackage{epsf}
\usepackage{verbatim}
%\usepackage{moreverb}
\usepackage{latexsym}


\def\eXit{$\epsilon$\kern-.100em \lower.5ex\hbox{X}\kern-.125em it}
%\usepackage[dvipsone]{color}

\setlength{\topmargin}{0 in}
\setlength{\headheight}{0 in}
\setlength{\headsep}{0 in}
\setlength{\topskip}{0 in}
\setlength{\textheight}{9 in}
\setlength{\footskip}{0.25 in}
%\setlength{\footheight}{0.25 in}
\setlength{\oddsidemargin}{0 in}
\setlength{\evensidemargin}{0 in}
\setlength{\textwidth}{6.5 in}
\setlength{\parskip}{1.5ex plus0.5ex minus0ex}

\newcommand{\bi}{\begin{itemize}}
\newcommand{\ei}{\end{itemize}}


% Number algorithms separately:
%\newtheorem{algorithm}{Algorithm}

% Number algorithms within sections:
\newtheorem{algorithm}{Algorithme}[section]

% For "floating" algorithms, use:
% \begin{figure}
% \begin{program}
% ...
% \end{program}
% \end{figure}

\newcommand{\bv}{\verb}

\newcommand{\bve}{\verb*}

\newcommand{\brun}{\noindent $\triangleright$}
\newcommand{\erun}{$\triangleleft$}

\newcommand{\key}{\textsf}
\newcommand{\ita}{\textit}
\newcommand{\dos}{\textsc}
\newcommand{\bld}{\textbf}
\newcommand{\ang}{\textsf}
\newcommand{\pro}{\texttt}
%\renewcommand{\thesection}{\Roman{section}}
%\renewcommand{\thesubsection}{\Roman{subsection}}
%\usepackage{graphics}
%
\renewcommand{\labelenumi}{\arabic{enumi}.}
\renewcommand{\labelenumii}{\arabic{enumi}.\arabic{enumii}.}
\renewcommand{\labelenumiii}{\arabic{enumi}.\arabic{enumii}.\arabic{enumiii}.}
\renewcommand{\labelenumiv}{\arabic{enumi}.\arabic{enumii}.\arabic{enumiii}.\arabic{enumiv}.}



\newcommand{\sch}{Schneider}

\newcommand{\saphir}{SAPHIR}
\newcommand{\diamant}{DIAMANT}
\newcommand{\diaBug}{diamantBug}
\newcommand{\diaDotc}{DIAMANT 1.5}
\newcommand{\diaDots}{DIAMANT 1.6}
\newcommand{\diaDe}{DIAMANT 2.0}
\newcommand{\diaTr}{DIAMANT 3.0}
\newcommand{\inno}{Innovallia}

\newcommand{\mef}{Maestro \`{e} Fresco}
\newcommand{\uds}{Universit\'{e} de Sherbrooke}
\newcommand{\mm}{GAMMI}


\title{\bf{Suivi Projet \diaDots{} de \eXit}}
\author{Ruben Gonzalez Rubio \\
D\'{e}partement de g\'{e}nie \'{e}lectrique et de g\'{e}nie
informatique\\Universit\'{e} de Sherbrooke\\
Ruben.Gonzalez-Rubio@USherbrooke.ca}
\date{1 septembre 04}

% Modif sur page de notes � la place de e.mail
% Nouvelle figure use1.eps pour gestionnaires.

\begin{document}
%\input{actives.tex}\StdMacActives
\maketitle


\chapter{Introduction}


    \section{But}

    \diamant{} est un logiciel servant � la construction d'horaires de
    cours et d'examens sur plusieurs sites � partir d'une interface
    utilisateur.

    \section{Conventions propres � ce document}

    Les anglicismes devront �tre en italique.

    \section{Auditoire cibl� et suggestions de lecture}

    Ce document s'adresse � toutes les personnes impliqu�es dans le
    d�veloppement de \diamant{} tout au long de son cycle de vie: il
    s'agit des utilisateurs, des analystes, des architectes, des
    concepteurs, des testeurs et du chef de projet.

    \section{�tendu du projet}

    \diamant{} est un logiciel de construction d'horaires pr�sent� aux
    utilisateurs sous forme de fen�tre avec une barre de menus,
    contenant des sous-menus. La fen�tre principale pr�sente une
    grille d�crivant l'horaire sur lequel l'utilisateur travaille. En
    dessous de la grille horaire se trouve une barre de t�ches
    \corrpascal{Est-ce une barre de ``t�che'' ou une barre de
    ``statut'' ?}  montrant les ressources en utilisation (�tudiants,
    enseignants, activit�s et locaux) et les conflits d�tect�s.

    Les sous-menus sont de deux types~: ceux qui d�clenchent
    l'ex�cution d'une fonctionnalit� et ceux qui appellent une bo�te
    de dialogue, puis d�clenchent des actions. En g�n�ral ces menus
    entra�nent une mise � jour des donn�es en fonction du traitement
    d�clench�.


    \section{R�f�rences}

    Manuel d'utilisation de \diamant{}. 



\subsection{Activit\'{e}s}
\begin{enumerate}
    \item Ajout des fonctionnalit�s (Voir section \ref{act1}).
    \begin{enumerate}
        \item Sous-activit\'{e} niveau 2b  (Voir section \ref{act11}).
        \begin{enumerate}
            \item Sous-sous-activit\'{e} niveau 3
            \begin{enumerate}
            \item Sous-sous-sous-activit\'{e} niveau 4  (fini le 4 sep04)
            \item Sous-sous-sous-activit\'{e} niveau 4  (en cours)
            \end{enumerate}
        \end{enumerate}
    \end{enumerate}
    \item Ajout de MouseTrap (Voir section \ref{act2}).
    \item Correction de bugs (Voir section \ref{act3}).
    \item Reorganization de packages et classes (Voir section \ref{act4}).
    \item Ant (Voir section \ref{act5}).
        \begin{enumerate}
            \item Preparation (Voir section \ref{act51}).
            \item Compile (Voir section \ref{act52}).
            \item Construire \pro{.jar} (Voir section \ref{act53}).
            \item Construire \pro{.zip} (Voir section \ref{act54}).
            \item Clean (Voir section \ref{act55}).
    \end{enumerate}
\end{enumerate}

\subsection{Dur\'{e}e du projet} Une ann�e.
\paragraph{Date de d\'{e}part~:} 1 septembre 2004
\paragraph{Estimations et contraintes}


\setcounter{section}{0}
\renewcommand{\labelenumi}{\thesubsection.\arabic{enumi}.}
\renewcommand{\labelenumii}{\thesubsection.\arabic{enumi}.\arabic{enumii}.}
%\renewcommand{\labelenumiii}{\arabic{enumi}.\arabic{enumii}.\arabic{enumiii}.}
%\renewcommand{\labelenumiv}{\arabic{enumi}.\arabic{enumii}.\arabic{enumiii}.\arabic{enumiv}.}

\section{Ajout des fonctionnalit�s} \label{act1}
\subsection*{But de l'activit\'{e}}
Ajouter des fonctionnalit�s demand�es par les utilisateurs.
\subsection*{Description}
La liste de caract�ristiques � ajouter~:

\begin{itemize}

    \item Horaire sans �tudiants. \label{nostudents}
    \item D�finition des ensembles et grille partielle (bug. 8, 12) \label{parti}.
    \item Assignation  automatique des locaux. \label{rooms}
    \item Le nombre de cours dans une p�riode doit prendre en
    compte la capacit� de locaux (bug. 3).
    \item Importation s�lective (bug. 11, 20).  Fichier exportation .horaire.\label{select}
    \item Algorithme d'optimisation plus efficace (bug. 38). \label{optim}
    \item Affectation manuelle plus efficace.\label{affect}
    \item Toolbar et horaire \label{toolbar}
    \item Raccourcis clavier (bug. 71). \label{clavier}
    \item Avoir une fonction "Undo" permettant de revenir en arri�re sur la derni�re
    action effectu�e et une fonction "Redo" pour refaire une action d�faite (bug. 4).
     \item Code de couleur (bug. 22). \label{couleur}

\end{itemize}

\subsection{Horaire sans �tudiants} \label{nostudents}

La fonctionnalit� permet de faire un horaire sans �tudiants.
Nouveau type de fichier avec le nombre maximum d'�tudiants par
activit�, doit �tre lu.

\subsection{Grille partielle} \label{partielle}
La fonctionnalit� Grille partielle doit permettre de d�finir un
ensemble d'activit�s, et de donner un nom � cet ensemble. Ensuite
afficher sur la grille uniquement les cours de l'ensemble choisi.

Il faut aussi sauvegarder les donn�es nom de l'ensemble et les
activit�s appartenant dans l'ensemble.

\begin{itemize}
    \item Fichiers. Sauvegarde en XML. Ind�pendance de l'horaire
    en cours. 
    \item Mod�le. Ensembles avec op�rations classiques ajout,
    suppression, recherche et tri.
    \item Vue. \pro{PeriodPanel} qui prend un Panel de plus, pour afficher les conflit partiels et
    le total de conflits.
\end{itemize}

Quelles classes sont � modifier?



\subsection{Assignation  automatique des locaux} \label{rooms}

La fonctionnalit� permet de assigner les locaux aux �v�nements
avec une utilisation optimal des locaux.


\subsection{Capacit� totale de locaux (bug. 3)} \label{capacity}
\subsection{Importation s�lective (bug. 11, 20)}
    \label{importationSel}
    La fonctionnalit� importation s�lective est de pouvoir importer un
fichier sans changer les donn�es dans les autres fichiers.

Quels fichiers peuvent �tre import�s de mani�re selective?

Quelle sont les cas possibles~: donn�e nouvelle, donn�e modifie,
donn�e disparue, autres?

Quoi faire dans chaque cas?


Quelles classes sont � modifier?
\subsection{Algorithme d'optimisation plus efficace (bug. 38)}\label{optim}

\subsection{Affectation manuelle plus efficace} \label{affect}
\subsection{Toolbar et horaire}\label{toolbar}

La fonctionnalit� Toolbar est de pouvoir effectuer des
modifications � une grille horaire qui est d�j� associ�e � un
ensemble de donn�es. Ceci c'est le cas lorsque on travaille avec
une grille faite par  Nouvel horaire (cours ou examens).

Il s'agit de modifier la priorit� de p�riodes, de ajouter ou
enlever des journ�es. Modifier les noms de journ�es.

Quoi faire avec les donn�es d�j� assign�s dans la grille si la
p�riode est modifi�e ou �limin�e.

 Quelles classes sont � modifier?

\subsection{Raccourcis clavier (bug. 71)} \label{clavier}
\subsection{Fonction "Undo"  (bug. 4)} \label{undo}
\subsection{Code de couleur (bug. 22)} \label{couleur}





\subsection{Ent�te de rapports} \label{rapport}

La fonctionnalit� garde toujours l'ent�te de rapports visible.


%

\subsection{Sous-Activit\'{e}} \label{act12}
\paragraph{But de la sous-activit\'{e}}
Produire la classe  traitant \ldots
\paragraph{Description}
Description de la sous-activit\'{e} 1.2, \`{a} d\'{e}crire.

\begin{enumerate}
    \item sous-sous-activit\'{e} \`{a} d\'{e}crire.
    \begin{enumerate}
    \item sous-sous-sous-activit\'{e} \`{a} d\'{e}crire.
    \end{enumerate}
\end{enumerate}


\section{Ajout de MouseTrap} \label{act2}
\subsection*{But de l'activit\'{e}}
MouseTrap doit servir � garder la trace de l'activit� d'un
utilisateur afin de faire ex�cuter le programme par un d�veloppeur
qui doit trouver le probl�me conduisant � un bug.
\subsection*{Description}
Description de l'activit\'{e} 1, \`{a} d\'{e}crire.


\subsection{Validation du prototype} \label{act21}
\paragraph{But de la sous-activit\'{e}}
� venir.
\paragraph{Description}
� venir.

\subsection{Modification du code} \label{act22}
\paragraph{But de la sous-activit\'{e}}
� venir.
\paragraph{Description}
� venir.


%

\subsection{Sous-Activit\'{e}} \label{act12}
\paragraph{But de la sous-activit\'{e}}
Produire la classe  traitant \ldots
\paragraph{Description}
Description de la sous-activit\'{e} 1.2, \`{a} d\'{e}crire.

\begin{enumerate}
    \item sous-sous-activit\'{e} \`{a} d\'{e}crire.
    \begin{enumerate}
    \item sous-sous-sous-activit\'{e} \`{a} d\'{e}crire.
    \end{enumerate}
\end{enumerate}


\section{Correction de bugs} \label{act3}
\subsection*{But de l'activit\'{e}}
Correction de bugs.
\subsection*{Description}
\begin{itemize}
    \item Horaire d'examens avec horaire de cycle.
    J'ai ajout� deux jours � la grille d'examen qui en contenait 10.
    Je n'ai pas �t� capable de modifier la priorit� associ�e � chacune
    des p�riodes des jours 11 et 12. (bug 62). \label{grilleH}
    \item Revoir les messages et les exceptions  lorsqu'un fichier est incorrect(bug. 69). \label{fichier}
    \item Revoir le rapport d'importation (bug. 70). \label{fichier}
    \item Ent�te de rapports \label{headers}
\end{itemize}


\subsection{Mauvais clonage de p�riodes} \label{act31}
\paragraph{But de la sous-activit\'{e}}
Corriger bug.
\paragraph{Description}
� venir. Mauvais clonage de p�riodes. Horaire d'examens avec
horaire de cycle.
    J'ai ajout� deux jours � la grille d'examen qui en contenait 10.
    Je n'ai pas �t� capable de modifier la priorit� associ�e � chacune
    des p�riodes des jours 11 et 12. (bug 62).

\subsection{Revoir les messages et les exceptions lorsqu'un fichier est incorrect} \label{act32}
\paragraph{But de la sous-activit\'{e}}
Corriger bug.
\paragraph{Description}
� venir.

\subsection{Revoir le rapport d'importation} \label{act33}
\paragraph{But de la sous-activit\'{e}}
Corriger bug.
\paragraph{Description}
� venir.

\subsection{(Ent�te de rapports} \label{act34}
\paragraph{But de la sous-activit\'{e}}
Corriger bug.
\paragraph{Description}
Garder toujours l'ent�te de rapports visible.


%

\subsection{Sous-Activit\'{e}} \label{act12}
\paragraph{But de la sous-activit\'{e}}
Produire la classe  traitant \ldots
\paragraph{Description}
Description de la sous-activit\'{e} 1.2, \`{a} d\'{e}crire.

\begin{enumerate}
    \item sous-sous-activit\'{e} \`{a} d\'{e}crire.
    \begin{enumerate}
    \item sous-sous-sous-activit\'{e} \`{a} d\'{e}crire.
    \end{enumerate}
\end{enumerate}



\section{Reorganization de packages et classes} \label{act4}
\subsection*{But de l'activit\'{e}}
Produire le logiciel traitant \ldots
\subsection*{Description}








\subsection{Int�grer les librairies} \label{act41}
\paragraph{But de la sous-activit\'{e}}
Int�grer les  libraries.
\paragraph{Description}
Nom du package \pro{eLib}, un package \pro{exit}, des packages
\pro{dialog}, \pro{exception}, \pro{txt} et \pro{xml}.

Cr�er les packages, mettre les fichiers aux bons endroits, changer
les noms, tester les packages, tester le tout.

\subsection{Inspection de les librairies} \label{act42}
\paragraph{But de la sous-activit\'{e}}
V�rifier tous les fichiers des libraries, commentaires, javaDoc,
noms, utilisation.
\paragraph{Description}
Faire l'inspection.


\subsection{Finir organization} \label{act43}
\paragraph{But de la sous-activit\'{e}}
� venir.
\paragraph{Description}
� venir.


%

\subsection{Sous-Activit\'{e}} \label{act12}
\paragraph{But de la sous-activit\'{e}}
Produire la classe  traitant \ldots
\paragraph{Description}
Description de la sous-activit\'{e} 1.2, \`{a} d\'{e}crire.

\begin{enumerate}
    \item sous-sous-activit\'{e} \`{a} d\'{e}crire.
    \begin{enumerate}
    \item sous-sous-sous-activit\'{e} \`{a} d\'{e}crire.
    \end{enumerate}
\end{enumerate}



\section{Ant} \label{act5}
\subsection*{But de l'activit\'{e}}
Produire un ou des fichiers Ant afin d'automatiser la construction
du \pro{.zip} livr� aux utilisateurs. Ceci doit se faire en
�vitant les redondances de fichiers.
\subsection*{Description}
Description un fichier Ant \cite{hatcher03} peut r�aliser
plusieurs t�ches, comme par exemple effacer des r�pertoires copier
des fichiers et bien entendu compiler et fabriquer des \pro{.jar}.
Les sous-activit�s indiquent les targets (terminologie Ant pour
une t�che.



\subsection{Sous-Activit\'{e}} \label{act51}
\paragraph{But de la sous-activit\'{e}}
Produire la classe  traitant \ldots
\paragraph{Description}


\subsection{Sous-Activit\'{e}} \label{act52}
\paragraph{But de la sous-activit\'{e}}
Produire la classe  traitant \ldots
\paragraph{Description}


%

\subsection{Sous-Activit\'{e}} \label{act12}
\paragraph{But de la sous-activit\'{e}}
Produire la classe  traitant \ldots
\paragraph{Description}
Description de la sous-activit\'{e} 1.2, \`{a} d\'{e}crire.

\begin{enumerate}
    \item sous-sous-activit\'{e} \`{a} d\'{e}crire.
    \begin{enumerate}
    \item sous-sous-sous-activit\'{e} \`{a} d\'{e}crire.
    \end{enumerate}
\end{enumerate}


\bibliography{bibDiamant}
\end{document}
