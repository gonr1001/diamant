\section{Introduction}
Ce document sert \`{a} faire le suivi du projet \diaDots{} et il
suit les directives du groupe \eXit{}.

\subsection{But du projet}
Le but de ce projet est de r�aliser \diaDots{}. Il s'agit
d'ajouter de nouvelles caract�ristiques � \diaDotc{}, l'ajout de
MouseTrap et de corriger les bugs.

Les nouvelles caract�ristiques on peut nommer~:

\begin{itemize}
\item Amelioration de l'optimisation.

\item D�finition des ensembles � visualiser et les visualiser.



\end{itemize}

MouseTrap est une caract�ristique particuli�re car elle n'est pas
visible � l'utilisateur.

Les bugs sont � r�soudre selon leur pr�sence.

Deux autres points doivent sont aussi inclus dans les t�ches �
faire~: reorganization des libraries et l'utilisation d'Ant pour
produire le \pro{.zip} � envoyer aux utilisateurs.
