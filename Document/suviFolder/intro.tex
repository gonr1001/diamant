\section{Introduction}
Ce document sert \`{a} faire le suivi du projet \diaDots{} et il
suit les directives du groupe \eXit{}.

\subsection{But du projet}
Le but de ce projet est de r�aliser \diaDots{}. Il s'agit
d'ajouter de nouvelles caract�ristiques � \diaDotc{}, l'ajout de
MouseTrap et de corriger les bugs.

Les nouvelles caract�ristiques on peut nommer~: Amelioration de
l'optimisation, d�finition des ensembles � visualiser et les
visualiser, l'assignation des locaux, l'horaire sans �tudiants. La
liste compl�te est dans la section \ref{act1}.

MouseTrap est une caract�ristique particuli�re car elle n'est pas
visible � l'utilisateur.

Les bugs sont � r�soudre selon leur pr�sence. Il est possible
qu'une amelioration vienne d'un rapport de bug (amelioration).
Bien entendu les bugs sont num�rot�s selon \diaBug{}.

Deux autres points doivent sont aussi inclus dans les t�ches �
faire~: reorganization des libraries et l'utilisation d'Ant pour
produire le \pro{.zip} � envoyer aux utilisateurs.
