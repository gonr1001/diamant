

\section*{Introduction}
Ce document sert \`{a} faire le suivi du projet \diaDots{} tout en
suivant  les directives du groupe \eXit{}.

\subsection*{But du projet}
Le but de ce projet est de r�aliser et livrer  \diaDots{}. Il
s'agit d'ajouter de nouvelles fonctionnalit�s  � \diaDotc{} (liste
plus loin), l'ajout de \do{} et de corriger les bugs.

Les nouvelles fonctionnalit�s sont~: Amelioration de
l'optimisation, d�finition des ensembles � visualiser et leur
visualisation, l'assignation des locaux, l'horaire sans �tudiants
et l`horaire dans plusieurs sites. La liste compl�te avec de
d�tails est dans la section \ref{ajoutFonc}.

La fonctionnalit� \do{} est  particuli�re car elle n'est pas
visible � l'utilisateur, mais doit servir pour trouver des bugs.

Les bugs sont � r�soudre selon leur pr�sence. Il est possible
qu'une amelioration vienne d'un rapport de bug (amelioration). Les
bugs sont num�rot�s selon \diaBug{}.

Deux autres points doivent sont aussi inclus dans les t�ches �
faire~: reorganization des libraries et des packages ainsi que
l'utilisation d'Ant pour produire le \pro{.zip} � envoyer aux
utilisateurs.
