



\subsection{R�organisation 1 (fini 10 sep 04)} \label{reorg1}
\paragraph{But de la sous-activit\'{e}}
R�organisation de packages
\paragraph{Description}
Voici les changements faits :
\begin{description}
  \item[\verb!dConstant!] Contient \verb!DConst! et
  \verb!DStringFrRes!, cette derni�re contient les cha�nes en
  Fran\c{c}ais.
  \item[\verb!dDeveloper!] Contient les classes qui ne sont pas
  inclues avec la version livrable.
  \item[\verb!dInterface!] Contient les classes  qui affichent
  quelque
  chose � l'�cran.
    \begin{description}
    \item[\verb!x!] � d�finir.
    \item[\verb!y!] � d�finir.
    \item[\verb!z!] � d�finir.
    \end{description}
  \item[\verb!dInternal!]

    \begin{description}
    \item Contient classe \verb!DModel! contenant les donn�es de
    l'application
    \item[\verb!dDataTxt!] Contient les classes o� on traite les
    donn�es textuelles de l'ext�rieur.
    \item[\verb!dDataXML!] Contient les classes o� on traite les
    donn�es XML de l'ext�rieur.
    \item[\verb!dOptimization!] Contient les classes o� on traite
    l'optimisation : algorithmes ou conditions ou structures
    auxiliaires.
    \item[\verb!dTimeTable!] Contient les classes n�cessaires � la
    grille horaire.
    \item[\verb!dUtil!] Contient les classes g�n�riques, mais n�cessaires uniquement �
    \diamant{}, utilis�es
    dans \verb!dInternal!
    \end{description}
  \item[\verb!dmains!] Contient les \verb!main!s pour lancer les
  diff�rentes applications.
  \item[\verb!dTest!]
\end{description}


Une librairie unique des classes g�n�riques regroupant entre
autres les libraries \verb!theLibrary! et \verb!xmlLibrary! plus
d'autres classes non d�pendantes de \diamant{}. Tout sur un
package qui fera un \pro{.jar} ind�pendant.
