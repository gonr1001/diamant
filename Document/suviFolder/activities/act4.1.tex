
Voil� le r�sultat et comment choisir un package pour une nouvelle
classe.

\begin{description}
  \item[\verb!dConstant!] Contient \verb!DConst! et
  \verb!DStringFrRes!, cette derni�re contient les cha�nes en
  Fran\c{c}ais.
  \item[\verb!dDeveloper!] Contient les classes qui ne sont pas
  inclues avec la version livrable.
  \item[\verb!dInterface!] Contient les classes  qui affichent
  quelque
  chose � l'�cran.
    \begin{description}
    \item[\verb!x!] � d�finir.
    \item[\verb!y!] � d�finir.
    \item[\verb!z!] � d�finir.
    \end{description}
  \item[\verb!dInternal!]

    \begin{description}
    \item Contient classe \verb!DModel! contenant les donn�es de
    l'application
    \item[\verb!dDataTxt!] Contient les classes o� on traite les
    donn�es textuelles de l'ext�rieur.
    \item[\verb!dDataXML!] Contient les classes o� on traite les
    donn�es XML de l'ext�rieur.
    \item[\verb!dOptimization!] Contient les classes o� on traite
    l'optimisation : algorithmes ou conditions ou structures
    auxiliaires.
    \item[\verb!dTimeTable!] Contient les classes n�cessaires � la
    grille horaire.
    \item[\verb!dUtil!] Contient les classes g�n�riques, mais n�cessaires uniquement �
    \diamant{}, utilis�es
    dans \verb!dInternal!
    \end{description}
  \item[\verb!dmains!] Contient les \verb!main!s pour lancer les
  diff�rentes applications.
  \item[\verb!dTest!]
    \begin{description}
    \item[\verb!dInterface!]
    \item[\verb!dInternal!]
        \begin{description}
        \item[\verb!dDataTxt!]Contient les classes pour tester les
    donn�es textuelles de l'ext�rieur.
        \item[\verb!dDataXML!]Contient les classes pour tester les
    donn�es XML de l'ext�rieur.
        \item[\verb!dOptimization!] Contient les classes pour
        tester les classes de
    l'optimisation : algorithmes ou conditions ou structures
    auxiliaires.
        \item[\verb!dTimeTable!]Contient les classes n�cessaires
        au test de classes de la
    grille horaire.
        \item[\verb!dUtil!]Contient les classes n�cessaires
        au test de classes de la
    \verb!dUtil!.
    \end{description}
    \item[\verb!dOptimization!]
    \item[\verb!dTimeTable!]
    \item[\verb!dUtil!]
    \end{description}
\end{description}


Une librairie unique des classes g�n�riques regroupant entre
autres les libraries \verb!theLibrary! et \verb!xmlLibrary! plus
d'autres classes non d�pendantes de \diamant{}. Tout sur un
package qui fera un \pro{.jar} ind�pendant.

\subsection{Les librairies}\label{act41}
\paragraph{But de la sous-activit\'{e}}
Faire la librarie.
\paragraph{Description}
Nom du package \pro{eLib}, un package \pro{exit}, des packages
\pro{dialog}, \pro{exception}, \pro{txt} et \pro{xml}.

Cr�er les packages, mettre les fichiers aux bons endroits, changer
les noms, tester les packages, tester le tout.

\subsection{Finir organization} \label{act42}
\paragraph{But de la sous-activit\'{e}}
� venir.
\paragraph{Description}
� venir.
