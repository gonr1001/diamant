\documentclass{llncs}
\bibliographystyle{alpha}
\usepackage[latin1]{inputenc}
\usepackage[T1]{fontenc}
\usepackage[francais]{babel}

\usepackage{dfd,hhline}
\usepackage{verbatim}
\setlength{\parskip}{1.5ex plus0.5ex minus0ex}

\newcommand{\ang}{\textsf}
\newcommand{\key}{\textsf}
\newcommand{\ita}{\textit}
\newcommand{\bld}{\textbf}
\newcommand{\dos}{\textsc}
\newcommand{\pro}{\texttt}

\newcommand{\saphir}{SAPHIR}
\newcommand{\diamant}{DIAMANT}
\newcommand{\tictac}{Tic-Tac}
\newcommand{\xp}{eXtreme Programming}
\def\eXit{$\epsilon$\kern-.100em \lower.5ex\hbox{X}\kern-.125emit}

\title{\bf{Mod�le math�matique pour construire un horaire d'examens}}
%\author{Ruben Gonzalez Rubio \\Domingo Palao Mu�oz}
%\institute{D�partement de g�nie �lectrique et de g�nie informatique \\
%Universit� de Sherbrooke,\\
%Sherbrooke, Qu�bec, J1K 2R1 \\
%Canada \\
%\email{Ruben.Gonzalez-Rubio@USherbrooke.ca} \\
%\email{Domingo.Palao@USherbrooke.ca}}
%\date{}

%%%%%%%%%%%%%%%%%%
\pagestyle{plain} %% to be commented when sent
\newcommand{\ints}{\renewcommand{\baselinestretch}{1.0}\small\normalsize}
\newcommand{\intm}{\renewcommand{\baselinestretch}{1.5}\small\normalsize}
\newcommand{\intd}{\renewcommand{\baselinestretch}{2.0}\small\normalsize}
%%%%%%%%%%%%%%%%%%

\begin{document}

%\maketitle
{\bf{Mod�le math�matique pour construire un horaire d'examens}}
%\intd

Consid�rations de base pour l'horaire d'examens~: Il y a un examen
pour chaque cours. Tous les examens on la m�me dur�e, donc toutes
les p�riodes de la grille horaire aussi.

Soit~:
\[
\begin{array}{l}
i~ \mbox{d�finie comme le num�ro de l'examen d'un cours}~(i = 1, \ldots, m) \\
t ~\mbox{d�finie comme le num�ro de la p�riode dans l'horaire}~(t
= 1 \ldots, n)\\
\end{array}
\]

\[
x_{i,t} \mbox{est d�finie} \left\{
\begin{array}{r@{\hspace{15pt}}l}
1 & \mbox{si l'examen}~i~\mbox{est affect� � une p�riode de temps}~t \\
0 &  \mbox{sinon} \\
\end{array} \right.
\]

\[
\begin{array}{l}
e_{i,i'}~ \mbox{le nombre d'�tudiants qui prenennt l'examen num�ro}~i~\mbox{et l'examen num�ro}~i'~\mbox{au m�me temps}\\
\end{array}
\]

Les conflits.

\[
x_{i,t} x_{i',t}  \left\{
\begin{array}{r@{\hspace{15pt}}l}
1 & \mbox{si}~i~\mbox{et}~i'~\mbox{sont dans la m�me p�riode de temps}~t \\
0 &  \mbox{sinon} \\
\end{array} \right.
\]


\[
\mbox{conflits entre}~i~\mbox{et}~i' = e_{i,i'} \sum^n_{t=1}
x_{i,t} x_{i',t}
\]

La fonction objectif est~:

\[
\mbox{min}\sum^{m-1}_{i=1}  \sum^{m}_{i'=i+1} e_{i,i'}
\sum^n_{t=1} x_{i,t} x_{i',t}
\]

Le mod�le final est~:
\[
%\begin{array}{l@{\hspace{15pt}}l@{\hspace{15pt}}l@{\hspace{15pt}}l}
%~ &
\mbox{min}\sum^{m-1}_{i=1}  \sum^{m}_{i'=i+1} e_{i,i'}
\sum^n_{t=1} x_{i,t} x_{i',t}  ~~~~~  \mbox{total de conflits}
%
\]


\[
\begin{array}{l@{\hspace{15pt}}l@{\hspace{15pt}}l@{\hspace{15pt}}l}
\mbox{avec les contraintes} & \sum\limits^{n}_{t=1}  x_{i,t} & i =
1,
\ldots, m  & \mbox{un seul examen}~i \\
~ & x_{i,t} = 0~\mbox{ou}~1 & i = 1, \ldots, m;~ t = 1 \ldots, n &
~ \\
\end{array}
\]


\end{document}
