\documentclass[francais]{report}

%-----------------------------------------------------------------------------
%----- Identification des packages n�cessaires
%-----------------------------------------------------------------------------

\usepackage{babel}
\usepackage[latin1]{inputenc}
\usepackage{DxTitle}
%\newcommand{\diamant}{Diamant}
\setlength{\oddsidemargin}{0.25in}
\setlength{\evensidemargin}{0.25in} \setlength{\textwidth}{6.0in}
%\setlength{\parskip}{0.2in}
\newcounter{auxcounter}
\renewcommand{\baselinestretch}{1.5}
\setlength{\parskip}{1.5ex plus0.5ex minus0ex}

\newcommand{\ints}{\renewcommand{\baselinestretch}{1.0}\small \normalsize}
\newcommand{\intm}{\renewcommand{\baselinestretch}{1.5}\small \normalsize}
\newcommand{\intd}{\renewcommand{\baselinestretch}{2.0}\small \normalsize}

\newcommand{\bi}{\begin{itemize}}
\newcommand{\ei}{\end{itemize}}
\newcommand{\be}{\begin{enumerate}}
\newcommand{\ee}{\end{enumerate}}
\newcommand{\bd}{\begin{description}}
\newcommand{\ed}{\end{description}}



\newcommand{\bv}{\verb}

\newcommand{\bve}{\verb*}

\newcommand{\brun}{\noindent $\triangleright$}
\newcommand{\erun}{$\triangleleft$}

\newcommand{\ang}{\textsf}
\newcommand{\key}{\textsf}
\newcommand{\ita}{\textit}
\newcommand{\bld}{\textbf}
\newcommand{\dos}{\textsc}
\newcommand{\pro}{\texttt}

\newcommand{\diamant}{DIAMANT}
\newcommand{\saphir}{SAPHIR}
\def\eXit{$\epsilon$\kern-.100em \lower.5ex\hbox{X}\kern-.125em
it}

\faculte{Facult\'e de g\'enie} \departement{G\'enie \'electrique
et g\'enie informatique} \adresse{Sherbrooke (Qu\'ebec) Canada}

%-----------------------------------------------------------------------------
%----- Page Titre
%-----------------------------------------------------------------------------


\Titre{DX\\
Choses � faire} \Logo{Images/logoDX.eps}
\Auteurs{Yannick Syam,\\
Alexander Jaramillo  et\\
Ruben Gonzalez-Rubio} \Date{28 avril 2003}

%-----------------------------------------------------------------------------
%----- Identification des fichiers des pages pr�liminaires et bibliographique
%-----------------------------------------------------------------------------

%\FichierResume{} %{Inputs/resume}
%\FichierRemerciements{}
%\FichierGlossaire{} % \FichierLexique est �quivalent
%\FichiersBibliographie{udsplain}{Inputs/bibDiamant,Inputs/bib2}

%-----------------------------------------------------------------------------
%----- Le document
%-----------------------------------------------------------------------------

%\includeonly{Inputs/resume,Inputs/intro,Inputs/glossaire,Inputs/index
%}

\begin{document}
%\begin{articleDX}
\pageTitre
%\chapter*{Liste de choses � faire}


\begin{enumerate}
    \item 
    \item 
    \item 
\end{enumerate}


\section{Changement de noms}

\begin{enumerate}
    \item \verb!Instructor! $\rightarrow$ \verb!InstructorAttach!
    \item \verb!Room! $\rightarrow$ \verb!RoomAttach!
    \item \verb!Room! $\rightarrow$ \verb!RoomAttach!
    \item \verb!Room! $\rightarrow$ \verb!y!
\end{enumerate}

\chapter*{Liste de choses � faire}



\section{Janvier 2004}

Voici les noms de fonctionnalit�s manquantes dans \diamant 1.5.
\begin{enumerate}
    \item Examens. Voir manuel utilisateur. \label{examen} Fait 6 f�v
    2004. Utilis� par 3 utilisateurs.
    \item Enseignants. Une activit� peut avoir $n$ enseignants.
    \label{enseig}Fait 13 avril 2004.
    \item Groupe. Un groupe est d�finit par une cha�ne de 2 chars \verb!00! �
    \verb!99! \label{groupe}. Fait 6 f�v 2004. Utilis� par 3 utilisateurs.
    \item Grille partielle. \label{parti}
    \item Importation s�lective. \label{select}
    \item Automate menus. \label{menus} Fait le 8 mars. Utilis� par 1 utilisateur.
    \item Petits d�tails. \label{details} Fait le 8 mars. Utilis� par 4 utilisateurs.
    \item Toolbar et horaire \label{toolbar}
    \item Affectation manuelle plus efficace.\label{affect}
    \item Corriger dialogue de liste d'�v�nements. \label{event}Fait le 8 mars. Utilis� par 2 utilisateurs.
    \item Horaire sans �tudiants. \label{nostudents}
    \item Horaire affectation des locaux. \label{rooms}
    \item Automate pour menu Affectation initiale.
    \label{affecIni} Fait le 18 mars.
    \item Remplacer les boutons Fermer par annuler.
    \label{buttons}. Fait 13 avril 2004.
    \item Rendre le rapport d'import visible une seule fois (lors de
creation d'un horaire) \label{rapports} Fait le 18 mars.
    \item Ent�te de rapports \label{headers}

\end{enumerate}
Toutes les modifications de la version 20 avril ont �t� utilis�es
par C. Villeneuve avant le 10 mai 04.


\subsection{Fonctionnalit� \ref{examen}~: Examens}
Cette fonctionnalit� est en r�alit� un ensemble de fonctionnalit�s
n�cessaires � la construction d'un horaire des examens. Elle est
d�crite dans le manuel utilisateur.

\subsection{Fonctionnalit� \ref{enseig}~: Enseignants}
Cette fonctionnalit� doit servir � assigner � une activit� $n$
enseignants. Dans les bo�tes de dialogue affectation d'une
activit� ou d'un �v�nement il faut offrir la possibilit� d'ajouter
$n$ enseignants. On peut ajouter uniquement des enseignants qui
sont connus par \diamant{}. C'est-�-dire ceux qui sont dans les
fichiers d'importation. La disponibilit� des enseignants ajout�s
est � prendre en compte.

� discuter ajout avec de \verb!JComboBox! ou comme ceux de
modifications.

Quelles classes sont � modifier?

\subsection{Fonctionnalit� \ref{groupe}~: Groupes}
Cette fonctionnalit� doit servir � assurer une compatibilit� entre
le nom assign� � un groupe et le nom affich� par \diamant{} et
retourn� avec l'horaire. Il y a des modifications � faire au SIG
et dans \diamant{}. Au SIG le nom d'un groupe est d�finit par une
cha�ne de 2 \verb!char!s \verb!00! ou \verb!01! � \verb!99!. Le
\verb!00! veut dire que l'�tudiant n'est pas dans aucun groupe.
 Cette
information arrive dans le fichier cours et arrive dans le fichier
�tudiants dans le cas de la facult� d'administration (afin
d'inscrire des �tudiants dans de groupes de soir). Le SIG doit
accepter la m�me information dans le fichier exportes horaire et
�tudiants. Avant les groupes �taient A, B, C \ldots. un seul
\verb!char!.


Dans \diamant nous avons pr�vu de traiter les groupes comme une
cha�ne des \verb!char!s, ainsi pour effectuer cette modification,
on doit simplement v�rifier que les donn�es arrivent � la bonne
place et que les automates de reconnaissance des fichiers lisent
les donn�es avec le nouveau format et que le nom d'un groupe
s'affiche correctement par tout. Bien entendu il faut v�rifier que
l'exportation est faite avec les bons donn�es et le bon format.

Quelles classes sont � modifier?

\subsection{Fonctionnalit� \ref{parti}~: Grille partielle}
La fonctionnalit� Grille partielle doit permettre de d�finir un
ensemble d'activit�s, et de donner un nom � cet ensemble. Ensuite
afficher sur la grille uniquement les cours de l'ensemble choisi.

Quelles classes sont � modifier?

Il faut aussi sauvegarder les donn�es nom de l'ensemble et les
activit�s appartenant dans l'ensemble.

\subsection{Fonctionnalit� \ref{select}~: Importation s�lective}
La fonctionnalit� importation s�lective est de pouvoir importer un
fichier sans changer les donn�es dans les autres fichiers.

Quels fichiers peuvent �tre import�s de mani�re selective?

Quelle sont les cas possibles~: donn�e nouvelle, donn�e modifie,
donn�e disparue, autres?

Quoi faire dans chaque cas?


Quelles classes sont � modifier?


\subsection{Fonctionnalit� \ref{menus}~: Menus}
Les menus doivent �tre disponibles (enabled) ou non disponibles
(disabled) en fonction de la derni�re op�ration effectu�.

Revision de toute l'interface avec l'utilisateur.

Quelles classes sont � modifier? \verb!DMenuBar! + toutes les
classes \verb!XCmd! + certaines classes avec dialogue
\verb!Annuler!

\subsection{Fonctionnalit� \ref{details}~: D�tails}
D�tails n'est pas une fonctionnalit�. Il s'agit de effectuer des
changements mineurs dans le logiciel pour assurer entre autres~:

\begin{enumerate}
    \item Les termes sont utilis�es de mani�re consistante.
    \item Les bo�tes de dialogue s'ajustent selon les informations
    � pr�senter.
    \item Autres ajustements, les utilisateurs vont les indiquer
\end{enumerate}

Revision de toute l'interface avec l'utilisateur.

Quelles classes sont � modifier?
\subsection{Fonctionnalit� \ref{toolbar}~: Toolbar}
La fonctionnalit� Toolbar est de pouvoir effectuer des
modifications � une grille horaire qui est d�j� associ�e � un
ensemble de donn�es. Ceci c'est le cas lorsque on travaille avec
une grille faite par  Nouvel horaire (cours ou examens).

Il s'agit de modifier la priorit� de p�riodes, de ajouter ou
enlever des journ�es. Modifier les noms de journ�es.

Quoi faire avec les donn�es d�j� assign�s dans la grille si la
p�riode est modifi�e ou �limin�e.

 Quelles classes sont � modifier?


\subsection{Fonctionnalit� \ref{event}~: Corriger dialogue de liste d'�v�nements}

La fonctionnalit� est d'assurer que lorsque on n'applique pas les
modifications l'�v�nement reste  o� il est.

\subsection{Fonctionnalit� \ref{nostudents}~: sans �tudiants}

La fonctionnalit� permet de faire un horaire sans �tudiants.

Il faut un  nouveau fichier avec le nombre d'�tudiants par
activit� (�v�nement). Ensuite un fichier de faux �tudiants sera
cr�e pour rester avec le m�me traitement. Ainsi, tout doit marcher
sans changements.


\subsection{Fonctionnalit� \ref{rooms}~: affectation locaux}

La fonctionnalit� permet de assigner les locaux aux �v�nements
avec une utilisation optimal des locaux. Il est convenu que les
locaux seront class�es en une partition avec des ensembles
disjoints.

\[ p = \{ e_0 \cup \ldots \cup e_i \cup \ldots \cup e_n \}
\]

o�  $e_0  \ldots \cup e_i  \ldots e_n$ sont disjoints.

�tapes :

\begin {enumerate}
\item
\end {enumerate}

\subsection{Fonctionnalit� \ref{affecIni}~: Automate pour menu Affectation initiale}

La fonctionnalit� forcer l'utilisateur � passer par un seul chemin
pour construire son horaire.

\subsection{Fonctionnalit� \ref{buttons}~: Remplacer les boutons de dialogues d'une mani�re unifi�e.}

La fonctionnalit� sert � utiliser la m�me terminologie pour une
m�me fonctionnalit�.

\subsection{Fonctionnalit� \ref{rapports}~: Rendre le rapport d'import visible une seule fois (lors de
creation d'un horaire)}

La fonctionnalit� forcer l'utilisateur � passer par un seul chemin
pour construire son horaire.

\subsection{Fonctionnalit� \ref{headers}~: Ent�te de rapports}

La fonctionnalit� garde toujours l'ent�te de rapports visible.

%\chapter{Description de la structure de donn�es de \diamant{}}


Chaque fichier contient une liste d'�l�ments qui constitue les
ressources dont le logiciel a besoin pour fonctionner. Ces
ressources sont classifi�es dans 4 cat�gories:
\begin{enumerate}
    \item Locaux.
    \item Instructeurs.
    \item �tudiants.
    \item Activit�s.
\end{enumerate}

Pour stocker l'information contenue dans chaque fichier, nous
avons d�velopp� pour la version 1.5 de \diamant{} la structure de
donn�es \emph{Resource} (pour plus de d�tails se reporter � la
section 1.1). Cette structure de donn�es contient les informations
d'un �l�ment d'une ressource donn�e (locaux, instructeurs,
�tudiants ou activit�s).

Le pr�sent document contient les exemples, les interpr�tations et
l'�num�ration des diff�rentes informations requises par \diamant{}
pour la construction de l'horaire.

 Le noeud principal de la
structure de donn�es de \diamant{} est repr�sent� par la classe
\emph{LoadData} (voir figure \ref{datastruct}). Cette classe
contient la liste des diff�rentes ressources, � savoir:

\begin{itemize}
    \item \emph{StudentsList} (voir figure \ref{datastruct}): elle repr�sente la structure de donn�es contenant la liste des �tudiants.\\
    \item \emph{RoomsList} (voir figure \ref{datastruct}): elle repr�sente la structure de donn�es contenant la liste des locaux.\\
    \item \emph{InstructorsList} (voir figure \ref{datastruct}): elle repr�sente la structure de donn�es contenant la liste d'instructeurs.\\
    \item \emph{ActivitiesList} (voir figure \ref{datastruct}): elle repr�sente la structure de donn�es contenant la liste d'activit�s.\\
\end{itemize}

Chacune des listes de ressources cit�es ci-dessus h�rite de la super classe \emph{ResourceList}. \emph{ResourceList} contient la liste de tous les �l�ments d'une ressource. Cette liste d'�l�ments est en r�alit� une liste d'objects de type \emph{Resource}. L'object \emph{Resource} d�crit quand � lui un �l�ment par sa cl� (\emph{resourceKey}), son identifiant (\emph{resourceID}) et la nature de cet object (\emph{resourceObject}). Il est � noter que \emph{resourceObject} peut �tre de type \emph{Instructor} ou \emph{Activity} ou \emph{Room} ou encore \emph{Student}

Nous d�crirons chacune des structures dans les prochaines sections.

\begin{figure}[h]
  % Requires \usepackage{graphicx}
  \begin{center}
    \includegraphics[width=430pt]{Images/loaddata.eps}
    \caption{Structure de donn�es}\label{datastruct}
  \end{center}
\end{figure}


\section{\emph{LoadData}}
Cette classe est charg�e de la gestion de deux taches principales
:
\begin{enumerate}
    \item \textbf{Le chargement de fichiers} :  Dans cette �tape, l'int�grit� des fichier des ressources est d'abord v�rifi�e et ensuite chaque fichier est charg� dans un vecteur de bytes pour leur post�rieur
    traitement. Cette t�che est effectu� � l'aide de la classe \emph{FilterFile} de la librairie
    \emph{com.iLib.gIO.FilterFile}.
    \item \textbf{La cr�ation des listes de ressources} : Dans cette �tape chaque liste de ressources est cr��e et peupl�e
    (remplie) avec les �l�ments correspondantes. Cette t�che est
    effectu�e en utilisant les classe d�crites � continuation.
\end{enumerate}

\section{\emph{ResourceList}}
Cette classe impl�mente la structure de donn�es
\emph{ResourceList}.  Elle d�clare les m�thodes
\emph{analyseTokens()} et \emph{buildResourceList()} qui
permettent de peupler la liste du ressource. Cette classe d�finie
les m�thodes n�cessaires pour ins�rer, �liminer, �diter, chercher
et trier les ressources.

\section{\emph{***List}}
Cet ensemble repr�sente les classes \emph{RoomsList},
\emph{InstructorsList}, \emph{ActivitiesList},
\emph{StudentsList}.  Ces classes h�ritent de la super classe
\emph{ResourceList} et, par cons�quent, elles impl�mentent les
m�thodes \emph{analyseTokens()} et \emph{buildResourceList()}.

Comme un cas sp�cial, la classe \emph{StudentsList} a son propre
impl�mentation des m�thodes \emph{removeStudent} et
\emph{removeStudent} car dans ce cas, il faut valider l'existence
de courses de choix de l'�tudiant avant d'ins�rer un nouveau
ressource � la liste et l'existence . \textbf{(V�rifier cette
phrase avec Yannick, pour quoi removeStudent??)}.

\section{\emph{Resource}}
Cette classe impl�mente la structure de donn�es \emph{Resource}.
Elle d�finie les m�thodes n�cessaires pour ins�rer, �diter et
obtenir les champs de la ressource.

\section{\emph{Instructor}}
Cette classe repr�sente la disponibilit� d'un instructeur
appartenant au fichier d'instructeurs (Voir section 1.2). Elle
impl�mente les m�thodes pour g�rer (ins�rer, �liminer, �diter,
obtenir) la disponibilit� d'un instructeur.

\section{\emph{Room}}
Cette classe repr�sente l'information d'un local appartenant au
fichier de locaux (Voir section 1.4).  Elle impl�mente les
m�thodes pour g�rer (ins�rer, �liminer, �diter, obtenir) les
diff�rents propri�t�s d'un local, cela inclut sa disponibilit�.

\section{\emph{Student}}
Cette classe repr�sente l'information d'un �tudiant appartenant au
fichier d'�tudiants (Voir section 1.1).  Elle impl�mente les
m�thodes pour g�rer (ins�rer, �liminer, �diter, obtenir) les
diff�rents propri�t�s d'un �tudiant, cela inclut se choix de
cours.

\section{\emph{Activity}}
Cette classe repr�sente l'information d'une activit� appartenant
au fichier d'activit�s (Voir section 1.3).  Elle impl�mente les
m�thodes pour g�rer (ins�rer, �liminer, �diter, obtenir) les
diff�rents propri�t�s d'une activit�, cela inclut ....\textbf(�
finir)


%\part{Une partie}
%\include{revue}
%\include{theorie}

%\part{Derni�re partie}
%\chapter{Conclusion}

Bla bla bla bla bla.

Bla bla bla bla bla.


%\appendix
%\chapter{Description des champs pour \diamant{} 1.5}

\begin{table}[h]
    \begin{tabular}{*{5}{|c}|} \hline
           \itshape Champ & \itshape �l�ment Diamant & \itshape  Description& \itshape Type & \itshape Genre\\ \hline
          Nom du local & Liste de locaux & - & A & \\ \cline{1-4}
          & Liste de locaux & - & A & \\ \cline{2-4}
          & Conflit de capacit� de locaux & Recalculer les conflits & C & \\ \cline{2-4}
          \raisebox{3.0ex}[1pt]{Capacit�} & P�riode & Rafra�chissement de la p�riode & C & \\ \cline{1-4}
          & Liste de locaux & - & A & \\ \cline{2-4}
          &  & V�rifier si les caract�ristiques & & \\
          &  & du local sont adapt�es & & \\
          \raisebox{4.5ex}[1pt]{Liste de} & \raisebox{3.0ex}[1pt]{Warning de locaux} & � la nature du cours & \raisebox{3.0ex}[1pt]{C} &  \\ \cline{2-4}
          \raisebox{4.5ex}[1pt]{caract�ristiques} & P�riode & Rafra�chissement de la p�riode & C & \raisebox{12.0ex}[1pt]{D} \\ \hline
    \end{tabular}
    \caption{\emph{Fichier de locaux}}
    \label{tableLocaux}
\end{table}


\begin{table}[h]
    \begin{tabular}{*{5}{|c}|} \hline
          \itshape Champ & \itshape �l�ment Diamant & \itshape Description & \itshape Type & \itshape Genre\\ \hline
          Instructeur ID(nom) & Liste de professeurs & - & A & S \\ \hline
          & Liste de disp. de profs. & - & A & \\ \cline{2-4}
          & Conflit de disp. de profs. & Recalculer les conflits & C & \\ \cline{2-4}
          \raisebox{3.0ex}[1pt]{Disponibilit�} & P�riode & Rafra�chissement de la p�riode & C & \raisebox{3.0ex}[1pt]{D} \\\hline
    \end{tabular}
    \caption{\emph{Fichier de Enseignants}}
    \label{tableEnseignants}
\end{table}

\begin{table}[h]
    \begin{tabular}{*{5}{|c}|} \hline
          \itshape Champ & \itshape �l�ment Diamant & \itshape Description & \itshape Type & \itshape Genre\\ \hline
          Matricule & & - & & S \\ \cline{1-1} \cline{3-3} \cline{5-5}
          Nom et pr�nom & & - & & S \\ \cline{1-1} \cline{3-3} \cline{5-5}
          Sexe & & - & & D \\ \cline{1-1} \cline{3-3} \cline{5-5}
          �tat & \raisebox{4.0ex}[1pt]{Liste d'�tudiants} & - & \raisebox{4.0ex}[1pt]{A} & D \\ \hline
          �tudiant.Activit�ID & & - & & \\ \cline{1-1} \cline{3-3}
          �tudiant.Activit�Nature & \raisebox{1.0ex}[1pt]{Conflits d'�tudiants et} & - & &  \\ \cline{1-1} \cline{3-3}
          �tudiant.Num�roGroupe & \raisebox{1.0ex}[1pt]{Conflits de capacit� de locaux} & - & \raisebox{3.0ex}[1pt]{C} & \raisebox{3.0ex}[1pt]{D} \\ \cline{1-1} \hline
    \end{tabular}
    \caption{\emph{Fichier d'�tudiants}}
    \label{tableEtudiants}
\end{table}


\begin{table}[h]
    \begin{tabular}{*{5}{|c}|} \hline
          \itshape Champ & \itshape �l�ment Diamant & \itshape Description & \itshape Type & \itshape Genre\\ \hline \hline
          Nom d'activit� & Liste d'activit�s & - & A & S \\ \hline
          & Liste d'activit�s & - & A & \\ \cline{2-4}
          & Conflits de locaux & Recalculer les conflits & & \\ \cline{2-2}
          & Conflits de disp. de profs. & si les activit�s sont d�j� & & \\ \cline{2-2}
          \raisebox{4.5ex}[1pt]{Include} & Conflit d'�tudiants & plac�es dans la grille & \raisebox{3.0ex}[1pt]{C} & \raisebox{4.5ex}[1pt]{D} \\ \hline
          Session & Liste d'activit�s & - & A & D \\ \hline
          & Liste de sous-activit�s & - & A & \\ \cline{2-4}
          & Liste d'activit�s & - & A & \\ \cline{2-4}
          & & Dans le cas o� il y a eu un & & \\
          & \raisebox{1.5ex}[1pt]{Warning de locaux} & changement de nature & \raisebox{1.5ex}[1pt]{C} & \\ \cline{2-4}
          & & Dans le cas o� il y a eu une & & \\
          \raisebox{7.0ex}[1pt]{Sous-Activit�.Nature} & \raisebox{1.5ex}[1pt]{Tous les conflits} & insertion/suppression de nature & \raisebox{1.5ex}[1pt]{C} & \raisebox{7.5ex}[1pt]{D} \\ \hline
          & Liste de groupes & - & & \\ \cline{2-3}
          & Liste de sous-activit�s & - & & \\ \cline{2-3}
          & Liste d'activit�s & - & \raisebox{3.0ex}[1pt]{A} &  \\ \cline{2-4}
          & & Dans le cas o� il y a eu une & & \\
          \raisebox{6.0ex}[1pt]{Groupe.Num�ro} & \raisebox{1.5ex}[1pt]{Tous les conflits} & suppression de groupe & \raisebox{1.5ex}[1pt]{C} & \raisebox{6.0ex}[1pt]{S} \\ \hline
          & Liste de bloques & - & & \\ \cline{2-3}
          & Liste de groupes & - & & \\ \cline{2-3}
          & Liste de sous-activit�s & - & & \\ \cline{2-3}
          & Liste d'activit�s & - & \raisebox{4.5ex}[1pt]{A} &  \\ \cline{2-4}
          & & Dans le cas o� il y a eu une & & \\
          \raisebox{6.0ex}[1pt]{Bloque.Num�ro} & \raisebox{1.5ex}[1pt]{Tous les conflits} & suppression de bloque & \raisebox{1.5ex}[1pt]{C} & \raisebox{7.5ex}[1pt]{S} \\ \hline
          & Toutes les listes & - & A & \\ \cline{2-4}
          & Conflit de locaux & - & C & \\ \cline{2-4}
          \raisebox{3.0ex}[1pt]{Bloque.LocalID} & Warning de locaux & - & C & \raisebox{3.0ex}[1pt]{D} \\ \hline
          & Toutes les listes & - & A & \\ \cline{2-4}
          \raisebox{1.5ex}[1pt]{Bloque.P�riodeID} & Tous les conflits & - & C & \raisebox{1.5ex}[1pt]{D} \\ \hline
          & Toutes les listes & - & A & \\ \cline{2-4}
          \raisebox{1.5ex}[1pt]{Bloque.Dure�} & Tous les conflits & - & C & \raisebox{1.5ex}[1pt]{D} \\ \hline
          & Toutes les listes & - & A & \\ \cline{2-4}
          \raisebox{1.5ex}[1pt]{Bloque.Activit�Type} & Warning de locaux & - & C & \raisebox{1.5ex}[1pt]{D} \\ \hline
          & Toutes les listes & - & A & \\ \cline{2-4}
          \raisebox{1.5ex}[1pt]{Bloque.Plac�} & Tous les conflits & - & C & \raisebox{1.5ex}[1pt]{D} \\ \hline
          & Toutes les listes & - & A & \\ \cline{2-4}
          \raisebox{1.5ex}[1pt]{Bloque.Fig�} & Tous les conflits & - & C & \raisebox{1.5ex}[1pt]{D} \\ \hline
          & Toutes les listes & - & A & \\ \cline{2-4}
          \raisebox{1.5ex}[1pt]{Bloque.Instructeur} & Conflit d'instructeur & - & C & \raisebox{1.5ex}[1pt]{D} \\ \hline
    \end{tabular}
    \caption{\emph{Fichier de Cours}}
    \label{tableCours}
\end{table}


  % after \\: \hline or \cline{col0-col2} \cline{col3-col4} .

%\chapter*{Glossaire}

Bla bla bla bla bla.

Bla bla bla bla bla.

%\chapter*{Index}

A
AAA
ABC.

B
Bla bla bla bla bla.

%\include{formules}
%\end{articleDX}
\end{document}
