% -*- TeX -*- -*- FR -*-
\documentclass[francais,letterpaper]{uds-article}

%-----------------------------------------------------------------------------
%----- Identification des packages n�cessaires
%-----------------------------------------------------------------------------

\usepackage{babel}
\usepackage[latin1]{inputenc}
%\usepackage{udstitle,dfd}
%\newcommand{\diamant}{Diamant}
\setlength{\oddsidemargin}{0.25in}
\setlength{\evensidemargin}{0.25in}
\setlength{\textwidth}{6.0in}
%\setlength{\parskip}{0.2in}
\newcounter{auxcounter}
\renewcommand{\baselinestretch}{1.5}
\setlength{\parskip}{1.5ex plus0.5ex minus0ex}

\newcommand{\ints}{\renewcommand{\baselinestretch}{1.0}\small \normalsize}
\newcommand{\intm}{\renewcommand{\baselinestretch}{1.5}\small \normalsize}
\newcommand{\intd}{\renewcommand{\baselinestretch}{2.0}\small \normalsize}

\newcommand{\bi}{\begin{itemize}}
\newcommand{\ei}{\end{itemize}}
\newcommand{\be}{\begin{enumerate}}
\newcommand{\ee}{\end{enumerate}}
\newcommand{\bd}{\begin{description}}
\newcommand{\ed}{\end{description}}



\newcommand{\bv}{\verb}

\newcommand{\bve}{\verb*}

\newcommand{\brun}{\noindent $\triangleright$}
\newcommand{\erun}{$\triangleleft$}

\newcommand{\ang}{\textsf}
\newcommand{\key}{\textsf}
\newcommand{\ita}{\textit}
\newcommand{\bld}{\textbf}
\newcommand{\dos}{\textsc}
\newcommand{\pro}{\texttt}

\newcommand{\diamant}{DIAMANT}
\newcommand{\dia}{DIAMANT 1.0}
\newcommand{\dx}{DIAMANT 1.5}
\newcommand{\saphir}{SAPHIR}
\newcommand{\sig}{SIG}

%-----------------------------------------------------------------------------
%----- Page Titre
%-----------------------------------------------------------------------------

\Titre{DX\\
Chargement et traitement de donn�es � partir\\
des fichiers textes}
\Logo{Images/logoDX.eps}
\Auteurs{Yannick Syam,\\
Alexander Jaramillo  et\\
Ruben Gonzalez-Rubio \\
 Mis-\`{a}-jour : 15 juillet 2003 } \Date{\today}

%-----------------------------------------------------------------------------
%----- Identification des fichiers des pages pr�liminaires et bibliographique
%-----------------------------------------------------------------------------

\FichierResume{Inputs/resume}
\FichierRemerciements{}
\FichierGlossaire{} % \FichierLexique est �quivalent
\FichiersBibliographie{udsplain}{Inputs/bibDiamant,Inputs/bib2}

%-----------------------------------------------------------------------------
%----- Le document
%-----------------------------------------------------------------------------

\includeonly{Inputs/resume,Inputs/intro,Inputs/glossaire,Inputs/index
}

\begin{document}
\begin{articleDX}

\chapter{Description des fichiers d'entr�e de  \diamant{}}

Les  versions 1.0 et 1.5 de \diamant{} utilisent les fichiers
suivantes~:

\begin{enumerate}
    \item �tudiants.
    \item Instructeurs.
    \item Activit�s.
    \item Locaux
\end{enumerate}

Les trois premiers sont produits au STI (Syst�me informatique
central) et transf�r�s via FTP.
Le fichier de locaux est produit
localement par l'utilisateur.

Tous les fichiers ont un format tr�s
rigide. Le nombre de caract�res et leur position est � respecter
obligatoirement.

Les indexes de cha�nes de caract�res commencent � \verb!0!.



\section{Fichier d'�tudiants}

Ce fichier est un h�ritage de format \saphir{}.

\subsection{Exemple des donn�es qu'il contient}

\begin{verbatim}
001342
009008132035030720003LUPIEN MY05
CTB301101 GIS251102 GIS351102 GRH111101 GRH332101
009011991290000520021AUDET FRE05
CTB341101 FEC111102 FEC444101 GIS114101 MAR221107
009022232035010720003AUDET STE05
CTB443102 CTB451101 CTB513101 CTB563101 CTB613102
009027042035010720003VEILLEUX 05
CTB443101 CTB451102 CTB513102 CTB563101 CTB613102
009031242035010720003FAUCHER M05
\end{verbatim}

\subsection{Signification des donn�es}

Il existe trois types de lignes :

\begin{enumerate}
    \item nombre d'�tudiants dans le fichier;
    \item identification de l'�tudiant et nombre de cours qu'il prend;
    \item l'identification de cours que l'�tudiant suit.
\end{enumerate}

La structure du fichier :

\begin{enumerate}
\item La cha�ne du d�but du fichier $n$, \verb!001342!, nous indique
le nombre d'�tudiants contenus dans ce fichier. Il s'agit d'une ligne du premier type.
\item Ensuite il y a $n$ couples de lignes du type 2 et 3.
    \begin{enumerate}
    \item La ligne de type 2 contient des chiffres et des lettres correspondant au num�ro
d'identification unique de l'�tudiant, son nom et le nombre de cours qu'il suit.

    \item La ligne de type 3 contient le cours et �ventuellement le groupes
    \end{enumerate}
\end{enumerate}

Les d�tails de la ligne de type 2.

\begin{itemize}
      \item Le matricule de l'�tudiant qui correspond aux huit premiers caract�res
      du num�ro d'identification (Indices 0 � 7).
       Dans notre exemple, \verb!00900813! pour l'�tudiant
       \verb!LUPIEN! et \verb!00901199! pour l'�tudiante \verb!AUDET!.
      \item Le programme auquel l'�tudiant appartient correspond ensuite aux
      six caract�res suivants.
      Soit \verb!203503! dans le cas du premier �tudiant (Indices 8 � 13).
     \item L'instance du programme que l'�tudiant suit correspond aux deux
      caract�res suivants.
     Soit \verb!07! dans le cas du premier �tudiant (Indices 14 � 15).
      \item Des informations sur l'admission de l'�tudiant sont donn�es
      par les cinq caract�res suivants.
       Les quatre premiers correspondent � l'ann�e d'admission tandis
       que le cinqui�me indique le trimestre d'admission.
        Soit \verb!20003! dans le cas du premier �tudiant (Indices 16 � 20).
    \item Suivent ensuite neuf caract�res pour indiquer le nom de famille
    de l'�tudiant suivi d'un espace et de son pr�nom (coup� apr�s neuf espaces) (Indices 21 � 29).
    Soit, dans notre exemple, \verb!LUPIEN MY! dans le cas de la premi�re �tudiante.
    Dans certains cas le nom prend les neuf caract�res.
    \item Vient alors un chiffre de deux caract�res qui correspond en fait au nombre de cours que suit l'�tudiant.
    \verb!05! dans le cas du premier �tudiant (Indices 30 � 31).
\end{itemize}

Les d�tails de la ligne de type 3.

Elle contient la liste des activit�s suivies par
l'�tudiant s�par�es par un espace. Voici un exemple d'information correspondant � chaque
cours.

\begin{itemize}
\item Soit \verb!CTB301101! dans le cas du premier �tudiant.
Les 6 premiers caract�res (\verb!GEI441!) correspondent au num�ro
du cours. Le septi�me caract�re (\verb!1!) correspond � la nature ou type d'activit�
du cours (1=Le�on Magistrale et 2=autre). Le 2 derniers caract�res
(\verb!01!) sont facultatifs dans un fichier et ils correspondent
au groupe d'activit� dans lequel l'�tudiant est assign�.
\end{itemize}

Un fichier peut avoir le groupe d'activit� pour certains �tudiants, Facult� d'administration.

\subsection{D�finition  et utilisation des champs}

\pro{long int eMatricule} correspond aux indices 0 � 7 de la ligne de type deux.
Il est mis dans \verb!_resourceKey!.

\pro{String eNom} correspond aux indices 21 � 29 de la ligne de type deux.
Il est mis dans \verb!_resourceID!.

\pro{String eAuxField} correspond aux indices 8 � 20 de la ligne de type deux.
Il est mis dans \verb!_auxField!.

\pro{String eSelectedCourse} il peut avoir 1 � $n$ \pro{eSelectedCourse}, qui sont consid�r�s comme
une liste de cours, ceci correspond aux cours dans la ligne de type trois.
Il est mis dans \verb!_resourceAttach!.

\section{Fichier d'instructeurs}\label{instructor}

Ce fichier contient les donn�es de disponibilit�s des enseignants.
Ce fichier est aussi un h�ritage de format \saphir{}.

\subsection{Exemple des donn�es qu'il contient}


\begin{verbatim}
116
ATALLA, NOUREDDINE
1 1 1 1 5 5 5 5 5 5 5 5 5 5
5 5 5 5 5 5 5 5 5 5 5 5 5 5
5 5 5 5 5 5 5 5 5 5 5 5 5 5
5 5 5 5 5 5 5 5 5 5 5 5 5 5
1 1 1 1 5 5 5 5 5 5 5 5 5 5
BALLIVY, G�RARD
1 1 1 1 5 1 1 1 1 1 5 5 5 5
1 1 1 1 5 1 1 1 1 1 5 5 5 5
1 1 1 1 5 1 1 1 1 1 5 5 5 5
1 1 1 5 5 1 1 1 1 5 5 5 5 5
1 1 1 1 5 1 1 1 5 5 5 5 5 5
\end{verbatim}


\subsection{Signification des donn�es}

Il existe trois types de lignes :

\begin{enumerate}
    \item le nombre d'enseignants dans le fichier;
    \item l'identification de l'enseignant;
    \item disponibilit� d'une journ�e.
\end{enumerate}

La structure du fichier :

\begin{enumerate}
\item La cha�ne du d�but du fichier $n$, \verb!116!, nous indique
le nombre d'enseignants contenus dans ce fichier. Il s'agit d'une ligne du premier type.
\item Ensuite il y a $n$ couples de lignes une de type 2 et cinq de type 3.
    \begin{enumerate}
    \item La ligne de type 2 est une cha�ne de caract�res correspondant nom et pr�nom
    de l'enseignant. La cha�ne ne doit pas exc�der trente caract�res.
    \item La ligne de type 3 d�crit la disponibilit� de l'enseignant sur une journ�e et comporte quatorze entiers s�par�s par
    un espace.
    \end{enumerate}
\end{enumerate}

Les d�tails de la ligne de type 3.

Elle contient la liste des disponibilit�s d'un enseignant sur une journ�e et comporte quatorze entiers (0 ou 1) s�par�s par un espace.

Chacune des cinq lignes de type 3 repr�sente une journ�e, en commen�ant par lundi, jusqu'� vendredi. Chaque colonne correspondant � un entier (0 ou 1) repr�sente une heure dans la journ�e (8h30, 9h30, 10h30, 11h30, 12h30, 13h30, 14h30, 15h30, 16h30, 17h30,  18h30, 19h00, 20h00 et 21h00). Les disponibilit�s sont ainsi connues entre 8h30 et 21h00.

\subsection{D�finition  et utilisation des champs}

\pro{String eNom} correspond aux nom et pr�nom de la ligne de type deux.
Il est mis dans \verb!_resourceID!.

\pro{String eVailability} correspond � la liste des disponibilit�s des lignes de type trois.
elle est mise dans \verb!_resourceAttach!.

\section{Fichier d'activit�s}

Ce fichier contient des informations concernant les cours offerts � une session. Nous y retrouvons la liste des
activit�s associ�es, l'agencement du cours, le nom de l'enseignant qui le donnera, etc.
C'est un fichier pr�alable � la construction de l'horaire.

\subsection{Exemple des donn�es qu'il contient}


\begin{verbatim}

ADM1111  A
1
1
LUC LAJOIE


 2
 2 1
 2 1 4 2
1 1
C1-387 C1-380
0 0
0 0
0 0

ADM1111  B
1
1
R�AL CAOUETTE


 1
 3
 2 12
1
C1-387
0
0
0

AMC6401  A
1
1
FADI AL-HAMED


 2
 2 1
 2 1 4 2
1
D73020
0
0
0
\end{verbatim}


\subsection{Signification des donn�es}

Il existe treize types de lignes :

\begin{enumerate}
    \item la ligne vide;
    \item l'identification du cours;
    \item l'�tat du cours (actif ou inactif);
    \item le nombre d'activit�s associ�es au cours;
    \item le nom de l'enseignant du cours;
    \item le nombre d'unit�s de p�riodes s�par�es (exp: une unit� de 2h et une autre de 1h);
    \item la dur�e de chacune des unit�s pr�vues � la ligne pr�c�dente;
    \item le jour et l'heure de d�but de chacune des unit�s d�crites � la ligne pr�c�dente;
    \item l'�tat du local affect� � chaque unit� (fix� ou non fix�);
    \item le nom du local assign� � chaque unit�;
    \item le type de locaux requis � chaque unit�;
    \item le type de locaux requis � chaque unit�;
    \item �tat d'affectation de chaque unit� dans la grille horaire.
\end{enumerate}

La structure du fichier :


\begin{enumerate}
\item Le fichier commence par une ligne vide, il s'agit d'une ligne du premier type.
\item Ensuite il y a $n$ couples de lignes allant du type 2 au type 13.
    \begin{enumerate}
    \item La ligne de type 2 est une cha�ne de caract�res correspondant au sigle
    du cours (\verb!ADM1111 A!).
    Ceci d�crit la le�on magistrale \verb!1!  du cours \verb!ADM111! avec le groupe \verb!A!).
    Dans l'exemple cette ligne contient \verb!ADM1111 A!
    \item La ligne de type 3 d�crit l'�tat du cours,
    c'est une valeur enti�re (0 si le cours est inactif et 1 dans l'autre cas).
    Dans l'exemple cette ligne contient \verb!1!
    \item La ligne de type 4 d�crit le nombre d'activit�s associ�es au cours, c'est une valeur enti�re.
    Dans l'exemple cette ligne contient \verb!1!
    \item La ligne de type 5 est une cha�ne de caract�res correspondant au nom de l'enseignant.
    Dans l'exemple cette ligne contient \verb!LUC LAJOIE!
    \item La ligne de type 6  le nombre de blocs de p�riodes s�par�es de la premi�re fiche
    (ex: si nous avons deux heures de cours coll�es �
    trois endroits diff�rents dans l'horaire, l'entier sera 3).
    Dans l'exemple cette ligne contient \verb!2!
    \item La ligne de type 7. Nous avons une suite de nombres qui correspond
    � la dur�e de chacun des blocs pr�vus � la ligne pr�c�dente.
    Dans ce cas-ci, la dur�e du premier bloc est de deux p�riodes
    alors que la dur�e du deuxi�me est de seulement une p�riode.
    Dans l'exemple cette ligne contient \verb!2 1!
    \item La ligne de type 8,  On a une  liste d'entiers repr�sentant
    le jour et l'heure de chacun des blocs pr�c�dents.
    Par exemple, \texttt{2 1} nous indique que le premier bloc a lieu le mardi � 8h30
    alors que \texttt{4 2} nous apprends que le deuxi�me bloc est le jeudi � 9h30.
    Dans le exemple cette ligne contient \verb!2 1 4 2!
    \item La ligne de type 9.  Cette ligne nous indique si le local de cette activit� est fix� ou non.
    Effectivement, un \texttt{1} nous dit qu'il l'est alors qu'un \texttt{0} nous indique le contraire.
    Dans l'exemple cette ligne contient \verb!1 1!
    \item La ligne de type 10. Sur cette ligne, nous avons les nom des locaux pour le premier
    et le deuxi�me bloc respectivement.
    Dans l'exemple cette ligne contient \verb!C1-387 C1-380!
    \item La ligne de type 11. Puis vient le type de locaux requis par cette activit�
    (voir le manuel d'utilisation de \saphir{} \cite{ruben94}).
    Dans l'exemple cette ligne contient \verb!0 0!
    \item La ligne de type 12. Idem (voir le manuel d'utilisation de \saphir{} \cite{ruben94}).
    Dans l'exemple cette ligne contient \verb!0 0!
    \item La ligne de type 13. Finalement, cette rang�e permet de savoir si le bloc � �t�
    pr�-affect� � l'horaire \texttt{(1)} ou non \texttt{(0)}.
    Dans le'exemple cette ligne contient \verb!0 0!
    \end{enumerate}
\end{enumerate}

Ensuite le fichier continue.

\section{Fichier de locaux}\label{room}

\subsection{Exemple des donn�es qu'il contient}

\begin{verbatim}
//Facult� des sciences;

//Nom du local;Capacit�;Fonction;Liste des caract�ristiques;Notes;

D13012;32;211;08,11,14,57;laboratoire de chimie;
D13013;40;211;08,11,57;laboratoire de chimie;
D13014;20;211;08,44,57;laboratoire de chimie;
D13016;18;211;08,57;laboratoire de chimie;
\end{verbatim}

\subsection{Signification des donn�es}

\begin{description}
    \item[Premi�re ligne] cette ligne repr�sente la facult�.
    \item[Deuxi�me ligne] cette ligne d�crit le fichier.
    \item[Troisi�me ligne] cette ligne est blanche.
    \item[Quatri�me ligne] cette ligne d�crit l'organisation des informations sur les locaux.
    \item[Cinqui�me ligne] cette ligne est blanche
    \item[De la sixi�me � la derni�re ligne ] chaque ligne donne les informations sur un local. Chaque information est s�par�e par un \texttt{;} et est d�crite comme suit:
    \begin{itemize}
        \item Le nom du local \texttt{D13012} pour le premier local.
        \item La capacit� du local \texttt{32} pour le premier local.
        \item La fonction du local \texttt{211} pour le premier local (voir le manuel d'utilisation de \saphir{} \cite{anoyme94})
        \item Liste des caract�ristiques 08,11,14,57 pour le premier local (voir le manuel d'utilisation de \saphir{} \cite{anoyme94}). Il peut y avoir plusieurs caract�ristique, elles seront s�par�es par des \texttt{,}
        \item La description du local \texttt{laboratoire de chimie} pour le premier local.
    \end{itemize}
\end{description}

\chapter[Description du fichier $.dia$ de sauvegarde]{Description du fichier $.dia$ de sauvegarde de projet de \diamant{} 1.5}

La sauvegarde d'un projet sous \dx{} se fait essentiellement �
travers un fichier $.dia$. Dans ce fichier sont stock�s diff�rents
types d'information du projet. Les types sont s�par�s entre eux
par le s�parateur suivant~:

\verb!=================================!

 Ce fichier comporte six
types de donn�es structur�s comme suit~:


\begin{enumerate}
    \item Version du logiciel,
    \item Emplacement de la grille horaire utilis�e,
    \item Donn�es relatives aux enseignants,
    \item Donn�es relatives aux locaux,
    \item Donn�es relatives aux activit�s,
    \item Donn�es relatives aux �tudiants.
\end{enumerate}

\section{Version du logiciel}
Le premier type de donn�es du fichier $.dia$ est une cha�ne de
caract�res sur une ligne indiquant la version du logiciel.

\section{Emplacement de la grille horaire}
Le second type de donn�es du fichier $.dia$ est une cha�ne de
caract�res indiquant l'emplacement du fichier $.xml$ contenant la
structure de la grille horaire.

\section{Donn�es relatives aux enseignants}
Le troisi�me type de donn�es du fichier $.dia$ repr�sente la
disponibilit� des enseignants, les donn�es sont d�crites dans la
section \ref{instructor}. Mais, dans la sauvegarde le nombre de
p�riodes de la disponibilit� est �gal au nombre de p�riodes de la
grille horaire.

\section{Donn�es relatives aux locaux}
Le quatri�me type de donn�es du fichier $.dia$ repr�sente les
caract�ristiques des locaux (voir section \ref{room} pour plus de
details), ainsi que leur disponibilit� (nouveaux champs rajout�s �
ceux d�crits � la section \ref{room}).

\subsection{Exemple des donn�es qu'il contient}

\begin{verbatim}
C2-1007;21;212;08,11,14,57;laboratoire de chimie;1 1 1 1 1 1 1 1 1
1 1 1, \\
1 1 1 1 1 1 1 1 1 1 1 1, 1 1 1 1 1 1 1 1 1 1 1 1, 1 1 1 1 1 1 1 1
1 1 1 1,  \\
1 1 1 1 1 1 1 1 1 1 1 1;  \\
C1-2018;10;212;11;Mat�riaux composites;1 1 1 1 1 1 1 1 1 1 1 1, 1
1 1 1 1 1 1 1 1 1 1 1, \\
1 1 1 1 1 1 1 1 1 1 1 1, 1 1 1 1 1 1 1 1 1 1 1 1, 1 1 1 1 1 1 1 1
1 1 1 1; \\
C1-3014;25;211;11;Laboratoire m�catronique;1 1 1 1 1 1 1 1 1 1 1
1, \\
1 1 1 1 1 1 1 1 1 1 1 1, 1 1 1 1 1 1 1 1 1 1 1 1, 1 1 1 1 1 1 1 1
1 1 1 1, \\
1 1 1 1 1 1 1 1 1 1 1 1; C1-3027;15;211;11;Petit laboratoire de
communication pour �lect; 1 1 1 1 1 1 1 1 1 1 1 1,1 1 1 1 1 1 1 1
1 1 1 1, \\
1 1 1 1 1 1 1 1 1 1 1 1, 1 1 1 1 1 1 1 1 1 1 1 1, 1 1 1 1 1 1 1 1
1 1 1 1;
\end{verbatim}

\subsection{Signification des donn�es}
Chaque ligne est compos�e de six types d'informations:
    \begin{itemize}
        \item Le nom du local \texttt{C2-1007} pour le premier local.
        \item La capacit� du local \texttt{21} pour le premier local.
        \item La fonction du local \texttt{212} pour le premier
        local.
        \item Liste des caract�ristiques 08,11,14,57 pour le premier local.
        Il peut y avoir plusieurs caract�ristiques, elles seront s�par�es par des \texttt{,}
        \item La description du local \texttt{laboratoire de chimie} pour le premier local.
        \item La disponibilit� du local. Une s�quence du genre \verb!1 1 1 1 1 1 1 1 1 1 1 1!
        repr�sente la disponibilit� du local durant une journ�e.
        Le s�parateur virgule \verb!,! s�pare la disponibilit� d'une journ�e � celle d'une autre. Le nombre de
p�riodes de la disponibilit� est �gal au nombre de p�riodes d'une journ�e de la
grille horaire.
    \end{itemize}

\section{Donn�es relatives aux activit�s}

Le cinqui�me type de donn�es du fichier $.dia$  contient des
informations concernant les cours offerts � une session. Nous y
retrouvons la liste des activit�s associ�es, l'agencement du
cours, le nom de l'enseignant qui le donnera, etc.

\subsection{Exemple des donn�es qu'il contient}


\begin{verbatim}
AMC6552  01
1
1
NEAIME, SAMIR F. ;
1
3
3.3.1
0
------
1
1
1 ;1
AMC9001  01
1
1
ST-AMANT, REN� : NEAIME, SAMIR F. : RGR; ST-AMANT, REN� ;
2
3 3
3.1.1 3.2.1
1 1
C1-5012 C1-5012
0 0
0  0
1 1 ;1 1
\end{verbatim}


\subsection{Signification des donn�es}

Il existe douze types de lignes :

\begin{enumerate}
    \item L'identification du cours;
    \item L'�tat du cours (actif ou inactif);
    \item Le nombre d'activit�s associ�es au cours;
    \item Le bloc d'enseignants du cours. 1 bloc d'enseignants = un ou plusieurs enseignants s�par�s par �~:~�. Si l'activit� a plus d'un bloc, il y'aura autant de blocs d'enseignants qu'il y'aura de blocs et les noms des enseignants seront s�par�s par des �~;~�.\\
\textbf{Exemple:}\\
L'activit� \verb!AMC9001  01! poss�de deux bloc. l'information sur les enseignants est repr�sent�e par la ligne: \\ \verb!ST-AMANT, REN� : NEAIME, SAMIR F. : RGR; ST-AMANT, REN� ;!\\
Ceci indique que le premier bloc de l'activit� est donn� par les enseignants \verb!ST-AMANT, REN�! et \verb! NEAIME, SAMIR F.!  et \verb! RGR!; tandis que le second bloc est donn� par l'enseignant \verb!ST-AMANT, REN�!

    \item Le nombre de blocs d'activit�;
    \item La dur�e de chacun des blocs identifi�s � la ligne pr�c�dente;
    \item La p�riode � laquelle le ou les blocs sont plac�s. La p�riode est sous la forme $a.b.c$ o� $a$ repr�sente le jour, $b$ repr�sente la s�quence et $c$ repr�sente la p�riode;
    \item l'�tat du local affect� � chaque unit� (fix� ou non fix�);
    \item le nom du local assign� � chaque unit�;
    \item le type de locaux requis � chaque unit�;
    \item le type de locaux requis � chaque unit�;
    \item Cette ligne poss�de deux types d'informations s�par�es par le �~;~�. Le premier type d'information indique l'�tat fig� (1) ou non (0) du bloc, tandis que le second type indique l'�tat plac� (1) ou non (0) du bloc.
\end{enumerate}

\section{Donn�es relatives aux �tudiants}

Le sixi�me type du fichier $.dia$ repr�sente les choix de cours des �tudiants.

\subsection{Exemple des donn�es qu'il contient}

\begin{verbatim}
004
022515472365000420031MERCIER B06
GEL410101;0 GEN400101;0 GIF400101;0 GIF420101;0 GIF430101;0 GIF440101;0
022519582145000520031BERGERON 06
GEL400101;0 GEL410101;0 GEL420101;0 GEL430101;0 GEL440101;0 GEN400101;0
915965442120000520023MARTINEAU05
GCH106101;0 GCH106201;0 GCH205101;0 GCH205201;0 GCH210101;0
965509672135000620012NGUYEN TH07
GCI200101;0 GCI200201;0 GCI220101;1 GCI220201;0 GCI600101;0
GCI600201;0 GIN600101;0
\end{verbatim}

\subsection{Signification des donn�es}

Il existe trois types de lignes :

\begin{enumerate}
    \item Nombre d'�tudiants dans le fichier;
    \item Identification de l'�tudiant et nombre de cours qu'il prend;
    \item L'identification de cours que l'�tudiant suit. Cette identification poss�de deux types d'informations s�par�es par le �~;~�.
    \begin{itemize}
    \item Le premier type, soit \verb!GEL410101! dans le cas du premier �tudiant.
    Les 6 premiers caract�res (\verb!GEL410!) correspondent au num�ro
    du cours. Le septi�me caract�re (\verb!1!) correspond � la nature ou type d'activit�
    du cours (1=Le�on Magistrale et 2=autre). Le 2 derniers caract�res
    (\verb!01!) correspondent au groupe d'activit� dans lequel l'�tudiant est assign�.
    \item Le second type indique si l'�tudiant est fig� (1) ou non (0) dans le groupe auquel il est assign�.
    \end{itemize}
\end{enumerate}

\chapter{Description de la structure de donn�es de \diamant{}}


Chaque fichier contient une liste d'�l�ments qui constitue les
ressources dont le logiciel a besoin pour fonctionner. Ces
ressources sont classifi�es dans 4 cat�gories:
\begin{enumerate}
    \item Locaux.
    \item Instructeurs.
    \item �tudiants.
    \item Activit�s.
\end{enumerate}

Pour stocker l'information contenue dans chaque fichier, nous
avons d�velopp� pour la version 1.5 de \diamant{} la structure de
donn�es \emph{Resource} (pour plus de d�tails se reporter � la
section 1.1). Cette structure de donn�es contient les informations
d'un �l�ment d'une ressource donn�e (locaux, instructeurs,
�tudiants ou activit�s).

Le pr�sent document contient les exemples, les interpr�tations et
l'�num�ration des diff�rentes informations requises par \diamant{}
pour la construction de l'horaire.

 Le noeud principal de la
structure de donn�es de \diamant{} est repr�sent� par la classe
\emph{LoadData} (voir figure \ref{datastruct}). Cette classe
contient la liste des diff�rentes ressources, � savoir:

\begin{itemize}
    \item \emph{StudentsList} (voir figure \ref{datastruct}): elle repr�sente la structure de donn�es contenant la liste des �tudiants.\\
    \item \emph{RoomsList} (voir figure \ref{datastruct}): elle repr�sente la structure de donn�es contenant la liste des locaux.\\
    \item \emph{InstructorsList} (voir figure \ref{datastruct}): elle repr�sente la structure de donn�es contenant la liste d'instructeurs.\\
    \item \emph{ActivitiesList} (voir figure \ref{datastruct}): elle repr�sente la structure de donn�es contenant la liste d'activit�s.\\
\end{itemize}

Chacune des listes de ressources cit�es ci-dessus h�rite de la super classe \emph{ResourceList}. \emph{ResourceList} contient la liste de tous les �l�ments d'une ressource. Cette liste d'�l�ments est en r�alit� une liste d'objects de type \emph{Resource}. L'object \emph{Resource} d�crit quand � lui un �l�ment par sa cl� (\emph{resourceKey}), son identifiant (\emph{resourceID}) et la nature de cet object (\emph{resourceObject}). Il est � noter que \emph{resourceObject} peut �tre de type \emph{Instructor} ou \emph{Activity} ou \emph{Room} ou encore \emph{Student}

Nous d�crirons chacune des structures dans les prochaines sections.

\begin{figure}[h]
  % Requires \usepackage{graphicx}
  \begin{center}
    \includegraphics[width=430pt]{Images/loaddata.eps}
    \caption{Structure de donn�es}\label{datastruct}
  \end{center}
\end{figure}


\section{\emph{LoadData}}
Cette classe est charg�e de la gestion de deux taches principales
:
\begin{enumerate}
    \item \textbf{Le chargement de fichiers} :  Dans cette �tape, l'int�grit� des fichier des ressources est d'abord v�rifi�e et ensuite chaque fichier est charg� dans un vecteur de bytes pour leur post�rieur
    traitement. Cette t�che est effectu� � l'aide de la classe \emph{FilterFile} de la librairie
    \emph{com.iLib.gIO.FilterFile}.
    \item \textbf{La cr�ation des listes de ressources} : Dans cette �tape chaque liste de ressources est cr��e et peupl�e
    (remplie) avec les �l�ments correspondantes. Cette t�che est
    effectu�e en utilisant les classe d�crites � continuation.
\end{enumerate}

\section{\emph{ResourceList}}
Cette classe impl�mente la structure de donn�es
\emph{ResourceList}.  Elle d�clare les m�thodes
\emph{analyseTokens()} et \emph{buildResourceList()} qui
permettent de peupler la liste du ressource. Cette classe d�finie
les m�thodes n�cessaires pour ins�rer, �liminer, �diter, chercher
et trier les ressources.

\section{\emph{***List}}
Cet ensemble repr�sente les classes \emph{RoomsList},
\emph{InstructorsList}, \emph{ActivitiesList},
\emph{StudentsList}.  Ces classes h�ritent de la super classe
\emph{ResourceList} et, par cons�quent, elles impl�mentent les
m�thodes \emph{analyseTokens()} et \emph{buildResourceList()}.

Comme un cas sp�cial, la classe \emph{StudentsList} a son propre
impl�mentation des m�thodes \emph{removeStudent} et
\emph{removeStudent} car dans ce cas, il faut valider l'existence
de courses de choix de l'�tudiant avant d'ins�rer un nouveau
ressource � la liste et l'existence . \textbf{(V�rifier cette
phrase avec Yannick, pour quoi removeStudent??)}.

\section{\emph{Resource}}
Cette classe impl�mente la structure de donn�es \emph{Resource}.
Elle d�finie les m�thodes n�cessaires pour ins�rer, �diter et
obtenir les champs de la ressource.

\section{\emph{Instructor}}
Cette classe repr�sente la disponibilit� d'un instructeur
appartenant au fichier d'instructeurs (Voir section 1.2). Elle
impl�mente les m�thodes pour g�rer (ins�rer, �liminer, �diter,
obtenir) la disponibilit� d'un instructeur.

\section{\emph{Room}}
Cette classe repr�sente l'information d'un local appartenant au
fichier de locaux (Voir section 1.4).  Elle impl�mente les
m�thodes pour g�rer (ins�rer, �liminer, �diter, obtenir) les
diff�rents propri�t�s d'un local, cela inclut sa disponibilit�.

\section{\emph{Student}}
Cette classe repr�sente l'information d'un �tudiant appartenant au
fichier d'�tudiants (Voir section 1.1).  Elle impl�mente les
m�thodes pour g�rer (ins�rer, �liminer, �diter, obtenir) les
diff�rents propri�t�s d'un �tudiant, cela inclut se choix de
cours.

\section{\emph{Activity}}
Cette classe repr�sente l'information d'une activit� appartenant
au fichier d'activit�s (Voir section 1.3).  Elle impl�mente les
m�thodes pour g�rer (ins�rer, �liminer, �diter, obtenir) les
diff�rents propri�t�s d'une activit�, cela inclut ....\textbf(�
finir)


%\part{Une partie}
%\include{revue}
%\include{theorie}

%\part{Derni�re partie}
%\chapter{Conclusion}

Bla bla bla bla bla.

Bla bla bla bla bla.


\begin{figure}[h]
\centering{\includegraphics[width=1.0\textwidth]{Images/loadData_correspondance.eps}}
\caption{\emph{Correspondance entre les ressources et les
classes}} \label{matchingResClass}
\end{figure}

\appendix
\chapter{Description des champs pour \diamant{} 1.5}

\begin{table}[h]
    \begin{tabular}{*{5}{|c}|} \hline
           \itshape Champ & \itshape �l�ment Diamant & \itshape  Description& \itshape Type & \itshape Genre\\ \hline
          Nom du local & Liste de locaux & - & A & \\ \cline{1-4}
          & Liste de locaux & - & A & \\ \cline{2-4}
          & Conflit de capacit� de locaux & Recalculer les conflits & C & \\ \cline{2-4}
          \raisebox{3.0ex}[1pt]{Capacit�} & P�riode & Rafra�chissement de la p�riode & C & \\ \cline{1-4}
          & Liste de locaux & - & A & \\ \cline{2-4}
          &  & V�rifier si les caract�ristiques & & \\
          &  & du local sont adapt�es & & \\
          \raisebox{4.5ex}[1pt]{Liste de} & \raisebox{3.0ex}[1pt]{Warning de locaux} & � la nature du cours & \raisebox{3.0ex}[1pt]{C} &  \\ \cline{2-4}
          \raisebox{4.5ex}[1pt]{caract�ristiques} & P�riode & Rafra�chissement de la p�riode & C & \raisebox{12.0ex}[1pt]{D} \\ \hline
    \end{tabular}
    \caption{\emph{Fichier de locaux}}
    \label{tableLocaux}
\end{table}


\begin{table}[h]
    \begin{tabular}{*{5}{|c}|} \hline
          \itshape Champ & \itshape �l�ment Diamant & \itshape Description & \itshape Type & \itshape Genre\\ \hline
          Instructeur ID(nom) & Liste de professeurs & - & A & S \\ \hline
          & Liste de disp. de profs. & - & A & \\ \cline{2-4}
          & Conflit de disp. de profs. & Recalculer les conflits & C & \\ \cline{2-4}
          \raisebox{3.0ex}[1pt]{Disponibilit�} & P�riode & Rafra�chissement de la p�riode & C & \raisebox{3.0ex}[1pt]{D} \\\hline
    \end{tabular}
    \caption{\emph{Fichier de Enseignants}}
    \label{tableEnseignants}
\end{table}

\begin{table}[h]
    \begin{tabular}{*{5}{|c}|} \hline
          \itshape Champ & \itshape �l�ment Diamant & \itshape Description & \itshape Type & \itshape Genre\\ \hline
          Matricule & & - & & S \\ \cline{1-1} \cline{3-3} \cline{5-5}
          Nom et pr�nom & & - & & S \\ \cline{1-1} \cline{3-3} \cline{5-5}
          Sexe & & - & & D \\ \cline{1-1} \cline{3-3} \cline{5-5}
          �tat & \raisebox{4.0ex}[1pt]{Liste d'�tudiants} & - & \raisebox{4.0ex}[1pt]{A} & D \\ \hline
          �tudiant.Activit�ID & & - & & \\ \cline{1-1} \cline{3-3}
          �tudiant.Activit�Nature & \raisebox{1.0ex}[1pt]{Conflits d'�tudiants et} & - & &  \\ \cline{1-1} \cline{3-3}
          �tudiant.Num�roGroupe & \raisebox{1.0ex}[1pt]{Conflits de capacit� de locaux} & - & \raisebox{3.0ex}[1pt]{C} & \raisebox{3.0ex}[1pt]{D} \\ \cline{1-1} \hline
    \end{tabular}
    \caption{\emph{Fichier d'�tudiants}}
    \label{tableEtudiants}
\end{table}


\begin{table}[h]
    \begin{tabular}{*{5}{|c}|} \hline
          \itshape Champ & \itshape �l�ment Diamant & \itshape Description & \itshape Type & \itshape Genre\\ \hline \hline
          Nom d'activit� & Liste d'activit�s & - & A & S \\ \hline
          & Liste d'activit�s & - & A & \\ \cline{2-4}
          & Conflits de locaux & Recalculer les conflits & & \\ \cline{2-2}
          & Conflits de disp. de profs. & si les activit�s sont d�j� & & \\ \cline{2-2}
          \raisebox{4.5ex}[1pt]{Include} & Conflit d'�tudiants & plac�es dans la grille & \raisebox{3.0ex}[1pt]{C} & \raisebox{4.5ex}[1pt]{D} \\ \hline
          Session & Liste d'activit�s & - & A & D \\ \hline
          & Liste de sous-activit�s & - & A & \\ \cline{2-4}
          & Liste d'activit�s & - & A & \\ \cline{2-4}
          & & Dans le cas o� il y a eu un & & \\
          & \raisebox{1.5ex}[1pt]{Warning de locaux} & changement de nature & \raisebox{1.5ex}[1pt]{C} & \\ \cline{2-4}
          & & Dans le cas o� il y a eu une & & \\
          \raisebox{7.0ex}[1pt]{Sous-Activit�.Nature} & \raisebox{1.5ex}[1pt]{Tous les conflits} & insertion/suppression de nature & \raisebox{1.5ex}[1pt]{C} & \raisebox{7.5ex}[1pt]{D} \\ \hline
          & Liste de groupes & - & & \\ \cline{2-3}
          & Liste de sous-activit�s & - & & \\ \cline{2-3}
          & Liste d'activit�s & - & \raisebox{3.0ex}[1pt]{A} &  \\ \cline{2-4}
          & & Dans le cas o� il y a eu une & & \\
          \raisebox{6.0ex}[1pt]{Groupe.Num�ro} & \raisebox{1.5ex}[1pt]{Tous les conflits} & suppression de groupe & \raisebox{1.5ex}[1pt]{C} & \raisebox{6.0ex}[1pt]{S} \\ \hline
          & Liste de bloques & - & & \\ \cline{2-3}
          & Liste de groupes & - & & \\ \cline{2-3}
          & Liste de sous-activit�s & - & & \\ \cline{2-3}
          & Liste d'activit�s & - & \raisebox{4.5ex}[1pt]{A} &  \\ \cline{2-4}
          & & Dans le cas o� il y a eu une & & \\
          \raisebox{6.0ex}[1pt]{Bloque.Num�ro} & \raisebox{1.5ex}[1pt]{Tous les conflits} & suppression de bloque & \raisebox{1.5ex}[1pt]{C} & \raisebox{7.5ex}[1pt]{S} \\ \hline
          & Toutes les listes & - & A & \\ \cline{2-4}
          & Conflit de locaux & - & C & \\ \cline{2-4}
          \raisebox{3.0ex}[1pt]{Bloque.LocalID} & Warning de locaux & - & C & \raisebox{3.0ex}[1pt]{D} \\ \hline
          & Toutes les listes & - & A & \\ \cline{2-4}
          \raisebox{1.5ex}[1pt]{Bloque.P�riodeID} & Tous les conflits & - & C & \raisebox{1.5ex}[1pt]{D} \\ \hline
          & Toutes les listes & - & A & \\ \cline{2-4}
          \raisebox{1.5ex}[1pt]{Bloque.Dure�} & Tous les conflits & - & C & \raisebox{1.5ex}[1pt]{D} \\ \hline
          & Toutes les listes & - & A & \\ \cline{2-4}
          \raisebox{1.5ex}[1pt]{Bloque.Activit�Type} & Warning de locaux & - & C & \raisebox{1.5ex}[1pt]{D} \\ \hline
          & Toutes les listes & - & A & \\ \cline{2-4}
          \raisebox{1.5ex}[1pt]{Bloque.Plac�} & Tous les conflits & - & C & \raisebox{1.5ex}[1pt]{D} \\ \hline
          & Toutes les listes & - & A & \\ \cline{2-4}
          \raisebox{1.5ex}[1pt]{Bloque.Fig�} & Tous les conflits & - & C & \raisebox{1.5ex}[1pt]{D} \\ \hline
          & Toutes les listes & - & A & \\ \cline{2-4}
          \raisebox{1.5ex}[1pt]{Bloque.Instructeur} & Conflit d'instructeur & - & C & \raisebox{1.5ex}[1pt]{D} \\ \hline
    \end{tabular}
    \caption{\emph{Fichier de Cours}}
    \label{tableCours}
\end{table}


  % after \\: \hline or \cline{col0-col2} \cline{col3-col4} .

%\chapter*{Glossaire}

Bla bla bla bla bla.

Bla bla bla bla bla.

%\chapter*{Index}

A
AAA
ABC.

B
Bla bla bla bla bla.

%\include{formules}
\end{articleDX}
\end{document}
