\chapter{Pr�paration d'un horaire avec DIAMANT}

Nous d�taillerons ici les �tapes n�cessaires � la construction d'un horaire (cycle et examen). Toutefois,
les options du logiciel ne seront pas toutes pr�sent�es. Certaines �tapes sont les m�mes tant pour les horaires de cours que ceux d'examen, tandis que d'autres sont diff�rentes.
On suppose que l'utilisateur a d�j� sur son ordinateur les fichiers n�cessaires (ceux du STI et
ceux de la Facult�) : cours.sig, choixet.sig, disprof.sig et locaux.txt. Ces fichiers peuvent porter
n'importe quel nom, mais pour notre exemple, les noms sont bien explicites. Il est n�cessaire de
regrouper toutes les informations concernant les horaires d'un trimestre dans un r�pertoire (par
exemple hiver04 ou h2004). Ainsi, les fichiers d'importation et les fichiers associ�s � un horaire
seront ensemble.

\section{Horaire de cycle}

On suppose aussi qu'il s'agit de la pr�paration d'un nouvel horaire.

\begin{enumerate}
    \item Lancer DIAMANT.
    \item Aller au menu \textbf{\emph{Fichier}}, puis selectionner le menu \textbf{\emph{Nouvel horaire}} et enfin selectionner le menu \textbf{\emph{Horaire cycle}}. Une bo�te de dialogue comme celle
de la Figure xxxx doit appara�tre et elle vous permettra de choisir le fichier (fichier avec extension \emph{xml}) contenant la configuration de votre grille horaire.
    \item xx
    \item xx
    \item xx
\end{enumerate}    

\section{Horaire d'examen} 