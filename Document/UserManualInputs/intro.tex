\chapter{Introduction}

\dx{} est un logiciel servant � la construction d'horaires de
cours et d'examens. Ce manuel de l'utilisateur
pr�sente\footnote{Ce manuel est fait pour servir les utilisateurs,
donc nous avons commenc� par les exemples d'utilisation.}~:

\begin{itemize}
\item Une description du processus de la construction d'horaires (voir section \ref{horaire}).
     \item Un exemple d'utilisation de \dx{} pour la pr�paration d'un horaire de cours
      (voir section \ref{prepacours}),
      consistant � placer dans une grille horaire les activit�s comportant plusieurs natures,
       plusieurs groupes et plusieurs blocs.

    \item Un exemple d'utilisation de \dx{} pour la pr�paration d'un horaire
    d'examens (voir section \ref{prepaexam}), consistant � placer dans une grille horaire les activit�s
    comportant une seule nature, un seul groupe et un seul bloc.

    �tant donn� que les informations utilis�es � l'Universit� de Sherbrooke
    pour construire un horaire d'examen comportent plusieurs natures, plusieurs des groupes, ainsi que plusieurs blocs, la construction d'un horaire d'examen fait un traitement pour  conserver uniquement la nature 1 d'une activit�,
    fusionner tous les groupes d'une nature en un seul groupe et  conserver uniquement le bloc 1 du groupe.


%\item Une description des menus, des bo�tes de dialogue et des fen�tres;
%\item Une description des menus;

%\item Une description des bo�tes de dialogue;

%\item Une description des fen�tres;

%\item Guide d'installation.
\end{itemize}

Il s'agit de la version de \diamant{} telle que disponible en
mars 2004.
