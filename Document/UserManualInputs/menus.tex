\chapter{Description des menus, des dialogues et des fen�tres}

\section{DESCRIPTION DES MENUS}


\begin{description}
    \item[Fichier :] permet de faire appel aux fonctions classiques relatives aux fichiers : Nouveau
projet, Nouveau, Ouvrir, Fermer, Enregistrer, Enregistrer sous, Importer manuellement,
D�finir le fichier Import Auto, Importer automatiquement, Exporter et Quitter.
    \item[Affichage :] permet de faire appel aux fonctions n�cessaire � l'affichage des fen�tres associ�es � un projet d'horaire.
    \item[Affectation :] permet d'affecter (fixer) certains choix relatifs � la construction d'un horaire, par exemple, un �tudiant dans un groupe donn� ou une activit� � une p�riode d�termin�e, etc. Contient aussi les menus qui permettent la construction automatique optimis�e d'un horaire.
    \item[Rapport :] permet de faire appel aux fonctions qui donnent acc�s aux rapports et de les sauvegarder comme des fichiers � texte �.
    \item[Pr�f�rences :] permet le changement de certains param�tres du programme.
    \item[Fen�tre :] Permet d'ouvrir une fen�tre de listes de conflits pr�c�demment ouverte.
    \item[Aide :] permet d'obtenir de l'information sur le logiciel ainsi que de l'aide.
\end{description}    
 

%  
% 
% 
% 
% 