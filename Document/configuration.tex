\documentclass{llncs}
\bibliographystyle{alpha}
\usepackage[T1]{fontenc}
\usepackage{dfd,hhline}
\usepackage{verbatim}

\setlength{\parskip}{1.5explus0.5ex minus0ex}

\newcommand{\diamant}{DIAMANT}
\newcommand{\tictac}{Tic-Tac}
\newcommand{\jreplay}{jReplay}
\newcommand{\xp}{eXtreme Programming}
\newcommand{\checklist}{$\hspace{0.5ex}\bigodot{}$}
\def\eXit{$\epsilon$\kern-.100em \lower.5ex\hbox{X}\kern-.125emit}


\title{\bf{Configuration des PC's}}
\author{Rosa Lourdes Garcia Diaz}
\institute{\eXit \\D�partement de g�nie �lectrique et de g�nie informatique \\
Universit� de Sherbrooke,\\
Sherbrooke, Qu�bec, J1K 2R1 \\
Canada}
\date{}


\begin{document}

\maketitle
\section{Versions}

\begin{tabular}{|l|l|l|}
  \hline
  % after \\: \hline or \cline{col1-col2} \cline{col3-col4} ...
  Diamant 1.6 & eclipse & 3.1 \\
   & JRE  System Library & 1.5.0$\_$01\\
   & junit & 3.8.1 \\
   & log4j & 1.2.8 \\
  \hline
\end{tabular}
\section{Repertoires}

\subsection{Disque C:}
    \begin{description}
    \item Program Files
        \begin{description}
        \item Adobe
        \item Microsoft Office
        \item Windows NT
        \item Outlook Express
        ...
        \end{description}
    \item WINDOWS
    \item Documents and Settings
    \end{description}

\subsection{Disque D:}
    \begin{itemize}
    \item exitDev
        \begin{itemize}
        \item exitProjets
            \begin{itemize}
            \item DiamantExtreme (CVS)
            \item DiamantWeb (CVS)
            \item siteWeb (CVS)
            \item diamantBug (CVS)
            \end{itemize}
        \item exitGroupe
            \begin{itemize}
            \item � d�finir
            \item exit (CVS)
            \item membres (CVS)
            \end{itemize}
        \end{itemize}
    \item Local
        \begin{itemize}
        \item MiKTeX
        \item texmf-local (CVS)
        \end{itemize}

    \item Program Files
        \begin{itemize}
        \item eclipse
        \item Ghostgum
        \item gs
        \item Java
            \begin{itemize}
            \item j2re1.4.2$\_$03
            \item jdk1.5.0$\_$01
            \item jre1.5.0$\_$01
            \end{itemize}
        \item MiKTeX
        \item NetBeans3.5.1
        \item Tortoise
        \item SSH Communications Security
        \item WinEdt Team
        \item WinMerge
        \end{itemize}
    \end{itemize}

\section{Eclipse}

Eclipse doit s'installer dans D:/Programs Files/eclipse.

%\begin{enumerate}
%    \item Fermer la fen�tre \emph{\textbf{Problems}}
%    \item Faire \emph{\textbf{Refresh}} dans la fen�tre \emph{\textbf{Package
%    Explorer}}. Il aura un dialog pour �liminer le projet Dx et on doit r�pondre
%    \emph{\textbf{Yes}}.
%    \item Tout est pr�t pour faire \emph{\textbf{File->Import}} du
%    \textbf{D:\//Developpements\//DiamantExtreme}
%\end{enumerate}

\subsection{Configuration du Package Explorer}
    \begin{enumerate}
    \item Choisir la perspective java avec \textbf{\emph{Window -> Open Perspective -> Java}}
    \item S�lectionner le folder \textbf{\emph{Package Explorer}}
    \item Ouvrir la option \textbf{Menu} (c'est un petit triangle).
    \item Choisir \textbf{\emph{Filters}} et crocher les options
    suivants
        \begin{itemize}
        \item .*files
        \item Binary plug-in and feature projects
        %\item Closed projects
        \item Empty package
        \item Empty parents packages
        \item Fields
        \item Import declarations
        \item Inner class files
        \item Local Types
        \item Package declarations
        \item Static fields and methods
        \end{itemize}
    \end{enumerate}

\subsection{Configuration du compilateur}
    \begin{enumerate}
    \item Choisir \emph{\textbf{Windows -> Preferences }}
    \item Ouvrir \emph{\textbf{Java -> Compiler }}
    \item Compiler Compilance level \emph{\textbf{5.0}}
    \item Ouvrir \emph{\textbf{Errors/Warnings}} et v�rifier les options suivantes
        \begin{itemize}
        \item Code Style

         $\begin{array}{ll}
            \hline
            % after \\: \hline or \cline{col1-col2} \cline{col3-col4} ...
            Non static access to static member & Warning \\
            Indirect access to static member & Warning \\
            Unqualified access to instance field & Ignore \\
            Undocumented empty block & Ignore \\
            Access to a non-accesible member of enclosig type & Ignore \\
            Methods with a constructor name & Warning \\
            \hline
            \end{array}$

        \item Potentials programming problems

            $\begin{array}{ll}
            \hline
            % after \\: \hline or \cline{col1-col2} \cline{col3-col4} ...
            Serializable class without serialVersionUID  & Ignore \\
            Assignment has no effect & Warning \\
            Possible accidental boolean asignement (e.g.(if a=b)) & Warning \\
            'finally' does not complete normaly & Warning \\
            Empty statement & Warning \\
            using a char arrary in string concatenation & Warning \\
            Hidden catch block & Warning \\
            \hline
            \end{array}$

        \item Name shadowing and conflicts

            $\begin{array}{ll}
            \hline
            % after \\: \hline or \cline{col1-col2} \cline{col3-col4} ...
            Field declaration hides another field or variable & Ignore \\
            Local variable declaration hides another field or variable & Ignore \\
            Methods overridden but not package visible & Warning \\
            Interface methods conflicts with protected 'Object' method & Warning \\
            \hline
            \end{array}$

        \item Using deprecated API

            $\begin{array}{ll}
            \hline
            % after \\: \hline or \cline{col1-col2} \cline{col3-col4} ...
            Usage of deprecate API & Warning \\
            \hline
            \end{array}$

        \item String externalization

            $\begin{array}{ll}
            \hline
            % after \\: \hline or \cline{col1-col2} \cline{col3-col4} ...
            Usage of non-externalized string & Ignore \\
            \hline
            \end{array}$

        \item Unnecessary code

            $\begin{array}{ll}
            \hline
            % after \\: \hline or \cline{col1-col2} \cline{col3-col4} ...
            Local variable is never read & Warning \\
            Parameter is never read & Warning \\
            Check overriding and implementing methods & / \\
            Unused imports & Warning \\
            Unused or unread private members & Warning \\
            Unnecessary else statement & Warning \\
            Unnecessary cas or 'instanceof' operation & Warning \\
            Unnecessary declaration of thrown checked option & Warning \\
            checking overriden and implemented methods & / \\
            \hline
            \end{array}$

        \item JDK 5.0 options

            $\begin{array}{ll}
            \hline
            % after \\: \hline or \cline{col1-col2} \cline{col3-col4} ...
            Unsafe type operation involving raw types & Ignore \\
            Generic type parameter used with a final type bound & Ignore \\
            Inexact type match for vararg arguments & Warning \\
            \hline
            \end{array}$
        \end{itemize}
    \end{enumerate}

\section{WinEdt}

WinEdt doit s'installer dans D:/Programs Files/WinEdt.
\begin{enumerate}
    \item Faire \emph{\textbf{Checkout}} du module \textbf{texmf-local} dans
    \textbf{D:/Programs Files/Local}
    \item D�marrer WinEdt
    \item Enregistrer la licence. \emph{\textbf{Help->Register WinEdt}}
    \item Ouvrir \emph{\textbf{MikTeX Options}} (l'ic�ne avec deux pignons)
    \item Choisir le folder \emph{\textbf{Roots}}
    \item S�lectionner \textbf{\emph{Add}} et choisir le repertoire \textbf{D:/Programs Files/Local}
    \item Faire \emph{\textbf{Apply}} et \emph{\textbf{OK}}.
    \item WinEdt va prend les fichiers du module \textbf{texmf-local}
\end{enumerate}


\end{document}
