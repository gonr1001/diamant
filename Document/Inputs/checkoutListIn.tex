\chapter{\eXit{} Conventions de codage}
Ce document indique quels sont les conventions de codage utilis�s dans les projets \eXit{}.
Il est obligatoire de suivre ces conventions.


\section*{Une classe doit suivre les conventions suivantes}

\begin{enumerate}
    \item Le code doit suivre les  conventions Java \verb!http://java.sun.com/docs/codeconv!.

    \item Aucun warning Eclipse sera permis. (voir configuration d'Eclipse).

    \item Le code doit �tre lisible, pour ce faire, les m�thodes ne doivent pas d�passer 20 lignes,
    une ligne ne doit pas d�passer 80 caract�res , afin de produire une impression lisible.
    Sans sacrifier la taille de noms de variables ou m�thodes.

    \item Un encha�nement d'appel de m�thodes \verb!m1().m2().m3().m4! ne doit pas d�passer 4 m�thodes.

    \item Utiliser un bon nom pour indiquer \verb!package!. Il faut le mettre apr�s le commentaire
    g�n�ral de la classe.

    \item commentaire g�n�ral de la classe, bien fait et � la bonne place.

    \item commentaire g�n�ral doit contenir le nom de la classe.

    \item \verb!import!s, toujours avec le nom de la classe � importer (pas de \verb!*!).
    Ordonn�es lignes comportant \verb!java!, ligne vide,
    lignes comportant \verb!javax.swing!, ligne vide,
    lignes comportant \verb!lib!, ligne vide lignes comportant les classes de l'application.

    \item commentaire de description de la responsabilit� de la classe.

    \item donn�es membres, toujours \verb!private!, sinon justifier par un commentaire. Un
    commentaire sur leur fonction est appr�ci�.

    \item les donn�es membres commencent toujours par un \verb!_!

    \item Les noms de variables et m�thodes ....

    \item m�thode avec commentaires sur param�tres et return, indiquer toutes les exceptions que la m�thode
    peut jeter.
    \item une m�thode doit aussi contenir des commentaires requires, modifies, effets. requires indique les
    contraintes possibles; modifies indique toutes les entr�es
    modifi�es; et  effets d�finit le comportement.
    \item m�thode � red�finir dans la plus part de classes \pro{toString}.
    \item class de tests de toutes les m�thodes.
    \item \verb!set! et \verb!get! n'ont pas besoin de
    commentaires.

\end{enumerate}


\section{Cvs}
Il faut mettre l'option Preferences/Global Check out read only
sans coche, et ne pas utiliser Edit selection et Edit reserved
(les crayons dans la r�gle). Pour �viter le base!
