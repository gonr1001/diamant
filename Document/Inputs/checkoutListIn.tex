\chapter{\eXit{} Conventions de codage}
Ce document est un compl�ment de Java Code conventions
\section*{Checkout List pour une classe}



\begin{enumerate}
    \item bon nom du \verb!package!.
    \item code en accord avec convention Java.
    \item commentaire g�n�ral bien fait et � la bonne place.
    \item commentaire g�n�ral avec le nom de la classe.
    \item \verb!import!s, toujours avec le nom de la classe � importer (pas de \verb!*!).
    Ordonn�es lignes comportant \verb!java!, ligne vide,
    lignes comportant \verb!javax.swing!, ligne vide,
    lignes comportant \verb!lib!, ligne vide lignes comportant les classes de l'application.
    \item commentaire de description de la responsabilit� de la classe.
    \item donn�es membres, toujours \verb!private!, sinon justifier. Un
    commentaire sur leur fonction.
    \item les donn�es membres commencent toujours par un \verb!_!
    \item m�thode avec commentaires sur param�tres et return, indiquer toutes les exceptions que la m�thode
    peut jeter.
    \item une m�thode doit aussi contenir des commentaires requires, modifies, effets. requires indique les
    contraintes possibles; modifies indique toutes les entr�es
    modifi�es; et  effets d�finit le comportement.
    \item m�thode � red�finir dans la plus part de classes \pro{toString}.
    \item class de tests de toutes les m�thodes.
    \item \verb!set! et \verb!get! n'ont pas besoin de
    commentaires.
\end{enumerate}


\section{Cvs}
Il faut mettre l'option Preferences/Global Check out read only
sans coche, et ne pas utiliser Edit selection et Edit reserved
(les crayons dans la r�gle).
