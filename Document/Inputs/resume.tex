\chapter*{Introduction}

\diamant{} est un logiciel de construction d'horaire dans lequel le chargement de donn�es se fait � partir de 4 fichiers de texte:
\begin{enumerate}
    \item Le fichier de locaux.
    \item Le fichier d'instructeurs.
    \item Le fichier d'�tudiants.
    \item Le fichier d'activit�s.
\end{enumerate}    

Chaque fichier contient une liste d'�l�ments qui constitue les ressources dont le logiciel a besoin pour fonctionner. Ces ressources sont classifi�es dans 4 cat�gories:
\begin{enumerate}
    \item Locaux.
    \item Instructeurs.
    \item �tudiants.
    \item Activit�s.
\end{enumerate}    

Pour stocker l'information contenue dans chaque fichier, nous avons d�velopp� pour la version 1.5 de \diamant{} la structure de donn�es \emph{Resource} (pour plus de d�tails se reporter � la section 1.1). Cette structure de donn�es contient les informations d'un �l�ment d'une ressource donn�e (locaux, instructeurs, �tudiants ou activit�s).

L'information contenue dans un fichier est stock�e dans la structure de donn�es \emph{ResourceList}  (pour plus de d�tails se reporter � la section 1.2). ResourceList est un ensemble constitu� d'une liste d'�l�ments de type  Resource et contient par cons�quent la liste des �l�ments d'un fichier. 

Cette approche donne les avantages suivants :
\begin{itemize}
    \item Le traitement de tous les fichiers est fait avec la m�me structure, ce qui favorise la modularit�, la clart� de l'application et le traitement des donn�es de fa�on uniforme et centralis�e.
    \item On obtient moins de lignes de code, ce qui diminue les efforts de tests et de debbugage.
\end{itemize}    

Une fois les donn�es charg�s dans \diamant{}, une s�rie de  classes et de m�thodes sont charg�s du traitement de l'information (ajout, suppression, tri, recherche et �dition d'�l�ments).

Nous vous pr�senterons dans ce document, la description du format des fichiers, la structure de donn�es et les classes d�velopp�es.
