\chapter*{R�sum�}
Ce document pr�sente principalement comment  des
donn�es r�sidant dans des fichiers sont mises dans la forme
d'objets, afin d'�tre traites � l'int�rieur de l'application  \diamant{}.

\diamant{} est un logiciel de construction d'horaires, dans lequel
les donn�es de r�f�rence sont des fichiers de texte.
Ces fichiers contiennent des enregistrements qui seront
``transform�s'' en objets reconnus par l'application.
Ces objets seront �ventuellement modifi�s (uniquement certains champs changent de valeur)
suite � des actions de l'utilisateur. Tous les objets
seront sauvegard�s par l'application dans un seul fichier texte XML, ainsi
l'application peut continuer un travail en prenant la derni�re
sauvegarde faite.

Ce papier pr�sente les d�tails de formats de fichiers, fichiers originaux et
fichiers produits; avec des indications sur l'utilisation de
champs et les classes concernant cette partie de l'application ainsi que
la correspondance entre champs des enregistrements fichiers et donn�es membres.

Cette approche donne les avantages suivants :
\begin{itemize}
    \item Le traitement de tous les fichiers est fait en suivant la m�me structure,
    ce qui favorise la modularit�,
    la clart� de l'application et le traitement des donn�es de fa�on uniforme et centralis�e;
    \item R�duit la redondance dans le  code, ce qui diminue les efforts de tests et de d�bogage.
\end{itemize}
