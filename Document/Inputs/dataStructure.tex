\chapter{Description de la structure de donn�es de \diamant{}} 

Le noeud principal de la structure de donn�es de \diamant{} est repr�sent� par la classe \emph{LoadData} (voir figure \ref{datastruct}). Cette classe contient la liste des diff�rentes ressources, � savoir:

\begin{itemize}
    \item \emph{StudentsList} (voir figure \ref{datastruct}): elle repr�sente la structure de donn�es contenant la liste des �tudiants.\\
    \item \emph{RoomsList} (voir figure \ref{datastruct}): elle repr�sente la structure de donn�es contenant la liste des locaux.\\
    \item \emph{InstructorsList} (voir figure \ref{datastruct}): elle repr�sente la structure de donn�es contenant la liste d'instructeurs.\\
    \item \emph{ActivitiesList} (voir figure \ref{datastruct}): elle repr�sente la structure de donn�es contenant la liste d'activit�s.\\
\end{itemize}  

Chacune des listes de ressources cit�es ci-dessus h�rite de la super classe \emph{ResourceList}. \emph{ResourceList} contient la liste de tous les �l�ments d'une ressource. Cette liste d'�l�ments est en r�alit� une liste d'objects de type \emph{Resource}. L'object \emph{Resource} d�crit quand � lui un �l�ment par sa cl� (\emph{resourceKey}), son identifiant (\emph{resourceID}) et la nature de cet object (\emph{resourceObject}). Il est � noter que \emph{resourceObject} peut �tre de type \emph{Instructor} ou \emph{Activity} ou \emph{Room} ou encore \emph{Student}

Nous d�crirons chacune des structures dans les prochaines sections.

\begin{figure}[h]
  % Requires \usepackage{graphicx}
  \begin{center}
    \includegraphics[width=430pt]{Images/loaddata.eps}
    \caption{Structure de donn�es}\label{datastruct}
  \end{center}
\end{figure}


\section{\emph{LoadData}}

\section{\emph{StudentsList}}

\section{\emph{RoomsList}}

\section{\emph{InstructorsList}}

\section{\emph{ActivitiesList}}

\section{\emph{ResourceList}}

\section{\emph{Resource}}

\section{\emph{Instructor}}

\section{\emph{Room}}

\section{\emph{Student}}

\section{\emph{Activity}}
