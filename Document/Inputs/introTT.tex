\chapter{Introduction}

\diamant{} est un logiciel de construction d'horaires dans lequel la d�finition de la grille horaire peut se faire � partir d'un fichier standard ou alors se faire � partir de l'interface graphique.

La grille horaire peut �tre consid�r�e comme le coeur de la construction d'horaires, car c'est � travers elle que les diff�rentes ressources sont affect�es. Elle est compos�e de journ�es, Chaque journ�e �tant elle m�me compos�e de p�riodes \cite{ruben94}. 

La p�riode est l'entit� �l�mentaire de la grille horaire et elle est caract�ris�e par une heure de debut, une heure de fin et sa priorit�.

Cette organisation de la grille horaire nous parait un peu limit�e, car elle ne permet d'affecter les resources que de fa�on cyclique. En effet, pour construire les horaires de cours d'une universit� fonctionnant avec 13 semaines de cours, nous ne construisons que l'horaire sur une semaine et cet horaire devra �tre utilis� pour toutes les semaines.

Cette approche aurait �t� sans reproches si les contraintes appliqu�es � certaines resources �taient elles aussi cycliques, c'est � dire, si par exemple la disponibilit� d'un enseignant restait inchang�e tout au long des 13 semaines ou encore si un cours devait �tre donn�e par le m�me enseignant et dans le m�me local tout au long des 13 semaines.

Afin d'outre passer ce probl�me d'horaire par cycle, nous avons d�velopp� une nouvelle architecture de la grille horaire. Cette architecture prend en compte les contraintes appliqu�es � chaque resource durant chaque cycle (un cycle peut �tre consid�r� comme une semaine dans le d'horaires de cours) et permet de construire un horaire particulier pour chaque cycle.

Nous vous pr�senterons dans ce document la description de la grille horaire actuelle, ses limites, l'architecture de la nouvelle grille horaire, ses avantages et ses limites. 