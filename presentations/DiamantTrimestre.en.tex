% $Header: /grp/cvs/cvstictac/DiamantExtremeCopy/presentations/DiamantTrimestre.en.tex,v 1.3 2005-11-14 00:24:20 gonzrubi Exp $

\documentclass{beamer}

\usepackage{dfd,hhline}

% This file is a solution template for:

% - Giving a talk on some subject.
% - The talk is between 15min and 45min long.
% - Style is ornate.



% Copyright 2004 by Till Tantau <tantau@users.sourceforge.net>.
%
% In principle, this file can be redistributed and/or modified under
% the terms of the GNU Public License, version 2.
%
% However, this file is supposed to be a template to be modified
% for your own needs. For this reason, if you use this file as a
% template and not specifically distribute it as part of a another
% package/program, I grant the extra permission to freely copy and
% modify this file as you see fit and even to delete this copyright
% notice.

\newcommand{\brun}{\noindent $\triangleright$}
\newcommand{\erun}{$\triangleleft$}
\newcommand{\key}{\textsf}
\newcommand{\ita}{\textit}
\newcommand{\dos}{\textsc}
\newcommand{\bld}{\textbf}
\newcommand{\ang}{\textsf}
\newcommand{\pro}{\texttt}
\newcommand{\uds}{Universit\'{e} de Sherbrooke}
\newcommand{\bu}{Bishop's University}
\def\eXit{$\epsilon$\kern-.100em \lower.5ex\hbox{X}\kern-.125emit}


\mode<presentation>
{
  \usetheme{Warsaw}
  % or ...
  \usecolortheme{green}
  \setbeamercovered{transparent}
  % or whatever (possibly just delete it)
}


\usepackage[english]{babel}
%\usepackage[french]{babel}
% or whatever

\usepackage[latin1]{inputenc}
% or whatever

\usepackage{times}
\usepackage[T1]{fontenc}
% Or whatever. Note that the encoding and the font should match. If T1
% does not look nice, try deleting the line with the fontenc.


\title[Diamant T.] % (optional, use only with long paper titles)
{Les horaires trimestriels et Diamant}

\subtitle
{Quelques id�es} % (optional)

\author[R.G.R.] % (optional, use only with lots of authors)
{Ruben Gonzalez-Rubio\inst{1}}
% - Use the \inst{?} command only if the authors have different
%   affiliation.


\institute[UdeS] % (optional, but mostly needed)
{
  \inst{1}%
   D\'epartement de g\'enie \'electrique et de g\'enie informatique \\
\uds{}, Sherbrooke, QC, J1K 2R1, Canada }
% - Use the \inst command only if there are several affiliations.
% - Keep it simple, no one is interested in your street address.

\date[] % (optional)
{Juin 05 / r�union}

\subject{Talks}
% This is only inserted into the PDF information catalog. Can be left
% out.



% If you have a file called "university-logo-filename.xxx", where xxx
% is a graphic format that can be processed by latex or pdflatex,
% resp., then you can add a logo as follows:

\pgfdeclareimage[height=0.25cm]{uLogo}{uLogo-udslogo}
\logo{\pgfuseimage{uLogo}}




% Delete this, if you do not want the table of contents to pop up at
% the beginning of each subsection:
\AtBeginSubsection[]
{
  \begin{frame}<beamer>
    \frametitle{Outline}
    \tableofcontents[currentsection,currentsubsection]
  \end{frame}
}


% If you wish to uncover everything in a step-wise fashion, uncomment
% the following command:

%\beamerdefaultoverlayspecification{<+->}


\begin{document}

\begin{frame}
  \titlepage
\end{frame}

\begin{frame}
  \frametitle{Outline}
  \tableofcontents
  % You might wish to add the option [pausesections]
\end{frame}


% Since this a solution template for a generic talk, very little can
% be said about how it should be structured. However, the talk length
% of between 15min and 45min and the theme suggest that you stick to
% the following rules:

% - Exactly two or three sections (other than the summary).
% - At *most* three subsections per section.
% - Talk about 30s to 2min per frame. So there should be between about
%   15 and 30 frames, all told.

\section{Introduction}

\subsection[]{Information}

\begin{frame}
  \frametitle{Entr�e d'information}
  \framesubtitle{Sources}
  % - A title should summarize the slide in an understandable fashion
  %   for anyone how does not follow everything on the slide itself.

  \begin{itemize}
  \item Registraire.
    \pause
    \begin{itemize}
        \item �tudiants.
        \pause
        \end{itemize}
    \item Facult�s.
       \pause
        \begin{itemize}
            \item Offre d'AP, donn�es de l'horaire.
        \pause
        \end{itemize}
    \item Diamant.
        \pause
        \begin{itemize}
            \item Aide � construire l'horaire.
        \pause
        \end{itemize}
   \end{itemize}
\end{frame}

\begin{frame}
  \frametitle{La production d'horaires}

\begin{figure}

\setlength{\unitlength}{0.1cm}
\begin{picture}(90,35)
\deffilebox{20}{10}{0.2}
\defprocessbox{20}{16}
\put(0, 15){\fichier[\pro{Donn�es}]}
 \put( 20,20){\vector(1,0){15}}
\put(35, 12){\process[Syst�me de\\production \\ d'horaires]}
\put(55, 20){\vector(1,0){7}} \put(62, 20){\corner[y]{-10}{8}}
\put(62, 20){\corner[y]{10}{8}} \put(70, 25){\fichier[\pro{Papier}]}
\put(70, 5){\fichier[\pro{Page Web}]}
\end{picture}
\caption{La production d'horaires}\label{fig:fOne}
\end{figure}



\end{frame}


\begin{frame}
  \frametitle{L'�volution des donn�es pendant la production d'un
horaire}

\begin{figure}[htb] \begin{center}
\setlength{\unitlength}{0.1cm}
\begin{picture}(90,40)
\deffilebox{20}{10}{0.2}
\defprocessbox{20}{16}
\put(0, 15){\fichier[$bd_{1}$]}

\put(0, 5){\makebox(20,10){D. de base}}

\put(20, 20){\vector(1,0){10}} \put(30, 15){\fichier[$bd_{2}$]}
\put(30, 5){\makebox(20,10){D. p�riode $P$}}
\put(50,20){\vector(1,0){10}}

\put(60, 15){\fichier[$bd_{3}$]}

\put(60, 5){\makebox(20,10){D. horaire}}
\end{picture}
\end{center}

\caption{L'�volution des donn�es pendant la production d'un
horaire}\label{fig:fTwo}
\end{figure}



\end{frame}


\section{Le trimestre}

\subsection[]{Information nouvelle}

\begin{frame}
  \frametitle{Donn�es � definir}
  \framesubtitle{Sources}
  % - A title should summarize the slide in an understandable fashion
  %   for anyone how does not follow everything on the slide itself.

  \begin{itemize}
  \item Registraire.
    \pause
    \begin{itemize}
        \item �tudiants. Rien � changer. �tudiant inscrit � AP.
        \pause
        \end{itemize}
    \item Facult�s.
       \pause
        \begin{itemize}
            \item Offre d'AP, donn�es de l'horaire. Correspondance
            entre AP et APP. �v�nements dans un APP. �quipes ou-et
            groupes? Grille horaire (d�j� fait). Reservation de
            locaux.
        \pause
        \end{itemize}
    \item Diamant.
        \pause
        \begin{itemize}
            \item Aide � construire l'horaire. M�mes modifications
            que sur les donn�es. Plus unification finale?
        \pause
        \end{itemize}
   \end{itemize}
\end{frame}

\begin{frame}
  \frametitle{R�union Facult� de g�nie}
  \framesubtitle{Sources}
  % - A title should summarize the slide in an understandable fashion
  %   for anyone how does not follow everything on the slide itself.

  \begin{itemize}
  \item G�nie m�canique.
    \pause
    \begin{itemize}
        \item Horaire dynamique (100\% s�r pour deux semaines).
        \pause
      \item Substitution de jours.
        \pause
        \end{itemize}
    \item G�nie �lectrique.
       \pause
        \begin{itemize}
            \item Horaire dynamique (100\% s�r pour deux semaines).
            \pause
             \item Correspondance entre AP et APP.
        \pause
        \item Glossaire des APPs.
        \end{itemize}
   \end{itemize}
\end{frame}
\section*{La suite}

\begin{frame}
  \frametitle<presentation>{La suite}

  % Keep the summary *very short*.
  \begin{itemize}
  \item
    R�sum� de discussions + actions.
  \item
    Validations. Unifier avec Facult� m�decine.
  \item
    Planification du projet. Unifier avec Facult� m�decine.
  \end{itemize}

  % The following outlook is optional.
  \vskip0pt plus.5fill
  \begin{itemize}
  \item
    Autres possibilit�s
    \begin{itemize}
    \item
      Possibilit�
    \item
      Possibilit�.
    \end{itemize}
  \end{itemize}
\end{frame}


\end{document}
