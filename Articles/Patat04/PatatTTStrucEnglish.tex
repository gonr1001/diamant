
\documentclass{llncs}
\bibliographystyle{alpha}
\usepackage[T1]{fontenc}
\usepackage{dfd,hhline}
\usepackage{verbatim}
\setlength{\parskip}{1.5ex plus0.5ex minus0ex}


\newcommand{\ang}{\textsf}
\newcommand{\key}{\textsf}
\newcommand{\ita}{\textit}
\newcommand{\bld}{\textbf}
\newcommand{\dos}{\textsc}
\newcommand{\pro}{\texttt}

\newcommand{\saphir}{SAPHIR}
\newcommand{\diamant}{DIAMANT}
\newcommand{\tictac}{Tic-Tac}
\newcommand{\xp}{eXtreme Programming}
\def\eXit{$\epsilon$\kern-.100em \lower.5ex\hbox{X}\kern-.125emit}


\title{\bf{Comparing  $1$, $2$ and $n$-dimensions  timetable data structures}}%% \\(document pr�paratoire)}}
\author{Bernard Beaulieu \\ Ruben Gonzalez Rubio}
\institute{D�partement de g�nie �lectrique et de g�nie informatique \\
Universit� de Sherbrooke,\\
Sherbrooke, Qu�bec, J1K 2R1 \\
Canada\\
\email{Ruben.Gonzalez-Rubio@USherbrooke.ca} \\
\email{Bernard.Beaulieu@USherbrooke.ca}}
\date{}



%%%%%%%%%%%%%%%%%%
\pagestyle{plain} %% to be commented when sent
\newcommand{\ints}{\renewcommand{\baselinestretch}{1.0}\small\normalsize}
\newcommand{\intm}{\renewcommand{\baselinestretch}{1.5}\small\normalsize}
\newcommand{\intd}{\renewcommand{\baselinestretch}{2.0}\small\normalsize}
%%%%%%%%%%%%%%%%%%

\begin{document}

\maketitle

%\intd

\begin{abstract} \label{intro}
Timetable construction is a problem that appears in several
fields~: academics institutions, health centers, personnel
management and others. It is a complex problem ($NP-$complet).
There are programs which  build timetables in all these fields.
The common point of these programs is to assign in a timetable the
events respecting most of the constraints.

There are several ways of timetable modelling. We propose in this
article, an evaluation of three data structures allowing to model
the same timetable, they are 1, 2 or $n$-dimensions. The
comparison criteria are~: capacity to represent constraints,
effectiveness, facility of adaptation and re-use.

The capacity to represent constraints is higher if the structure
lets to express them simply and directly. The effectiveness is
better if the number of operations to validate a constraint is
small. The adaptation is easier when the structure has a
parametric representation and when it is simple. The re-use is a
quality when the structure can be used in several other
application fields.

The objective of this comparison is to let us make a clear choice
of the timetable structure in the development of timetable
software.

\end{abstract}

{\bf Keywords :} Timetable, Timetable data structures,
Periodicity, Object-oriented programming.



\bibliography{bibDiamant,bibProg}


\end{document}



%%%%%%%%%%%%%%%%%%%%%%%%%%%%%%%%%%%%%%%%%%%%%%%%%%%%%%%%%%%%%%%%%%%%%%%%%%%%%%%
