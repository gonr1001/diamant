La construction d'horaires est un probl�me qui se pr�sente dans
plusieurs domaines, comme dans les institutions d'enseignement,
dans les centres de sant�, dans la gestion de personnel et autres.
Il s'agit d'un probl�me complexe ($NP-$complet). Il existe de
logiciels pour construire des horaires dans tous ces domaines. Le
point commun de ces logiciels, est d'assigner dans une grille
horaire des �v�nements respectant des contraintes.

Il existe plusieurs mani�res de mod�liser une grille horaire. Nous
proposons, dans cet article, une �valuation de trois structures de
donn�es permettant de mod�liser une m�me grille horaire, soit des
structures � 1, 2 et $n$ dimensions. Les crit�res de comparaison
sont~: capacit� de repr�senter des contraintes, efficacit�,
facilit� d'adaptation et r�utilisation. La capacit� de repr�senter
des contraintes est sup�rieure si la structure permet de les
exprimer directement. L'efficacit� est meilleure si le nombre
d'op�rations pour valider une contrainte est petit. La facilit�
d'adaptation est am�liore lorsque  la structure poss�de un
repr�sentation param�trable et lorsqu'elle est simple. La
r�utilisation est une qualit� quand la structure peut �tre utilis�
dans plusieurs autres domaines d'application.

L'objectif de cette comparaison est permettre de faire un choix
�clair� de la structure de la grille horaire dans  le
d�veloppement de logiciel d'horaires.
