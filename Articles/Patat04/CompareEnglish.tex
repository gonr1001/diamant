The timetable construction is a problem that appears in several
fields~: the academics institutions, the health centers, the
personnel management and others. It is a complex problem
$NP-$complet). There are software to build timetables in all these
fields. The common point of this software, is to assign in a
timetable the events respecting the constraints.

There are several ways of timetable modelling. We propose in this
article, an evaluation of three data structures allowing to model
the same timetable, it could be structures with 1, 2 or $n$
dimensions. The comparison criteria are~: capacity to represent
constraints, effectiveness, facility of adaptation and re-use.

The capacity to represent constraints is higher if the structure
lets to express them directly. The effectiveness is better if the
number of operations to validate a constraint is small. The
adaptation is easier when the structure has a parametric
representation and when it is simple. The re-use is a quality when
the structure can be used in several other application fields.

The objective of this comparison is to let us make a clear choice
of the timetable structure in the development of timetable
software.
