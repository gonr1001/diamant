\documentclass{llncs}
\bibliographystyle{alpha}
\usepackage[latin1]{inputenc}
\usepackage[T1]{fontenc}
\usepackage[english]{babel}
\usepackage{dfd,hhline}
\usepackage{verbatim}
\setlength{\parskip}{1.5ex plus0.5ex minus0ex}


\newcommand{\ang}{\textsf}
\newcommand{\key}{\textsf}
\newcommand{\ita}{\textit}
\newcommand{\bld}{\textbf}
\newcommand{\dos}{\textsc}
\newcommand{\pro}{\texttt}

\newcommand{\saphir}{SAPHIR}
\newcommand{\diamant}{DIAMANT}
\newcommand{\tictac}{Tic-Tac}
\newcommand{\xp}{eXtreme Programming}
\def\eXit{$\epsilon$\kern-.100em \lower.5ex\hbox{X}\kern-.125emit}


\title{\bf{A timetable production system architecture for courses and exams}}

\author{Ruben Gonzalez Rubio \\Domingo Palao Mu�oz}
\institute{D�partement de g�nie �lectrique et de g�nie informatique \\
Universit� de Sherbrooke,\\
Sherbrooke, Qu�bec, J1K 2R1 \\
Canada \\
\email{Ruben.Gonzalez-Rubio@USherbrooke.ca} \\
\email{Domingo.Palao@USherbrooke.ca}}
\date{}



%%%%%%%%%%%%%%%%%%
\pagestyle{plain} %% to be commented when sent
\newcommand{\ints}{\renewcommand{\baselinestretch}{1.0}\small\normalsize}
\newcommand{\intm}{\renewcommand{\baselinestretch}{1.5}\small\normalsize}
\newcommand{\intd}{\renewcommand{\baselinestretch}{2.0}\small\normalsize}
%%%%%%%%%%%%%%%%%%

\begin{document}

\maketitle

%\intd

\begin{abstract} \label{intro}
The timetable production in an academic institution is a periodic
task, where the frequency depends on the institution~: each year,
each six-months or each quarter. A part of this production is the
timetable construction for a set of courses. A timetable must
satisfy the hard constraints and try to respect a maximum of soft
constraints. It is recognized that the timetable construction is a
very complex problem, often software helps to carry out this
construction.

In this article, we propose to analyse the timetable construction
problem in a larger view, i.e., to consider the timetable
production as a process which goes from the gathering and data
input until the timetable delivery, including the construction.
Thus, the timetable production can be automated by using a
database, a Web site and a software for timetable construction. We
analyzed the flow of information necessary to the timetable
production, which enabled us to better understand the
characteristics of these systems. We propose a generic
architecture of a system of timetable production for courses and
exams. This architecture was built around two fundamental
qualities of software development~: the opening and extensibility.

\end{abstract}

{\bf Keywords :} Timetable production, Timetable construction,
Open architecture, Extensible Architecture, Databases, Web.

\section{Introduction}

Timetable production is a periodic activity carried out in many
academic institutions. These institutions produce one or more
timetable for their courses and exams each year. Computers can
help this production to reduce its cost in an important way. In
certain cases, automation starts with the use of timetable
construction software. In others, it starts with the automation of
the treatments associated with the work of the institution, for
example the follow of students. There are software to carry out
the timetable construction, as example DIAMANT
\cite{gonzalezrubio00} and SAPHIR \cite{ferland94}, but, there
exists many others.

In general, timetable creation software uses one or more files as
input, and one or more files as output. Working with files is a
problems source. It is necessary to build the input files each
time that a timetable is produced. There exists, two cases to
produce those files~: manual production and automated production.
If the files are manually produced, there are the typing errors.
These kind of errors can produce unexpected effects in the
software. These errors are very hard to detect, because all the
data should be manually checked. In the other case, the files are
produced in an automatic way. It is necessary that the software be
ensured to check the validity and the coherence of all the data.

Another source of problems even more expensive than the preceding
is that when a change is introduced into the system. It could be
in the construction software, or in the production software or in
both. The changes come on behalf of persons in charge for the
institution, for example a new nomenclature for the buildings; a
new teaching approach, which change the way to create the groups,
etc. The problem of modifying software is not only applied to the
timetable production software, it is a general problem in software
development.

The solution to the first problem passes by the automation of the
data checking and data validation at the time of data acquisition.
In order to ensure that the construction software works with the
right data.

The second problem is more delicate to treat. If the production
software was conceived to be modified (extensible), the cost to
modify is weak, but if the system is monolithic and closed, a
modification will be very expensive. A way of decreasing the
effects of this problem is the development of software having the
extensibility property.

In this article, we propose an timetable production system
architecture for courses and exams, having the following
qualities: openness and extensible. This architecture must
contribute to work in an easy way with the two problems mentioned,
by reducing the cost of the modifications and by carrying out the
checking and the validation of the data in precise stages of the
process and by avoiding the redundancy of the code and the data.
This architecture implements a database, a Web site and a software
for timetable construction. The checking and the validation of
data will be made with the input and with integrity constraints in
the database. The Web site is used mainly for the data input, but
at the same time, it can check the data validity, using the
business rules before the data will be recorded at the database.
Another function of the Web site is to offer the timetable in a
personalized way, once it is built. The database allows the data
retrieval, according to various formats. Thus, the input of the
software of construction is simplified. Of course, with the
assistance of the timetable construction software, it is possible
to create a timetable with a minimum of conflicts.


A lot of research was made in the timetable construction~:
algorithms, mathematical formulations and several software were
proposed in the PATAT conferences \cite{patat97,patat00,patat02}.
On the other hand, few efforts were devoted, until now, to the
general development of the timetable production. We think that the
fact of having a total view on the process can help the research
in timetable construction. White \cite{white00} proposed that the
Web should be used to the diffusion of timetables. We are trying
to go far by proposing also to use the Web like an interface of
data input; moreover, we propose the architecture necessary to
support our ideas. De Causmaecker et al. \cite{decausmaecker02}
introduced the idea that the semantics Web can be used by the
researchers in the timetable construction. They also introduced
the idea to use XML as a communication language  in the timetable
construction by agents. We propose also the use of XML, in order
to facilitate the exchanges between certain modules of our
architecture. In Burke and al \cite{burke97} it is proposed to
have a standard for the data format, it will be used by the
timetable software; their goal was to compare algorithms
(programs) in benchmarks. In a more recent work, Kingston
\cite{kingston02} proposes the STTL language and an interpreter,
with the same objective. �zcan \cite{ozcan03} suggests also
another standard to define instances of timetable construction
problems, having XML as a base language. It seems that it would be
desirable to have a standard, but it is a goal very hard to reach.
Instead, our approach is the use of a database like a tool for
extraction and data formatting, this way the software of timetable
construction can receive the necessary data. Thus, it would be
possible to generate other instances of problems for other
software of timetable construction.

The development of our architecture, was divided into two parts.
We analyzed the data flow in a generic timetable production
system. Then we define the architecture, the basic blocks and the
responsibility of each block participating in the architecture.
The objectives to define an open and extensible architecture were
respect during our work. This is presented in the sections
\ref{sec:data} and \ref{sec:archi}.

Once the architecture is defined, we created an instance. In order
to establish this architecture in a timetable production system at
the University of Sherbrooke. We made an overview of existing
technologies in order to make the good choices still respecting
the objectives of openness and extensibility. This is presented in
the sections \ref{sec:tech} and \ref{sec:tictac}.


We finish by an assessment and the conclusions in the section
\ref{sec:conc}.



\section{The information flow in the timetable production}
\label{sec:data}

First, we define the process of timetable production as a process
which starts at the time to enter the data for the period of
validity of the timetable $P$. This period $P$ is one six-month
period, one quarter, or a year period\footnote{At the University
of Sherbrooke we talk about quarter, but it really comprises
sixteen weeks (almost four months), therefore the system must be
adapted to any period $P$.}. The process ends when all the
concerned people receive the schedule on a paper or via the Web
(see figure \ref{fig:fOne}) and the period of validity of the
schedule touches at its end. Multiple data are entered to the
system, for example the courses which will take place during the
period $P$; the format of the courses (3h, or 2h+1h); the possible
or assigned teachers for certain courses; the possible or assigned
classrooms; the availability of the teachers and the classrooms,
etc. We know that there are institutions where the students make a
pre-selection and others where the timetable is made without
pre-selection. The pre-choice of the students is part of the data
to be supplied, if there is no pre-choice, it is necessary to
indicate the number of students to be admitted for each course.

\begin{figure}[htb] \begin{center}
\setlength{\unitlength}{0.1cm}
\begin{picture}(90,35)
\deffilebox{20}{10}{0.2}
\defprocessbox{20}{16}
\put(0, 15){\fichier[\pro{Data}]}

\put( 20,20){\vector(1,0){15}}

\put(35, 12){\process[Timetable\\production\\software]}

\put(55, 20){\vector(1,0){7}} \put(62, 20){\corner[y]{-10}{8}}
\put(62, 20){\corner[y]{10}{8}}

\put(70, 25){\fichier[\pro{Paper}]}

\put(70, 5){\fichier[\pro{Web page}]}
\end{picture}
\end{center}
\caption{Timetable production}\label{fig:fOne}
\end{figure}

The figure \ref{fig:fOne} is an abstract view of the timetable
production, the system of timetable production includes software
of timetable construction.

\begin{description}
\item[Timetable production system.] It is composed of several
software, or mo\-dules, which works in an autonomous, but
coordinated way. It is clear that the system must have two
principal components~: the system to save data and the timetable
construction software.\\

\item[Data.] Represent the necessary data to the timetable
production. We will save all the necessary data in a system of
data back up. In the figure, it is not indicated that these data
can be handled by modules of the system of timetable production.
For example, the module of acquisition of pre-choice will save the
courses followed by a student. The module of data construction
will take its inputs from the database, and when the construction
is made the new data obtained will be saved in a data warehouse.\\


\item[Paper or Web page.] It is the output of the system once that
the timetable is built. This output can be saved in files, which
will be printed on paper or data that will be presented on a Web
page. In both cases, the data source is the database.

\end{description}


\subsection{Data preparation}
The figure \ref{fig:fTwo} shows the data evolution during the
timetable production. The evolution presents the instance of the
data at different moments of the process.

The data warehouse can represent one or several databases, we can
think to a distributed database. The choice of the database is a
implementation decision. In a conceptual way it is only one data
warehouse.


\begin{figure}[htb] \begin{center}
\setlength{\unitlength}{0.1cm}
\begin{picture}(90,40)
\deffilebox{20}{10}{0.2}
\defprocessbox{20}{16}
\put(0, 15){\fichier[$bd_{1}$]}

\put(0, 5){\makebox(20,10){Basic data}}

\put(20, 20){\vector(1,0){10}}

\put(30, 15){\fichier[$bd_{2}$]}

\put(30, 5){\makebox(20,10){Period $P$ data}}

\put(50, 20){\vector(1,0){10}}

\put(60, 15){\fichier[$bd_{3}$]}

\put(60, 5){\makebox(20,10){Timetable data}}
\end{picture}%}
\end{center}

\caption{Data evolution during the timetable
production}\label{fig:fTwo}
\end{figure}


\begin{description}

\item[Basic data.] Theses data are defined as the long-term data
in the data warehouse. The information for the academic activities
of a program, the programs offered, the teachers, the
characteristics of the classrooms, the students, all of them are
examples of this type of information. In general, this information
is shared by other applications, for example, a student is
described by his first name, last name, identification number,
program, etc. An application that uses the data of the students is
for example that which print the notes of a student on his
scorecard. Here, we are interested in the data associated with the
timetable production, which can be a view (a part) of the complete
institutional database.\\

\item[Period $P$ data.] These data are necessary to the timetable
construction for the concerned period; they contain the courses
that will be offered in this period, the availability of teachers
for this period, etc. These data are a precondition to the
timetable construction and will be entered before each
construction. However, these data will change (update) during the
timetable construction. For example when a external teacher is
assigned for a given course, we will enter his name and
availability in the database.\\

\item[Timetable data.] These data are the result of the timetable
construction. They will contain all the necessary information to
print or to show a general and personalized timetable.

\end{description}

Aside from the timetable construction for courses, the system must
be used to build the timetable for exams. The evolution of data,
in the case of exams, follows the same path as courses. With the
exception that the data extracted from the database can be
different, for example, for an exam timetable the availability of
the teachers can not be necessary.

There are two major activities in timetable production: enter data
(including the update) and timetable construction.

\begin{figure}[htb] \begin{center}
\setlength{\unitlength}{0.1cm}

\begin{picture}(90,40)
 \put(-6,7){ \shortstack{b)}}
 \put(0, 10){\vector(1,0){90}}
 \put(5, 9){\line(0,1){2}}
 \put(4,7){\shortstack{$t_0$}}
 \put(85, 9){\line(0,1){2}}
 \put(84,7){\shortstack{$t_f$}}
 \put(5, 12){\vector(1,0){75}}
 \put(45, 14){\vector(1,0){40}}
 \put(9,13){ \shortstack{Data input}}
 \put(54,16){ \shortstack{Construction}}
 \put(-6,22){ \shortstack{a)}}
 \put(0, 25){\vector(1,0){90}}
 \put(5, 24){\line(0,1){2}}
 \put(4,22){ \shortstack{$t_0$}}
 \put(5, 27){\vector(1,0){40}}
 \put(45, 27){\vector(1,0){40}}
 \put(85, 24){\line(0,1){2}}
 \put(84,22){\shortstack{$t_f$}}
 \put(9,28){ \shortstack{Data input}}
 \put(54,28){ \shortstack{Construction}}

\end{picture}%}
\end{center}

\caption{Activities of data input and timetable construction for
one period $P$}\label{fig:fTime}
\end{figure}

The figure \ref{fig:fTime} shows us the two cases of these
activities evolution. In case $a)$ the data input and the
construction are two activities carried out in sequence. This
represents the ideal case. In case $b)$ which shows the real
world, the data input (especially the update) is done at the same
time as the construction. At the beginning of a construction, some
data are fixed, but if the software or the person in charge does
not find a satisfactory solution it is necessary to make changes
even in the fixed data.


The activities of data input and construction can be carried out
by many users. Each one will have an associated role and
privileges to be able to change the data permitted by its role.


In order to simplify, we consider two types of users~: the clerk
and the person in charge for the timetable construction. The clerk
is that which will enter the information in the database. The
person in charge for the timetable construction interacts with the
timetable construction program.

The clerk will enter or update the data in the database, if the
data is valid, it will be accepted and added to the database. For
example if the data waited is an integer number in a specific
range, and the input value corresponds with the attended value, it
will be entered in the database. If not, the system must force the
clerk to return a new value.


The timetable construction is an activity of iterative nature. In
general, the algorithms of timetable construction propose to
respect the hard constraints and to try to respect soft
constraints as much as possible; therefore, the person in charge
will try to find several solutions to choose one of them. In
certain cases, it is possible to leave the program with the same
data to find a new solution, but in other cases, it is necessary
to change the starting data to obtain a new solution. If it is
necessary to make $n$ tests the chosen solution by the person in
charge will be the solution $s$ where $1 \leq s \leq n$. It is
necessary to look for a way of working with the database in order
to save the changes carried out for each iteration and to make a
''commit'' when the solution is chosen and timetable becomes final
for the period.


\section{The architecture} \label{sec:archi}


Follows to the study of the data flow in the timetable production,
we are able to propose the architecture to implement the system.

\subsection{An architecture for the timetable production}

In an abstracted way, software architecture describes the elements
of a system; it shows also the interactions between these
elements, the models that guides its composition and the
constraints of these models. \cite{shaw96}

In a general way, when we have a complex problem, the best
approach is to cut out in fragments that are easier to solve with
a simple solution. Then, if we put all these small solutions
together, we will find the solution of our complex problem
\cite{buschmann96}.

Now there are several architectures available to develop software
applications, the most adapted for the timetable production system
is that in layers.

The architecture shown in the figure \ref{fig:fThree} proposes the
division of the system in three layers, each one with a
well-defined function~:

\begin{description}
\item [The interface.] Presents the data to the user, to allow the
data input, and ensure the exchanges with other layers.\\

\item [Business logic layer.] Ensures the data exchanges with the
interface layer. Validates and checks the data input and sending
the data to be presented in an adequate format. It also ensures
the data exchanges with the data persistence layer. The business
rules are used to ensure the coherence of the system.\\

\item [The data persistence layer.] It manages the physical
storage of the data that is in files with a certain format, or in
a system of traditional database or in other models of persistence
that are able to manage complex databases.

\end {description}

\begin{figure}
\setlength{\unitlength}{1cm}
\begin{picture}(3,4.5)
  \put(3,4){\framebox(4,1){Interface}}
  \put(4.5,3.5){\vector(0,1){0.5}}
  \put(5.5,4){\vector(0,-1){0.5}}
  \put(3,2.5){\framebox(4,1){Business logic}}
  \put(4.5,2){\vector(0,1){0.5}}
  \put(5.5,2.5){\vector(0,-1){0.5}}
  \put(3,1){\framebox(4,1){Data persistance}}
\end{picture}
\caption{The three layers architecture}\label{fig:fThree}
\end{figure}



\subsection {The three layers architecture and the system of timetable production.}

For each element of the architecture that we presented in the
earlier subsection, we propose a component that is part of the
application of timetable production. In this subsection, we will
show the integration between the data flow for a system of
timetable production and the suggested architecture.


The high-level design of our system of timetable production is
simple (see figure \ref{fig:fOne})~:


\begin{enumerate}
\item We defined three categories of data in a timetable
production system. The figure \ref{fig:fTwo} shows these
categories. The interface layer treats this data as follows~:\\


\begin{enumerate}

\item {\bf Basic data.} This data is in a database belonging to
the institution. This database can have its own interfaces to get
and show the data. The system of production needs a part of these
data. It is necessary to have an interface with other systems; its
responsibility is to seek the necessary data and to place it in
the production system. An important remark is that the users of
the system should not have the right to modify the data of the
institution for security reasons.\\


\item {\bf Direct input data for the period $P$.} This activity is
made by a clerk. Normally, these data change with each creation of
a timetable. To enter the data, the clerk has a Web page that is
displayed by a navigator, and it will have fields to fill. At this
moment, the interface layer can make some minimal validations of
the data coherence, for example, check that the data type is
correct, to check that the obligatory fields are filled, etc.
Depending on the size of the institution, it can have several
clerks who work at the same time in the application, consequently
it is necessary, to take the necessary measures to allow the
concurrent work in the data persistence layer.\\

\item {\bf The timetable data.} Created by the timetable creation
software. This program is managed by the user responsible for the
timetable creation. Since this activity is of iterative nature,
there can be a certain number of versions of the work that the
user responsible for the timetable creation must analyze to find
that which is appropriate to him. Once that he found this version,
he must apply the operation ``commit'' to make the information
available to the rest of the users. Those are the output data from
our system. They are sent to other systems or other users, for
example, the student will receive a paper with their personalized
timetable or he can consult this information by using a Web site
created for this ends.\\
\end{enumerate}

It is necessary to keep in consideration that the interface layer
communicates only with the business logic layer. An application is
not able to add, modify or to erase data directly in the database,
it must pass irremediably by the business logic layer. This
characteristic enables us to ensure the coherence of the data.

\item The data is treated according to certain rules and
constraints, these rules are defined by each institution and they
are coded in the business logic layer. It is here that we will be
able to express constrains, for example, the maximum of students
in a group, the maximum number of courses that a professor can
teach, etc.

\item The data results of these treatments must be store in a
place to be able to use them after its generation. The data
persistence layer is responsible for this activity. In a system of
timetable production, the data can be stored in a database
(relational, object or another technology), in a XML file, in a
text file or in another storage media.
\end{enumerate}

For elements of composition and its relations see figure
\ref{fig:fFour}.

\begin{figure}
\setlength{\unitlength}{0.5cm}
\begin{picture}(19,7)

  \put (0,6){\line(1,0){22}}
  \put (19,6.5){\makebox(0,0){\scriptsize Interface layer}}
  \put(0,6.5){\framebox(4,1){\tiny Data input}}
  \put(2,6.65){\makebox(0,0){\tiny (Web client)}}
  \put(5,6.5){\framebox(4,1){\tiny Timetable cr�ation}}
  \put(7,6.65){\makebox(0,0){\tiny (Interface)}}
  \put(10,6.5){\framebox(4,1){\tiny Other syst�mes}}
  \put(12,6.65){\makebox(0,0){\tiny (Interface)}}
  \put (2,6.5){\vector(1,-1){2}} %SD -> SW
  \put (4,4.5){\vector(-1,1){2}} %SW -> SD
  \put (7,6.5){\vector(-1,-1){2}} %CH -> SW
  \put (5,4.5){\vector(1,1){2}} %SW -> CH
  \put (7,6.5){\vector(1,-1){2}} %CH -> RA
  \put (9,4.5){\vector(-1,1){2}} %RA -> CH
  \put (12,6.5){\vector(-1,-1){2}} %AS -> RA
  \put (10,4.5){\vector(1,1){2}} %RA -> AS


  \put (0,3){\line(1,0){22}}
  \put (19,3.5){\makebox(0,0){\scriptsize Business logic layer}}
  \put (2,3.5){\framebox(4,1){\tiny Web server}}
  \put (8,3.5){\framebox(4,1){\tiny Business rules}}
  \put (6,4){\vector(1,0){2}} %RA -> SW
  \put (8,4){\vector(-1,0){2}} %SW -> RA
  \put (10,3.5){\vector(-1,-1){2}} %RA -> BD
  \put (8,1.5){\vector(1,1){2}} %BD -> RA


  \put (0,0){\line(1,0){22}}
  \put (19,0.5){\makebox(0,0){\scriptsize Data persistance layer}}
  \put (5,0.5){\framebox(6,1){\tiny Database server}}




\end{picture}
\caption{The implementation of the three layers architecture for
the system of timetable production}\label{fig:fFour}
\end{figure}


\section{Possible technologies}\label{sec:tech}


For the implementation of architecture, first it is necessary, to
find what technological elements we need. The needs are:

\begin{enumerate}
\item A language to facilitate the display of the data to the user
and the creation of forms to enter data.

\item A set of rules which define the way of establishing the
exchange of data with the other systems.

\item A language to code the business logic.

\item A tool to manage the data persistence.
\end{enumerate}

After our analysis, we found that there is not only one option;
rather, there are several possibilities to implement this
architecture.

In the Table \ref{tb:tbOne}, we show a list of possible instances,
it is not exhaustive.


\begin{table}
\begin{tabular}{|c||c|c|c|}\hline
Technology & Instance 1 & Instance 2 & Instance 3\\ \hline\hline
Language for display & HTML et JSP& Applet Java & Visual Basic\\
\hline Format to data exchange & XML & Text & Text\\ \hline

Language to code the business logic & Java & Java & Visual Basic\\
\hline

The data persistence & Database & Database & Database \\
\hline
\end{tabular}
~ \\
 \caption{ Some possible instances of the architecture}\label{tb:tbOne}
\end{table}

For the construction of each instance we put together the elements
which have common characteristics or which belongs to the same
software family and which is open.

For the database section, we did not mention a specific name of a
database management system; the choice remains open to the
implementation.

We have an open architecture, since it accepts several instances.
The architecture is also extensible, because we could change the
modules in an easy way, for example, we can start with an element
of data persistence like a file in text format, and then, we could
carry out some changes in the element of interface to be able to
interact with files in XML format. Later, if we want to make
evolve the data persistence element to a database system, it would
be necessary to make changes in the interface layer and to program
the data persistence element. We can make all these changes
without touching the business logic layer. In the Table
\ref{tb:tbTwo}, we show the possible evolution of an instance of
our architecture.

\begin{table}
\begin{tabular}{|c||c|c|c|}\hline
Technologie & Version 1 & Version 2 & Version 3\\ \hline\hline
Language for display & HTML et JSP & HTML et PHP & HTML et JSP\\
\hline

Format to data exchange & Text & XML & Database SQL\\
\hline

Language to code the business logic & Java & Java & Java\\
\hline

The data persistence & Text & XML & SQL, Database SQL \\
\hline

\end{tabular}
~ \\

\caption{ The evolution of an instance of the
architecture}\label{tb:tbTwo}
\end{table}

We can also imagine a system that starts with an interface based
in textual mode to show the information, which will evolve with a
display based in HTML and JSP by using a JSP server.


\section{An instance of the architecture} \label{sec:tictac}

Once we established the architecture and that we know that it is
open and extensible, it is necessary that the instance that we
will implement will be open and extensible too. We showed three
options of instance of the architecture (see Table
\ref{tb:tbOne}), for our implementation, we chose option 1.

First, it should be known that an element is open if it is built
with approved standards; or, if it is built with a private
specification but made public by the
developers\footnote{http:\//\//www.webopedia.com\//TERM\//O\//open\_architecture.html}.

It is important to remember that an element is extensible if it is
easy to adapt this product to the changes in the
specification\cite{meyer97}.


With these considerations in hand, we show here the analysis that
brought us to our choice:


\begin{description}

\item [Language for display.] For the language for display and to
enter data we chose HTML and JSP, because those languages are one
of the most used to create Web pages, and we want to benefit from
all the characteristics that the site Web offer for the
transactional applications.\\

\item [Format to data exchange.] To exchange the data we chose
XML. The language XML (eXtensible Markup Language) lets us to
store, exchange and show data or information in a structured way.
The language also makes it possible to handle, to transform and to
visualize the data with many tools of formatting, for example, the
information stored in a XML document can be displayed in a Web
navigator. Moreover, when XML documents are stored in a database,
they can be the object of requests and treated like any other
data.\\

\item [Language to code the business logic.] We chose Java. Java
is an interpreted language that uses the Java Virtual Machine
(JVM). There are several versions of this virtual machine for
several different platforms, which means that the programmers can
write for this language by using a platform, and then execute
their programs in others.\\


\item [The data persistence.] For the implementation of the data
persistence layer, we chose to use a database. A database is an
entity in which it is possible to store data in a structured way
with the minimum of redundancy. This data must be used by programs
written in different languages by using standard technology ODBC
(Open DataBase Connectivity). If we use Java as language to
establish the connection to the database, we can use JDBC (Java
Database Connectivity). The standard JDBC is an application
program (API) to establish the connection between the Java
language and a large numbers databases by using "Write Once, Run
Anywhere" philosophy.
\end{description}

Even if it is not the only option for the implementation of our
system of timetable production, we are convinced that our choice
will enable us to make the system evolve by versions, adding or
modifying the new characteristics required by the institutions.


The system implementation seems complex. But, we can reuse several
existing modules. This is one of the advantages of the opening and
and the extensibility of the architecture and the instance. We can
analyze the components by layers to know which ones can be re-used
and which ones will be implemented:


\begin{description}

\item [Interface layer.] For this layer, there is a technological
element which facilitates the work enormously, the Web browser.
This tool is available in most of the computers with Internet
access. It is necessary to centralize the efforts to develop Web
pages with the quality of compatibility between two Web browsers:
Internet Explorer\footnote{http:\//\//www.microsoft.com},
Netscape\footnote{http:\//\//www.netscape.com}, etc.

The Internet site http://validator.w3.org/, offers a tool that
test the compatibility of a site with the standard HTML, this tool
is provided by the "World
Wide Web Consortium"\footnote{http:\//\//www.w3.org/}.\\

\item [Businesses logic layer.] There are two fundamental elements
in this layer: a Web server and a server deploy the business
logic. Since we chose Java to code the business logic, we can use
Apache HTTP\footnote{http:\//\//www.apache.org} as Web server and
Jakarta-Tomcat\footnote{http:\//\//jakarta.apache.org} as a server
to provide the business logic.


The project ''Apache HTTP'' is an effort to develop and maintain
an HTTP server ''Open source'' for the operating systems more used
like UNIX and Windows. The goal of this project is to provide a
secure, effective and extensible server able to offer HTTP
services according to current HTTP standards.

We will program the business rules in some small Java programs
called servlets. A servlet is a Java program used to extend the
functionalities of a Web server. A servlet is a program used to
generate dynamic contents of the applications. It is an
application carried out in the server which is downloaded
dynamically when it is required. Jakarta-Tomcat is a servlet
container.

We can say that the couple Apache and Jakarta-Tomcat enable us to
make the deployment of a complete Web application, in more these
tools are free.\\


\item [Data Persistence layer.] For the data persistence layer we
said that we will benefit from the use of JDBC (Java Database
Connectivity). The JDBC is based on the use of a pilot to give the
access to several databases, the advantage is that we can change
the pilot according the database supplier selected. For the choice
of the database we said that it was opened to the implementation,
but the standards always should be considered. The more recently
standard published by the ANSI (American National Standards
Institute) for the language which processes the data of a database
is SQL-99\footnote{http:\//\//web.ansi.org}. This standard is used
by the most important databases developers like
Oracle\footnote{http:\//\//www.oracle.com},
Informix\footnote{http:\//\//www-306.ibm.com/software/data/informix/},
IBM-DB2\footnote{http:\//\//www-306.ibm.com/software/data/db2/udb/},
PostgreSQL\footnote{http:\//\//www.postgresql.org/}, and others.
For PostgreSQL, using the license is free of charges.\\
\end{description}


As we looked in this section, we benefit from several open
technological components which enable us to speed the development
keeping the low cost.



\section{Conclusions}\label{sec:conc}

We presented the architecture of a timetable production system in
an academic institution. An objective guided our approach: to have
an open and extensible architecture so that it would ensure a
long-life system.

We started by showing the needs for information exchange in the
timetable systems in order to present how the architecture is able
to satisfy these needs. Moreover, in the presentation of the
architecture we showed that it has two qualities. Our architecture
is open: it is possible to build several instances. The
architecture is also extensible because we could change the
modules in an easy way. The responsibilities for each module are
defined since it is possible to identify the part of the system
that has been modified. It is necessary to continue developing
having as objective the openness (compatibility) and the
extensibility in all the software.

It is now possible to give an answer to the question: Can the
timetable construction take advantage of developments associated
with the timetable production? Our experience in software
development for the timetable construction software showed us that
most of the problems came from the fact that the input data has
errors. The construction software must dedicate a large part of
its code to detect those errors. Moreover, users take time to
check the coherence of the data before beginning the building
work. Then, the answer is yes, there is a real benefit. By
developing a system of production, it is easier to ensure that the
software of construction has the responsibility for timetable
construction and not the responsibility of checking the data. And
it's easier to adapt the software to the changes.

Other benefits can also be considered such as consulting the
timetable on the Web from the user's home. Specially in countries
where sometimes it is $-30^\circ C$, that is well appreciated.

\bibliography{bibDiamant,bibProg}


\end{document}



%%%%%%%%%%%%%%%%%%%%%%%%%%%%%%%%%%%%%%%%%%%%%%%%%%%%%%%%%%%%%%%%%%%%%%%%%%%%%%%


\section{Glossaire}


\begin{description}

\item [Architecture logiciel:] "Abstractly, software architecture
involves the description of elements from which systems are built,
interactions among those elements, patterns that guide their
composition, and constraints on there patterns." \cite{shaw96}

\item [Architecture Ouverte:] "An architecture whose
specifications are public. This includes officially approved
standards as well as privately designed architectures whose
specifications are made public by the designers. The opposite of
open is closed or
proprietary."\footnote{http:\//\//www.webopedia.com\//TERM\//O\//open\_architecture.html}

\item [Portability] "Portability is the easy of transferring
software products to various hardware and software
environments."\cite{meyer97}

\item Architecture Modulaire Ouverte Extensible

\item [Architecture Extensible] "Extendibility is the easy of
adapting software products to change of specification."
\cite{meyer97}

\item [Reusability] "Reusability is the ability of software
elements to serve for the construction of many different
applications." \cite{meyer97}

\item [Compatibility] "Compatibility is the easy of combining
software elements with others." \cite{meyer97}

\item [Module]"A module is a work assignment for a programmer or
programmer team. Each module consists of a group of closely
related programs. The module structure is the decomposition of a
program into modules and the assumption that the team responsible
for each module is allowed to make about the other modules."
\cite{meyer97}

\end{description}
