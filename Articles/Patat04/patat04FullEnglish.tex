\documentclass{llncs}
\bibliographystyle{alpha}
%\usepackage[latin1]{inputenc}
%\usepackage[T1]{fontenc}
\usepackage[applemac]{inputenc}
\usepackage[english]{babel}
\usepackage{dfd,hhline}
\usepackage{verbatim}
\setlength{\parskip}{1.5ex plus0.5ex minus0ex}


\newcommand{\ang}{\textsf}
\newcommand{\key}{\textsf}
\newcommand{\ita}{\textit}
\newcommand{\bld}{\textbf}
\newcommand{\dos}{\textsc}
\newcommand{\pro}{\texttt}

\newcommand{\saphir}{SAPHIR}
\newcommand{\diamant}{DIAMANT}
\newcommand{\tictac}{Tic-Tac}
\newcommand{\xp}{eXtreme Programming}
\def\eXit{$\epsilon$\kern-.100em \lower.5ex\hbox{X}\kern-.125emit}


\title{\bf{A timetable production system architecture for courses and
exams}}

\author{Ruben Gonzalez Rubio \\Domingo Palao Mu�oz}
\institute{D�partement de g�nie �lectrique et de g�nie informatique \\
Universit� de Sherbrooke,\\
Sherbrooke, Qu�bec, J1K 2R1 \\
Canada \\
\email{Ruben.Gonzalez-Rubio@USherbrooke.ca} \\
\email{Domingo.Palao@USherbrooke.ca}}
\date{}



%%%%%%%%%%%%%%%%%%
\pagestyle{plain} %% to be commented when sent
\newcommand{\ints}{\renewcommand{\baselinestretch}{1.0}\small\normalsize}

\newcommand{\intm}{\renewcommand{\baselinestretch}{1.5}\small\normalsize}

\newcommand{\intd}{\renewcommand{\baselinestretch}{2.0}\small\normalsize}

%%%%%%%%%%%%%%%%%%

\begin{document}

\maketitle

\intd

\begin{abstract} \label{intro}
The timetable production in an academic institution is a periodic
task, where the frequency depends on the institution~: each year,
each six-months or each quarter. A part of this production is the
timetable construction for a set of courses. A timetable must
satisfy the hard constraints and try to respect a maximum of soft
constraints. It is recognized that the timetable construction is a
very complex problem, often software helps to carry out this
construction.

In this article, we propose to analyse the timetable construction
problem in a larger view, i.e., to consider the timetable
production as a process which goes from the gathering and data
input until the timetable delivery, including the construction.
Thus, the timetable production can be automated by using a
database, a Web site and a software for timetable construction. We
analyzed the flow of information necessary to the timetable
production, which enabled us to better understand the
characteristics of these systems. We propose a generic
architecture of a system of timetable production for courses and
exams. This architecture was built around two fundamental
qualities of software development~: the opening and extensibility.

\end{abstract}

{\bf Keywords :} Timetable Production,
Timetable Construction,
Open Architecture,
Extensible Architecture,
Databases, Web.


\section{Introduction}
Timetable production is a periodic activity carried out in many
academic institutions. Each year, they produce one or more timetables
for their courses and exams. Computers can help significantly reduce
production costs. In certain cases, automation starts with the use of
timetabling software; in others, with the automatic processing of the
data related with the work of the institution, for example the
follow-up of students.  Some software applications, such as \diamant{}
\cite{gonzalezrubio00}
and \saphir{} \cite{ferland94}, are designed to carry out timetable
construction, but there are many others.

In general, timetable software uses one or more files as
input, and one or more files as output. Working with files generates
problems. It is necessary to build input files each time a timetable
is produced. There are two ways of producing those files: manually or
automatically. If the files are produced manually, typing errors can
occur. Such errors can produce unexpected effects in the software.
Furthermore, they are very hard to detect, because all the data have
to be checked manually. In the other case, the files are produced in
an automatic way. One must ensure that the software can check the
validity and the coherence of all the data.

Another source of problems, and even more expensive than the former
one, is when a change is introduced into the system, either in the
construction software, in the production software or in both. The
change could originate from persons in charge in the institution, for
example a new nomenclature for the buildings, a new teaching
approach, or anything which can affect the way of creating groups,
etc. The problem of modifying software is not solely limited to
timetable production software; it is a general problem in software
development.

The solution to the first problem rests upon the automation of data
checking and data validation at the time of data acquisition, in
order to ensure that the construction software will work with the
right data.

The second problem is more difficult to solve. With a production
software system designed to be modified (i.e. extensible), the cost
of a modification is low, but if the system is monolithic and closed,
any modification will be very expensive. A way of minimizing the
effects of this problem is the development of software featuring
extensibility.

In this article, we propose a timetable production system
architecture for courses and exams having the following qualities of
openness and extensibility. This architecture represents an easy
solution to the two problems mentioned above. It reduces the cost of
modifications, performs data checking and validation in precise
stages of the process, and prevents redundancy of both code and data.
This architecture makes use of a database, a Web site and a software
for timetable construction. Data checking and validation will be made
with the input and with integrity constraints in the database. The
Web site is mainly used for data input, but at the same time, it can
check the data validity, using the business rules before the data are
recorded in the database.
Another function of the Web site is to offer a personalized
timetable, once it is built. The database allows data retrieval,
according to various formats. Thus, the input of the construction
software is simplified. Of course, with the assistance of the
timetable construction software, it is possible
to create a timetable with a minimum of conflicts.


Much research has been done in timetable construction:
algorithms, mathematical formulations, and several softwares were
proposed in PATAT conferences \cite{patat97,patat00,patat02}.
On the other hand, little effort has been devoted, until now, to the
overall development of timetable production. We think that the
fact of viewing the process as a whole can contribute to the research
in timetable construction. White \cite{white00} proposed that the
Web should be used for the diffusion of timetables. We are trying
to go further by proposing to use the Web also as an interface for
data input; moreover, we propose an architecture that can support
support our ideas. De Causmaecker et al. \cite{decausmaecker02}
introduced the idea that the semantics Web can be used by researchers
in timetable construction. They also introduced
the idea of using XML as a communication language in timetable
construction by agents. We also propose the use of XML, in order
to facilitate exchanges between certain modules of our
architecture. In Burke and al \cite{burke97} it is proposed to
have a standard for the data format, which will be used by the
timetable software; their goal was to compare algorithms
(programs) in benchmarks. In a more recent work, Kingston
\cite{kingston02} proposes the STTL language and an interpreter,
with the same objective. Furthermore, �zcan \cite{ozcan03} suggests
another standard to define instances of timetable construction
problems, using XML as a base language. It seems that it would be
desirable to have a standard, but this goal is very difficult to
reach.
So our approach has been to use a database as a tool for
extraction and data formatting. This way, the timetable
construction software can receive the necessary data. Similarly, it
would be
possible to generate other instances of problems for other timetable
construction softwares.

The development of our architecture was divided into two parts.
We first analyzed the data flow in a generic timetable production
system. Then, we defined the architecture, the basic blocks and the
responsibility of each block participating in the architecture.
Our objective of defining an open and extensible architecture was
met during the project. This is presented in Sections
\ref{sec:data} and \ref{sec:archi}.

Once the architecture was defined, we created an instance in order to
implement that architecture in a timetable production system at
the University of Sherbrooke. We reviewed existing
technologies to make the right choices in accordance with
the features of openness and extensibility. This is presented in
Sections   \ref{sec:tech} and \ref{sec:tictac}.


We will finish with an assessment and our conclusion in Section
\ref{sec:conc}.



\section{The Information Flow in Timetable Production}
\label{sec:data}

First, let us define the process of timetable production as a process
which starts when the data for the timetable period $P$ of
validity are entered. This $P$ period is a six-month,
 a quarter, or a year period\footnote{At the University
of Sherbrooke we talk about a quarter, but actually it comprises
sixteen weeks (almost four months), therefore the system must
adapt to any $P$ period.}. The process ends when all the
concerned people receive the schedule on paper or via the Web
(see Figure  \ref{fig:fOne}) and the period of validity of the
schedule comes to an end. Multiple data are entered into the
system, for example courses offered during period $P$, format of
courses (3h, or 2h+1h), assigned or potential
teachers for certain courses, assigned or potential classrooms,
availability of teachers and classrooms, etc. We know that in some
institutions, the students make a pre-selection while in others the
timetable is made with no pre-selection. If any, the students�
pre-choice is part of the data
to be supplied. If there is no pre-choice, it is necessary to
indicate the number of students enrolled in each course.

\begin{figure}[htb] \begin{center}
\setlength{\unitlength}{0.1cm}
\begin{picture}(90,35)
\deffilebox{20}{10}{0.2}
\defprocessbox{20}{16}
\put(0, 15){\fichier[\pro{Data}]}

\put( 20,20){\vector(1,0){15}}

\put(35, 12){\process[Timetable\\Production\\Software]}

\put(55, 20){\vector(1,0){7}} \put(62, 20){\corner[y]{-10}{8}}
\put(62, 20){\corner[y]{10}{8}}

\put(70, 25){\fichier[\pro{Paper}]}

\put(70, 5){\fichier[\pro{Web page}]}
\end{picture}
\end{center}
\caption{Timetable Production}\label{fig:fOne}
\end{figure}

Figure \ref{fig:fOne} is an abstract view of timetable
production, the system of timetable production including timetable
construction software.

\begin{description}
\item[Timetable Production System.] It is composed of several
softwares, or mo\-dules, which work in an autonomous, but
co-ordinated way. It is clear that the system must include two
main components: the system to save data and the timetable
construction software.

\item[Data.] Represents the data required to produce the timetable.
We will save all the necessary data in a data back up system. In the
figure, it is not indicated that these data can be handled by modules
of the timetable production system. For example, the pre-choice
acquisition module will save the
courses followed by a student; the data construction module
will take its inputs from the database, and, when construction
is performed, the new data obtained will be saved in a data
warehouse.


\item[Paper or Web page.] It is the output of the system once the
timetable is built. This output can be saved in files to be printed
on paper or as data to be presented on a Web page. In both cases, the
data source is the database.

\end{description}


\subsection{Data Preparation}
Figure  \ref{fig:fTwo} shows data evolution during timetable
production. The evolution presents instances of the data at different
steps of the process.

The data warehouse can represent one or several databases. We can
think of a distributed database. The choice of the database is an
implementation decision. In a conceptual way it is only one data
warehouse.


\begin{figure}[htb] \begin{center}
\setlength{\unitlength}{0.1cm}
\begin{picture}(90,40)
\deffilebox{20}{10}{0.2}
\defprocessbox{20}{16}
\put(0, 15){\fichier[$bd1_{1}$]}

\put(0, 5){\makebox(20,10){Basic Data}}

\put(20, 20){\vector(1,0){10}}

\put(30, 15){\fichier[$bd2_{2}$]}

\put(30, 5){\makebox(20,10){Period $P$ Data}}

\put(50, 20){\vector(1,0){10}}

\put(60, 15){\fichier[$bd3_{3}$]}

\put(60, 5){\makebox(20,10){Timetable Data}}
\end{picture}%}
\end{center}

\caption{Evolution of Data During Timetable
Production}\label{fig:fTwo}
\end{figure}


\begin{description}

\item[Basic Data.] These data are defined as long-term data
in the data warehouse. Information on the academic activities
of a program, the programs offered, the teachers, the characteristics
of classrooms, the students, all of them are
examples of this type of information. In general, this information
is shared by other applications. For example, a student is
described by his (her) first name, last name, identification number,
program, etc. An application using data on students is
for example that which prints students grades on their evaluation
sheet. Here, we are interested in the data associated with timetable
production, which can be part of the complete institutional
database.

\item[Period $P$ Data.] These data are essential to the construction
of a timetable for a given period. They contain the courses offered
during that period, the availability of teachers for that period,
etc. These data are a prerequisite for timetable construction, and
will be entered before each
construction. However, there will be data updates during the process.
For example when an external teacher is assigned to a given course,
we will enter his (her) name and availability in the database.

\item[Timetable Data.] These data result from timetable
construction. They will contain all the information necessary to
print or show a general and personalized timetable.

\end{description}

In addition to constructing timetables for courses, the system must
be able to build timetables for exams. The evolution of data, in the
case of exams, follows the same path as courses. With the exception
that the data extracted from the database can be different: for
example, for an exam timetable the teachers' availability may not be
necessary.

There are two major activities in timetable production: data entering
(including updates), and timetable construction.

\begin{figure}[htb] \begin{center}
\setlength{\unitlength}{0.1cm}

\begin{picture}(90,40)
 \put(-6,7){ \shortstack{b)}}
 \put(0, 10){\vector(1,0){90}}
 \put(5, 9){\line(0,1){2}}
 \put(4,7){\shortstack{$t0_0$}}
 \put(85, 9){\line(0,1){2}}
 \put(84,7){\shortstack{$tf_f$}}
 \put(5, 12){\vector(1,0){75}}
 \put(45, 14){\vector(1,0){40}}
 \put(9,13){ \shortstack{Data Input}}
 \put(54,16){ \shortstack{Construction}}
 \put(-6,22){ \shortstack{a)}}
 \put(0, 25){\vector(1,0){90}}
 \put(5, 24){\line(0,1){2}}
 \put(4,22){ \shortstack{$t0_0$}}
 \put(5, 27){\vector(1,0){40}}
 \put(45, 27){\vector(1,0){40}}
 \put(85, 24){\line(0,1){2}}
 \put(84,22){\shortstack{$tf_f$}}
 \put(9,28){ \shortstack{Data Input}}
 \put(54,28){ \shortstack{Construction}}

\end{picture}%}
\end{center}

\caption{Data Input and Timetable Construction Activities for
a $P$ Period }\label{fig:fTime}
\end{figure}

Figure  \ref{fig:fTime} shows two cases of activities evolution. In
case $a)$, data input and construction are two activities carried out
in sequence. This represents the ideal case. In case $b)$ which shows
the real
world, data input (especially updates) is done at the same
time as construction. At the beginning of construction, some
data are fixed, but if the software or the person in charge does
not find a satisfactory solution, it becomes necessary to make changes
even in the fixed data.


The activities of data input and construction can be shared
by many users. Each one has a role and associated privileges to
modify a number of data.


Roughly speaking, we consider two types of users: the clerk
and the person in charge of timetable construction. The clerk
is the one who will enter the information into the database. The
person in charge of timetable construction interacts with the
timetable construction program.

The clerk will enter or update the data in the database. Valid data
will be accepted and added to the database. For example if the data
expected is an integer number in a specific range and the input value
corresponds is within the
set values, it will be entered into the database. Otherwise, the
system will force the clerk to return a new value.


Timetable construction is an activity of an iterative nature.
Generally speaking, the algorithms of timetable construction propose
meeting
the hard constraints and trying to comply with the soft constraints
as far as possible. Therefore, the person in charge will try to find
several solutions from which to choose. In certain cases, it is
possible to leave the program with unchanged data and find a new
solution; but in other cases, it is necessary
to change the original data to obtain a new solution. If it is
necessary to make $n$ tests, the solution selected by the person in
charge will be solution $s$ where $1 \leq s \leq n$. One has to
devise a way of working with the database in order to save the
changes carried out for each iteration, and to make a ``commit'' when
the solution is chosen and the timetable becomes final for the period.


\section{The Architecture} \label{sec:archi}


Further to the study of data flows in timetable production,
we are able to propose an architecture for implementing the system.

\subsection{An Architecture for Timetable Production}

Abstractly speaking, software architecture describes the elements of
a system. It also shows the interactions between these
elements, the models governing its composition and the
constraints of these models \cite{shaw96}.

Generally, when facing a complex problem, the best
approach is to break it down into parts that become easier to solve
with
simple solutions. Then, when we combine all these small solutions, we
can find the solution to our complex problem \cite{buschmann96}.

Of all architectures now available for the development of software
applications, the one most appropriate to timetable production systems
is that in layers.

The architecture shown in Figure  \ref{fig:fThree} proposes dividing
the system into three layers, each one with a well-defined function:

\begin{description}
\item [The Interface.] Presents the data to the user, to allow data
input and ensure exchanges with other layers.

\item [The Business Logic Layer.] Ensures data exchanges with the
interface layer. Checks and validates the data input and sends
the data to be presented in an adequate format. It also ensures
data exchanges with the data persistence layer. Business
rules are used to ensure the coherence of the system.

\item [The Data Persistence Layer.] Manages the physical
storage of data in files with a certain format, or in
a traditional database system, or in other persistence models able to
manage complex databases.

\end {description}

\begin{figure}
\setlength{\unitlength}{1cm}
\begin{picture}(3,4.5)
  \put(3,4){\framebox(4,1){Interface}}
  \put(4.5,3.5){\vector(0,1){0.5}}
  \put(5.5,4){\vector(0,-1){0.5}}
  \put(3,2.5){\framebox(4,1){Business Logic}}
  \put(4.5,2){\vector(0,1){0.5}}
  \put(5.5,2.5){\vector(0,-1){0.5}}
  \put(3,1){\framebox(4,1){Data Persistence}}
\end{picture}
\caption{The Three-Layers Architecture}\label{fig:fThree }
\end{figure}



\subsection {The Three-Layers Architecture and the Timetable
Production System.}

For each element of the architecture we�ve introduced in the
earlier subsection, we propose a component that is part of the
timetable production application. In this subsection, we will
show the integration between the data flow for a timetable
production system and the suggested architecture.


The high-level design of our system of timetable production is
simple (see Figure \ref{fig:fOne}):


\begin{enumerate}
\item We defined three categories of data in a timetable
production system. Figure \ref{fig:fTwo} shows these
categories. The interface layer treats this data as follows:


\begin{enumerate}

\item {\bf Basic data.} This type of data is in a database belonging
to
the institution. The database may have its own interfaces for data
input
and display. The production system will use part of such
data. An interface with other systems is required, and its
responsibility is to seek the data required and save them in
the production system. It�s important to note that, for security
reasons, the users of the system should not be allowed to modify the
data of the
institution.


\item {\bf Direct Input Data for Period $P$.} This is performed
by a clerk. Usually, this type of data changes with each timetable
creation.
To enter the data, the clerk has to complete fields on a Web page
displayed by a navigator. At that point, the interface layer can make
some minimal validations of data coherence, for example verify data
type correctness, check that all mandatory fields are filled, etc.
Depending on the size of the institution, there may be several clerks
working at the same time in the application. Consequently, the
necessary measures should be taken to allow concurrent work in the
data persistence layer.

\item {\bf Timetable Data.} Generated by the timetable creation
software. This program is managed by the user in charge of
timetable creation. Since this activity is of an iterative nature,
the user responsible for creating the timetable must analyze the
various versions of the work to find out which one is best suited.
Once he (she) has found that version, he (she) proceeds with the
``commit'' operation to make the information available to the rest of
users. Those are the output data from
our system that are sent to other systems or users. For
example, a student will receive a personalized paper-based
timetable or view the information on a Web site created to this end.
\end{enumerate}

We should keep mind that the interface layer
communicates only with the business logic layer. An application is
unable to add, modify or erase data directly in the database.  It
must inevitably go through the business logic layer. This
characteristic enables us to ensure the coherence of the data.

\item  The data is treated according to certain rules and
constraints. These rules are defined by each institution and they
are coded in the business logic layer. It is here that we can express
constraints; for example, the maximum number of students
per group, the maximum number of courses a professor can
teach, etc.

\item  The data resulting from such processing must be stored in a
place where it can be used after being generated. The data
persistence layer is responsible for this activity. In a
timetable production system, the data can be stored in a database
(relational, object or other technology), an XML file, a
text file or another storage media.
\end{enumerate}

For composition elements and their relationship, see Figure
\ref{fig:fFour}.

\begin{figure}
\setlength{\unitlength}{0.5cm}
\begin{picture}(19,7)

  \put (0,6){\line(1,0){22}}
  \put (19,6.5){\makebox(0,0){\scriptsize Interface Layer}}
  \put(0,6.5){\framebox(4,1){\tiny Data Input}}
  \put(2,6.65){\makebox(0,0){\tiny (Web Client)}}
  \put(5,6.5){\framebox(4,1){\tiny Timetable Creation}}
  \put(7,6.65){\makebox(0,0){\tiny (Interface)}}
  \put(10,6.5){\framebox(4,1){\tiny Other Systems}}
  \put(12,6.65){\makebox(0,0){\tiny (Interface)}}
  \put (2,6.5){\vector(1,-1){2}} %SD -> SW
  \put (4,4.5){\vector(-1,1){2}} %SW -> SD
  \put (7,6.5){\vector(-1,-1){2}} %CH -> SW
  \put (5,4.5){\vector(1,1){2}} %SW -> CH
  \put (7,6.5){\vector(1,-1){2}} %CH -> RA
  \put (9,4.5){\vector(-1,1){2}} %RA -> CH
  \put (12,6.5){\vector(-1,-1){2}} %AS -> RA
  \put (10,4.5){\vector(1,1){2}} %RA -> AS


  \put (0,3){\line(1,0){22}}
  \put (19,3.5){\makebox(0,0){\scriptsize Business Logic Layer}}
  \put (2,3.5){\framebox(4,1){\tiny Web Server}}
  \put (8,3.5){\framebox(4,1){\tiny Business Rules}}
  \put (6,4){\vector(1,0){2}} %RA -> SW
  \put (8,4){\vector(-1,0){2}} %SW -> RA
  \put (10,3.5){\vector(-1,-1){2}} %RA -> BD
  \put (8,1.5){\vector(1,1){2}} %BD -> RA


  \put (0,0){\line(1,0){22}}
  \put (19,0.5){\makebox(0,0){\scriptsize Data Persistence Layer}}
  \put (5,0.5){\framebox(6,1){\tiny Database Server}}




\end{picture}
\caption{Implementation of the Three-layers Architecture for
the Timetable Production System}\label{fig:fFour}
\end{figure}


\section{Technology Resources}\label{sec:tech}


To implement the architecture, we first need to
find out what technological elements are required. The needs are as
follows:

\begin{enumerate}
\item A language to facilitate data display by the user,
and the creation of forms to enter data.

\item A set of rules to define how the data exchange is established
with other systems.

\item A language to code the business logic.

\item A tool to manage data persistence.
\end{enumerate}

Through our analysis, we found there is more than one option:
there are actually several ways of implementing this
architecture.

In Table  \ref{tb:tbOne}, we show an open-ended list of possible
instances.


\begin{table}
\begin{tabular}{|c||c|c|c|}\hline
Technology & Instance 1 & Instance 2 & Instance 3\\ \hline\hline
Language for Display & HTML \& JSP& Applet Java & Visual Basic\\
\hline Format to Data Exchange & XML & Text & Text\\ \hline
Language to Code the Business Logic & Java & Java & Visual Basic\\
\hline
Data Persistence & Database & Database & Database \\
\hline
\end{tabular}
 \caption{Possible Instances of
Architecture}\label{tb:tbOne}
\end{table}

For the construction of each instance, we put together the elements
sharing common characteristics, or belonging to the same
software family, and which are open.

For the database section, we did not mention any specific name of a
database management system. The choice remains open to the
implementation.

We have an open architecture, since it accepts several instances.
The architecture is also extensible, because we could easily change
the
modules. For example, we can start with an element
of data persistence like a file in text format, and then, we could
make some changes to the interface element to be able to
interact with files in XML format. Later, if we want to have
the data persistence element evolve to a database system, it would
be necessary to make changes in the interface layer and to program
the data persistence element. We can make all these changes
without touching the business logic layer. In Table
\ref{tb:tbTwo}, we show the possible evolution of an instance of
our architecture.

\begin{table}
\begin{tabular}{|c||c|c|c|}\hline
Technology & Version 1 & Version 2 & Version 3\\ \hline\hline
Language for Display & HTML \& JSP & HTML and PHP & HTML and JSP\\
\hline
Format to Data Exchange & Text & XML & Database SQL\\
\hline
Language to Code the Business Logic & Java & Java & Java\\
\hline
Data Persistence & Text & XML & SQL, SQL Database \\
\hline
\end{tabular}

\caption{The Evolution of an Instance of the
Architecture}\label{tb:tbTwo}
\end{table}

We can also imagine a system starting with an interface based
in textual mode to show the information, but which will evolve with a
display based in HTML and JSP by using a JSP server.


\section{An Instance of the Architecture} \label{sec:tictac}

Once we�ve established the architecture and we know that it is
open and extensible, it is necessary that the instance we
will implement be open and extensible too. We�ve shown three
options of architecture instances (see Table
\ref{tb:tbOne}). For our implementation, we chose option 1.

First, it should remembered that an element is open if it is built
with approved standards, or if built with a private
specification but made public by the
developers\footnote{http:\//\//www.webopedia.com\//TERM\//O\//open\_architecture.html}.


It is important to remember that an element is extensible if it is
easily adaptable to specification changes \cite{meyer97}.


Bearing these considerations in mind, we will now explain the
analysis that
brought us to our choice:


\begin{description}

\item [Language for Display.] For the language for display and to
enter data we chose HTML and JSP, because those languages are among
the most widely used to create Web pages, and because we want to
benefit from
all the characteristics the Web site can offer for
transactional applications.

\item [Format to Data Exchange.] To exchange data we chose
XML. The XML language (eXtensible Markup Language) allows us
to store, exchange and show data or information in a structured way.
The language also makes it possible to handle, transform and
visualize data with many formatting tools. For example, the
information stored in an XML document can be displayed in a Web
navigator. Moreover, when XML documents are stored in a database,
they can be the object of requests and treated like any other
data.

\item [Language to Code the Business Logic.] We chose Java. Java
is an interpreted language that uses the Java Virtual Machine
(JVM). There are several versions of this virtual machine for
several different platforms, which means that the programmers can
write for this language by using a platform, and then execute
their programs in others.


\item [Data Persistence.] For the implementation of the data
persistence layer, we chose to use a database. A database is an
entity in which it is possible to store data in a structured way
with minimum redundancy. This data must be used by programs
written in different languages by resorting to ODBC standard
technology
(Open DataBase Connectivity). If we use Java as the language to
establish a connection to the database, we can use JDBC (Java
Database Connectivity). The standard JDBC is an application
program (API) to establish the connection between the Java
language and a large number of databases by using the ``Write Once, Run
Anywhere'' philosophy.
\end{description}

Even if it is not the only option for the implementation of our
timetable production system, we are convinced that our choice
will enable us to have the system evolve to other versions, adding or
modifying new characteristics as required by institutions.


The system implementation seems complex. But, we can reuse several
existing modules. This is one of the advantages of the opening and
and the extensibility of the architecture and the instance. We can
analyze the components by layers to know which ones can be reused,
and which ones will be implemented:


\begin{description}

\item [Interface Layer.] For this layer, there is a technological
element which facilitates the work enormously, the Web browser.
This tool is available in most computers with Internet
access. It is necessary to centralize the efforts to develop Web
pages with the quality of compatibility between two Web browsers:
Internet Explorer\footnote{http:\//\//www.microsoft.com},
Netscape\footnote{http:\//\//www.netscape.com}, etc.

The Internet site http://validator.w3.org/, offers a tool to
test the compatibility of a site with the standard HTML. This tool
is provided by the �World
Wide Web Consortium�\footnote{http:\//\//www.w3.org/}.\\

\item [Businesses Logic Layer.] There are two fundamental elements
in this layer: a Web server and a server to deploy the business
logic. Since we chose Java to code the business logic, we can use
Apache HTTP6 as a Web server\footnote{http:\//\//www.apache.org as a
Web server} and, and Jakarta Tomcat\footnote{http:\//\//jakarta.apache.org}
as a server to
provide the business logic.


The ``Apache HTTP'' project is an effort to develop and maintain
an ``Open source'' HTTP server for more widely used operating systems
like UNIX and Windows. The goal of this project is to provide a
secure, effective and extensible server able to offer HTTP
services according to current HTTP standards.

We will program the business rules in some small Java programs
called servlets. A servlet is a Java program used to extend the
functionalities of a Web server. A servlet is a program used to
generate dynamic contents of the applications. It is an
application carried out in the server which is downloaded
dynamically when it is required. Jakarta-Tomcat is a servlet
container.

We can say that the Apache and Jakarta-Tomcat couple enables us to
make the deployment of a complete Web application. Besides, these
tools are free.


\item [Data Persistence Layer.] For the data persistence layer, we
said that we will benefit from the use of JDBC (Java Database
Connectivity). The JDBC is based on the use of a pilot to give
access to several databases, the advantage being that we can change
the pilot according to the database supplier selected. As to the
choice
of the database, we said that it was open to the implementation,
but the standards should always be considered. The latest standard
published by the ANSI (American National Standards
Institute) for the language which processes the data of a database
is SQL-99\footnote{http:\//\//web.ansi.org}. This standard is used
by the most important database developers like
Oracle\footnote{http:\//\//www.oracle.com},
Informix\footnote{http:\//\//www-306.ibm.com/software/data/informix/},
IBM-DB2\footnote{http:\//\//www-306.ibm.com/software/data/db2/udb/},
PostgreSQL\footnote{http:\//\//www.postgresql.org/}, and others.
As to PostgreSQL, using the license is free of charge.
\end{description}


As we have seen in this section, we benefit from several open
technological components which enable us to speed up development
while keeping costs low.



\section{Conclusion}\label{sec:conc}

We�ve presented the architecture of a timetable production system in
an academic institution. An objective guided our approach: to have
an open and extensible architecture so that it would ensure a
long-life system.

We started by showing the needs for information exchange in the
timetable systems in order to explain how the architecture is able
to satisfy those needs. Moreover, in the presentation of the
architecture, we showed that it has two qualities. Our architecture
is open: it is possible to build several instances. The
architecture is also extensible because we could easily change the
modules. The responsibilities for each module are
defined since it is possible to identify the part of the system
that has been modified. It is necessary to continue developing,
keeping in mind the objectives of openness (compatibility) and
extensibility throughout the software development process.

It is now possible to answer the question: ``Can the
timetable construction take advantage of developments associated
with timetable production?'' Our experience in software
development for the timetable construction software showed us that
most problems come from the fact that the input data contains
errors. The construction software must dedicate a large part of
its code to detect those errors. Moreover, users spend time checking
the coherence of the data before beginning the building
process. The answer is then yes, there is a real benefit. By
developing a production system, it is easier to ensure that the
construction software has responsibility for timetable construction
rather than that of checking the data. And it's easier to adapt the
software to the changes.

Other benefits can also be considered, such as the user being able to
consult the timetable on the Web, at home. That is especially well
appreciated in countries where the temperature where the temperature
sometimes drops to $-30^\circ C$.
%%%%%%%%%%%%%%%%%%%%%%%%%%%%%%%%%%%%%%%%%%%%%%%%%%%%%%%%%%%%%%%
%%%%%%%%%%%%%%%%%%%%%%%%%%%%%%%%%%%%%%%%%%%%%%%%%%%%%%%%%%%%%%%
%%%%%%%%%%%%%%%%%%%%%%%%%%%%%%%%%%%%%%%%%%%%%%%%%%%%%%%%%%%%%%%
%%%%%%%%%%%%%%%%%%%%%%%%%%%%%%%%%%%%%%%%%%%%%%%%%%%%%%%%%%%%%%%
%%%%%%%%%%%%%%%%%%%%%%%%%%%%%%%%%%%%%%%%%%%%%%%%%%%%%%%%%%%%%%%
%%%%%%%%%%%%%%%%%%%%%%%%%%%%%%%%%%%%%%%%%%%%%%%%%%%%%%%%%%%%%%%
%%%%%%%%%%%%%%%%%%%%%%%%%%%%%%%%%%%%%%%%%%%%%%%%%%%%%%%%%%%%%%%
%%%%%%%%%%%%%%%%%%%%%%%%%%%%%%%%%%%%%%%%%%%%%%%%%%%%%%%%%%%%%%%
%%%%%%%%%%%%%%%%%%%%%%%%%%%%%%%%%%%%%%%%%%%%%%%%%%%%%%%%%%%%%%%
%%%%%%%%%%%%%%%%%%%%%%%%%%%%%%%%%%%%%%%%%%%%%%%%%%%%%%%%%%%%%%%



\bibliography{bibDiamant,bibProg}


\end{document}



%%%%%%%%%%%%%%%%%%%%%%%%%%%%%%%%%%%%%%%%%%%%%%%%%%%%%%%%%%%%%%%%%%%%%%%%%%%%%%%



\section{Glossaire}


\begin{description}

\item [Architecture logiciel:] ``Abstractly, software architecture
involves the description of elements from which systems are built,
interactions among those elements, patterns that guide their
composition, and constraints on there patterns.'' \cite{shaw96}

\item [Architecture Ouverte:] ``An architecture whose
specifications are public. This includes officially approved
standards as well as privately designed architectures whose
specifications are made public by the designers. The opposite of
open is closed or
proprietary''\footnote{http:\//\//www.webopedia.com\//TERM\//O\//open\_architecture.html}.


\item [Portability] "Portability is the easy of transferring
software products to various hardware and software
environments."\cite{meyer97}

\item Architecture Modulaire Ouverte Extensible

\item [Architecture Extensible] "Extendibility is the easy of
adapting software products to change of specification."
\cite{meyer97}

\item [Reusability] "Reusability is the ability of software
elements to serve for the construction of many different
applications." \cite{meyer97}

\item [Compatibility] "Compatibility is the easy of combining
software elements with others." \cite{meyer97}

\item [Module]"A module is a work assignment for a programmer or
programmer team. Each module consists of a group of closely
related programs. The module structure is the decomposition of a
program into modules and the assumption that the team responsible
for each module is allowed to make about the other modules."
\cite{meyer97}

\end{description}
