
\documentclass{llncs}
\bibliographystyle{alpha}
\usepackage[T1]{fontenc}
\usepackage{dfd,hhline}
\usepackage{verbatim}
\setlength{\parskip}{1.5ex plus0.5ex minus0ex}


\newcommand{\ang}{\textsf}
\newcommand{\key}{\textsf}
\newcommand{\ita}{\textit}
\newcommand{\bld}{\textbf}
\newcommand{\dos}{\textsc}
\newcommand{\pro}{\texttt}

\newcommand{\saphir}{SAPHIR}
\newcommand{\diamant}{DIAMANT}
\newcommand{\tictac}{Tic-Tac}
\newcommand{\xp}{eXtreme Programming}
\def\eXit{$\epsilon$\kern-.100em \lower.5ex\hbox{X}\kern-.125emit}


\title{\bf{The architecture of a timetable production system for courses and exams}}

\author{Ruben Gonzalez Rubio \\Domingo Palao Mu�oz}
\institute{D�partement de g�nie �lectrique et de g�nie informatique \\
Universit� de Sherbrooke,\\
\email{Ruben.Gonzalez-Rubio@USherbrooke.ca} \\
\email{Domingo.Palao@USherbrooke.ca}}
\date{}

%%%%%%%%%%%%%%%%%%
\pagestyle{plain} %% to be commented when sent
\newcommand{\ints}{\renewcommand{\baselinestretch}{1.0}\small\normalsize}
\newcommand{\intm}{\renewcommand{\baselinestretch}{1.5}\small\normalsize}
\newcommand{\intd}{\renewcommand{\baselinestretch}{2.0}\small\normalsize}
%%%%%%%%%%%%%%%%%%

\begin{document}

\maketitle

%\intd

\begin{abstract} \label{intro}
The production of a timetable for courses and exams in an academic 
institution is a periodic activity, the frequency of which depends on 
the institution: yearly, semi-annually, or quarterly. A timetable 
must satisfy all the hard constraints and as many soft constraints as 
possible. Timetable construction is one step in timetable production. 
Since it is a very complex activity, the use of software helps 
carrying it out. In this article, we propose to analyze timetable 
construction from a broader point of view. We will consider the full 
range of  timetable production activities from data gathering and 
data input to timetable delivery, including construction. We analyzed 
the information flow required for timetable production. It led us to 
understand the characteristics of such systems. Timetable production 
can be
automated by using databases, a Web site, and timetable construction 
software. We propose a generic architecture of a timetable production 
system for courses and exams. We�ve developed that architecture 
around two fundamental qualities of software development: openness 
and extensibility.


\end{abstract}

{\bf Keywords :} Timetable production, Timetable construction,
Open architecture, Extensible Architecture, Databases, Web.

\section{Introduction}

The timetable production is a periodic task carried out in many
institutions of teaching. These institutions produce one or more
timetable per year for the courses and the exams. The automation
of this production can reduce its cost in an important way. In
certain cases, automation starts with the use of a software of
timetable construction; in others, it starts with the automation
of the treatments associated with the work of the institution, as
for example the follow of students. There are software to carry
out the timetable construction, by example DIAMANT
\cite{gonzalezrubio00} and SAPHIR \cite{ferland94}, but, there
exists many others.

In general, the software for timetable creation uses as input one
or more files. Their output is saved on one or more files. The
work with files is a source of problems. It is necessary to build
the input files each time that a timetable is produced. There
exists, two cases~: a manual production or an automated production
of the files . If the files are produced manually, there are the
typing errors that can produce unexpected effects in the software.
This kind of errors are very hard to find, because all the data
should be checked manually. In the case, where the files are
produced in an automatic way, it is necessary that the software is
ensured to check the validity and the coherence of all the data.
The second problem appears when a change is introduced into the
system (either in the application of construction, or in that who
produce the files) or if the application of construction is used
in several different sites, using different data formats.

Another source of problems even more expensive than the preceding
one, is that when a change is introduced into the system, it could
be in the software of construction, or in the software of
production or in both. The changes come on behalf of persons in
charge for the institution, for example a new nomenclature for the
buildings; a new teaching approach, which change the way to create
the groups, etc. The problem to modify a software is not only for
the timetable production software, it acts as a general problem in
the software development.

The solution to the first problem passes by the automation of the
checking and validation of data at the time of data acquisition,
in order to ensure that the software of construction works with
the right data. The second problem, is more delicate to treat, if
the software of production were conceived to be modified
(extensible), the cost to make a modification is weak, but if the
system is monolithic and closed, a modification will be very
expensive. A way of decreasing the effects of this problem is the
development of software having the quality of extensibility.

In this article we propose the architecture of a system of
timetable production for courses and exams, having the following
qualities: open and extensible. This architecture must contribute
to more easily work with the two problems mentioned, by reducing
the cost of the modifications and by carrying out the checking and
the validation of the data in precise stages of the process and by
avoiding the redundancy of the code and the data. This
architecture implements a database, an Web site and a software of
timetable construction. The checking and the validation of data
will be made with the input and with integrity constraints in the
database. The Web site is used mainly for the data input, but at
the same time, it can check the validity of the data, using the
business rules before the data will be entered to the database.
Another function of the Web site is to offer the timetable in a
personalized way, once that it is built. The database allows the
retrieval of the data, according to various formats. Thus, the
input of the software of construction is simplified. Of course,
with the assistance of the software of timetable construction, it
is possible to create a timetable with a minimum of conflicts.


Many research was made in the timetable construction~: algorithms,
mathematical formulations and several software were proposed in
the PATAT conferences \cite{patat97, patat00, patat02}. On the
other hand, few efforts were devoted, until now, to the general
development of the timetable production. We think that the fact of
having a total view on the process can help the research in
timetable construction. White \cite{white00} highlighted that the
Web must be used to the diffusion of timetables. We are trying to
go far proposing also to use it like an interface of data input,
moreover we propose the architecture necessary to support our
ideas. De Causmaecker et al. \cite{decausmaecker02} introduced the
idea that the semantics Web can be used by the researchers in the
timetable construction. They also introduced the idea to use XML
like a language of communication in the timetable construction by
agents. We propose also the use of XML, in order to facilitate the
exchanges between certain modules of our architecture. In Burke
and al \cite{burke97} it is proposed to have a standard for the
data format, it will be used by the timetable software, their goal
was to compare algorithms (programs) in benchmarks. In a more
recent work, Kingston \cite{kingston02} proposes the STTL language
and an interpreter, with the same objective. �zcan \cite{ozcan03}
suggests also another standard to define instances of problems of
timetable construction, having XML as a base language. It seems
that it would be desirable to have a standard, but it is a goal
very hard to reach. At the place, our approach is the use of a
database like a tool of extraction and data formatting, in this
way the software of timetable construction can receive the
necessary data. Thus, it would be possible to generate other
instances of problems for other software of timetable
construction.

The development of our architecture, was divided into two parts,
we analyzed the data flow in a generic system of timetable
production. Then we define an architecture, the basic blocks and
the responsibility of each block which takes part in the
architecture. The objectives to define an open and extensible
architecture were respect during our work. This is presented in
the sections \ref{sec:data} and \ref{sec:archi}.

Once defined the architecture, we created an instance in order to
establish it in a system of timetable production at the University
of Sherbrooke. We made an overflight of existing technologies in
order to make the good choices still respecting the objectives of
opening and extensibility. This is presented in the sections
\ref{sec:tech} and \ref{sec:tictac}.


We finish by an assessment and the conclusions in the section
\ref{sec:conc}.


\section{The information flow in the timetable production}
\label{sec:data}

First of all, we define the process of timetable production as a
process which starts at the time to enter the data for the period
of validity of the timetable $P$. This period $P$ is one six-month
period, one quarter, or a year period\footnote{In the University
of Sherbrooke we talk about quarter, but it really comprises
sixteen weeks (almost four months), therefore the system must be
adapted to any period $P$.}. The process ends when all the
concerned people receive the schedule on a paper or via the Web
(see figure \ref{fig:fOne}) and the period of validity of the
schedule touches at its end. The are multiple data to feed to the
system, for example the courses which will take place during the
period $P$; the format of the courses (3h, or 2h+1h); possible or
assigned teachers for certain courses; possible or assigned
classrooms; the availability of the teachers and the classrooms,
etc. We know that there are institutions where the students make a
pre-choice and others where the timetable are made without
pre-choice. The pre-choice of the students is part of the data to
be supplied, if there is no pre-choice, it is necessary to
indicate the number of students to be admitted for each course.

\begin{figure}[htb] \begin{center}
\setlength{\unitlength}{0.1cm}
\begin{picture}(90,35)
\deffilebox{20}{10}{0.2}
\defprocessbox{20}{16}
\put(0, 15){\fichier[\pro{Data}]}

\put( 20,20){\vector(1,0){15}}

\put(35, 12){\process[Timetable\\production\\software]}

\put(55, 20){\vector(1,0){7}} \put(62, 20){\corner[y]{-10}{8}}
\put(62, 20){\corner[y]{10}{8}}

\put(70, 25){\fichier[\pro{Paper}]}

\put(70, 5){\fichier[\pro{Web page}]}
\end{picture}
\end{center}
\caption{Timetable production}\label{fig:fOne}
\end{figure}

The figure \ref{fig:fOne} is an abstract view of the timetable
production, the system of timetable production includes a software
of timetable construction.

\begin{description}
\item[System of timetable production.] It is composed of several
software, or modules, which works in an autonomous way, but
coordinated. Its clear that the system must have two principal
components~: the system to save data and the software of timetable
construction.

\item[Data.] Represent the necessary data to the timetable
production. We will save all the necessary data in a system of
data back-up. In the figure, it is not indicated that these data
can be handled by modules of the system of timetable production.
For example, the module of acquisition of pre-choice, will save
the courses followed by a student. The module of data construction
will take its entry in the database and when the construction is
made the new data obtained will be saved in a data warehouse.


\item[Paper or Web page.] It is the output of the system once that
the timetable is built. These output can to be files, which will
be printed on paper or data that will be presented on a Web page,
in both cases the data sources are the database.

\end{description}


\subsection{Data preparation}
The figure \ref{fig:fTwo} shows the evolution of the data during
the timetable production. The evolution presents the instance of
the data at different moments of the process.

The data warehouse can represent only one database or several
databases (distributed databases) it is a decision of the
implementation. In a conceptual way it can be a data warehouse or
only one database.

\begin{figure}[htb] \begin{center}
\setlength{\unitlength}{0.1cm}
\begin{picture}(90,40)
\deffilebox{20}{10}{0.2}
\defprocessbox{20}{16}
\put(0, 15){\fichier[$bd_{1}$]}

\put(0, 5){\makebox(20,10){Basic data}}

\put(20, 20){\vector(1,0){10}}

\put(30, 15){\fichier[$bd_{2}$]}

\put(30, 5){\makebox(20,10){Period $P$ data}}

\put(50, 20){\vector(1,0){10}}

\put(60, 15){\fichier[$bd_{3}$]}

\put(60, 5){\makebox(20,10){Timetable data}}
\end{picture}%}
\end{center}

\caption{Evolution of the data during the production of a
timetable}\label{fig:fTwo}
\end{figure}


\begin{description}

\item[Basic data.] These data is defined as the long term data in
the data warehouse. The information for the academic activities of
a program, the programs offered, the teachers, the characteristics
of the classrooms, the students, all of them are examples of this
type of information. In general this information is shared by
other applications. For example a student is described by his
first name, last name, number identification, program, etc. An
application which uses the data students is for example that which
print the notes of a student on his scorecard. Here, we are
interested in the data associated with the timetable production,
which can be a view (partially) of the complete institutional
database.

\item[Period $P$ data.] These data are necessary to the timetable
construction for the concerned period, they contain the courses
which will be offered in this period, the availability of teachers
for this period, etc. These data are a precondition to the
timetable construction and will be entered before each
construction. However, these data will change (update) during the
timetable construction. For example when a external teacher is
assigned for a given course, we will enter his information name
and availability in the database.

\item[Timetable data.] These data are the result of the timetable
construction. They will contain all the necessary information to
print or to show general and personalized timetable.

\end{description}

Additionally of the timetable construction for courses, the system
must be used to build the timetable for exams. The evolution of
data, in the case of exams, follows the same path as courses. With
the exception that the data extracted from the database can be
different, for example, for a exam timetable the availability of
the teachers can not be necessary.

There are two great activities in the timetable production: enter
the data (including the update) and the timetable construction.

\begin{figure}[htb] \begin{center}
\setlength{\unitlength}{0.1cm}

\begin{picture}(90,40)
 \put(-6,7){ \shortstack{b)}}
 \put(0, 10){\vector(1,0){90}}
 \put(5, 9){\line(0,1){2}}
 \put(4,7){\shortstack{$t_0$}}
 \put(85, 9){\line(0,1){2}}
 \put(84,7){\shortstack{$t_f$}}
 \put(5, 12){\vector(1,0){75}}
 \put(45, 14){\vector(1,0){40}}
 \put(9,13){ \shortstack{Data input}}
 \put(54,16){ \shortstack{Construction}}
 \put(-6,22){ \shortstack{a)}}
 \put(0, 25){\vector(1,0){90}}
 \put(5, 24){\line(0,1){2}}
 \put(4,22){ \shortstack{$t_0$}}
 \put(5, 27){\vector(1,0){40}}
 \put(45, 27){\vector(1,0){40}}
 \put(85, 24){\line(0,1){2}}
 \put(84,22){\shortstack{$t_f$}}
 \put(9,28){ \shortstack{Data input}}
 \put(54,28){ \shortstack{Construction}}

\end{picture}%}
\end{center}

\caption{Activities of data input and construction for one period
$P$}\label{fig:fTime}
\end{figure}

The figure \ref{fig:fTime} shows two cases of the evolution of
these activities. In the case A the data input and the
construction are two activities carried out in sequence. This
arrives in very rare cases. In the case B, who shows the reality,
the data input (especially the update) is done at the same time as
the construction. This is the general case. Much of the data are
fixed, as the offer of course, however, during the construction we
find situations where the conflicts oblige to make changes on the
data, by example, the format of a course, who pass from 3h
consecutive to 2h consecutive more 1h. Other cases can be
presented.


The activities of data input and construction can be carried out
by many users. Each one will have an associated role and
privileges to be able to change the data permitted by its role.


In order to simplify, we consider two types of users~: the clerk
and the person in charge for the timetable construction. The clerk
is that which will enter the information in the database. The
person in charge for the timetable construction interacts with the
timetable construction program.

The clerk will enter or update the data in the database, if the
data is valid, it will be accepted and committed in the database.
For example if the data waited is an integer number in a specific
range, and the input value correspond with the attended value, it
will be entered in the database. If not, the system must force the
clerk to return a new value.


The construction of schedules is an activity of iterative nature.
In general, the algorithms of timetable construction propose to
respect the hard constraints and to try to respect soft
constraints as much as possible, therefore, the person in charge
will try to find several solutions to choose one of them. In
certain cases, it is possible to leave the program with the same
data to find a new solution, but in other cases it is necessary to
change the starting data to obtain a new solution. If it is
necessary to make $n$ tests the solution chosen by the person in
charge will be the solution $s$ where $1 \leq s \leq n$. It is
necessary to envisage a way of working with the database in order
to save the changes carried out for each iteration and to make a
''commit'' when the solution is chosen and that the timetable
becomes final for the period.



\section{The architecture} \label{sec:archi}


Following the study of the data flow in the timetable production ,
we are able to propose an architecture to implement the system.

\subsection{An architecture for the timetable production}

In an abstracted way, a software architecture describes the
component elements of a system, it shows also the interactions
between these elements, the models which guide its composition and
the constraints of these models. \cite{shaw96}

In a general way, when we have a complex problem, the best
approach is to cut out the problem in fragments which are easier
to solve with a simple solution. Then, if we put all these small
solutions together, we will arrive at the solution of our
complexes problem \cite{buschmann96}.

At the present time there are several architectures available to
develop software applications, the most adapted for this type of
problem is that in layers.

The architecture shown in the figure \ref{fig:fThree} proposes the
division of the system in three layers, each one with a well
defined function~:

\begin{description}
\item [The interface~:] Is responsible to validate the data
acquisition of the user, as well as the entry of the data which
comes from other systems.

\item [Business logic layer~:] Is responsible for the behavior of
the system, i.e., the rules of businesses are coded here.

\item [The data persistence layer~:] Manages the physical storage
of the data, that is in files with a certain format, or in a
system of traditional databases or in other models of persistence
which are able to manage complex databases.
\end {description}

\begin{figure}
\setlength{\unitlength}{1cm}
\begin{picture}(3,4.5)
  \put(3,4){\framebox(4,1){interface}}
  \put(4.5,3.5){\vector(0,1){0.5}}
  \put(5.5,4){\vector(0,-1){0.5}}
  \put(3,2.5){\framebox(4,1){business logic}}
  \put(4.5,2){\vector(0,1){0.5}}
  \put(5.5,2.5){\vector(0,-1){0.5}}
  \put(3,1){\framebox(4,1){data persistance}}
\end{picture}
\caption{The three layers architecture .}\label{fig:fThree}
\end{figure}

\subsection {The three layers architecture and the system of timetable production.}


For each element of the architecture that we presented in the
earlier section, we propose a component which is part of the
application of timetable production. In this section we will show
the integration between the data flow for a system of timetable
production and suggested architecture.



The high level design of our system of timetable production is
simple (see figure \ref{fig:fOne})~:

\begin{enumerate}
\item We have data which enter the system. These data are managed
by the interface layer. There are three possible sources of data
(see figure \ref{fig:fTwo}):

\begin{enumerate}
\item Direct data entry for one period data. This activity is made
by the made user clerk. Normally, these data change with each
creation of timetable. To enter the data, the user clerk has a Web
page which is showed by a navigator where it will have all the
fields to fill. At this moment, the interface layer can make some
minimal validations of the coherence of data, for example, check
that the type of data is correct, to check that the fields which
are obligatory are filled, etc. Depending on the size of the
institution, it can have several users clerk who work at the same
time in the application, in that case it is necessary, to take
right measures to allow the concurrent work in the data
persistence layer.

\item Data generated by other systems (basic data). It is proven
long time ago that the best way of exchanging information between
two systems is the file transfer. For this reason our system must
be able to receive and send files in a compatible format between
several systems.

\item The timetable data are created by the application of
timetable creation. This application is managed by the user
responsible for the timetable creation. Since this activity is of
iterative nature, it can have a certain number of versions of work
that the user responsible for the timetable creation must analyze
to find that which is appropriate to him. Once that he found this
version, he must apply the "commit"operation to make available the
information to the rest of the users.

Those are the data that leave our system, and they are sent
towards to other systems or to other users, for example, the
students who will receive a paper with their timetable
personalized or, he can consult this information by using a Web
site creates for these ends, or, the central administration of the
institution which must be informed of the assignment of the
schedules.

\end{enumerate}

It is necessary to keep in consideration that the interface layer
communicates only with the business logic layer. An application is
not able to add, modify or to erase data directly in the data
base, it must pass irremediably by the business logic layer. This
characteristic enables us to ensure the coherence of the data.

\item The data are treated according to certain rules and
constraints, these rules are defined by each institution and they
are coded in the business logic layer. It is here that we will be
able to express the constrains for each institution, for example,
the maximum quantity of students in a group, that all the courses
of mathematics are held the morning, etc.

\item The data results of these treatments must be store in a
place to be able to use them after its generation. It is the layer
of data persistence which is responsible for this task. In a
system of timetable production, the data can be stored in a data
base (relational, object or another technology), in a XML file, a
text file or in another storage media.
\end{enumerate}

These elements of composition and its relations are illustrated in
the figure \ref{fig:fFour}.

\begin{figure}
\setlength{\unitlength}{0.5cm}
\begin{picture}(19,7)

  \put (0,6){\line(1,0){22}}
  \put (19,6.5){\makebox(0,0){\scriptsize Interface layer}}
  \put(0,6.5){\framebox(4,1){\tiny Data input}}
  \put(2,6.65){\makebox(0,0){\tiny (Web client)}}
  \put(5,6.5){\framebox(4,1){\tiny Timetable cr�ation}}
  \put(7,6.65){\makebox(0,0){\tiny (Interface)}}
  \put(10,6.5){\framebox(4,1){\tiny Other syst�mes}}
  \put(12,6.65){\makebox(0,0){\tiny (Interface)}}
  \put (2,6.5){\vector(1,-1){2}} %SD -> SW
  \put (4,4.5){\vector(-1,1){2}} %SW -> SD
  \put (7,6.5){\vector(-1,-1){2}} %CH -> SW
  \put (5,4.5){\vector(1,1){2}} %SW -> CH
  \put (7,6.5){\vector(1,-1){2}} %CH -> RA
  \put (9,4.5){\vector(-1,1){2}} %RA -> CH
  \put (12,6.5){\vector(-1,-1){2}} %AS -> RA
  \put (10,4.5){\vector(1,1){2}} %RA -> AS


  \put (0,3){\line(1,0){22}}
  \put (19,3.5){\makebox(0,0){\scriptsize Business logic layer}}
  \put (2,3.5){\framebox(4,1){\tiny Web server}}
  \put (8,3.5){\framebox(4,1){\tiny Business rules}}
  \put (6,4){\vector(1,0){2}} %RA -> SW
  \put (8,4){\vector(-1,0){2}} %SW -> RA
  \put (10,3.5){\vector(-1,-1){2}} %RA -> BD
  \put (8,1.5){\vector(1,1){2}} %BD -> RA


  \put (0,0){\line(1,0){22}}
  \put (19,0.5){\makebox(0,0){\scriptsize Data persistance layer}}
  \put (5,0.5){\framebox(6,1){\tiny Database server}}




\end{picture}
\caption{The implementation of the three layers architecture for
the system of timetable production.}\label{fig:fFour}
\end{figure}


\section{Possible technologies}\label{sec:tech}

For the implementation of the architecture, first of all it is
necessary, to find which are the technological elements that we
need. We find that it is necessary to have~:

\begin{enumerate}
\item A language to facilitate the show of the data to the user
and the creation of forms to enter data.

\item A set of rules which define the way of establishing the
exchange of data with the other systems.

\item A language to code the business logic.

\item A tool to manage the data persistence.
\end{enumerate}

After our analysis, we found that there is not only one option,
rather there are several possibilities to implement this
architecture.

In the Table \ref{tb:tbOne}, we show a list of possible instances,
it is not exhaustive.


\begin{table}
\begin{tabular}{|c||c|c|c|}\hline
Technology & Instance 1 & Instance 2 & Instance 3\\ \hline\hline
Language to show & HTML et JSP& Applet Java & Visual Basic\\
\hline Format to data exchange & XML & Text & Text\\ \hline

Language to code the business logic & Java & Java & Visual Basic\\
\hline

The data persistence & Database & Database & Database \\
\hline
\end{tabular}
~ \\
 \caption{ Some possible instances of the architecture}\label{tb:tbOne}
\end{table}

For the construction of each instance we considered a certain
logic, we have put together the elements which have common
characteristics or which belong to the same software family and
which are open, it is a criterion of construction which was
selected in an arbitrary way.

For the database section, we did not mention a specific name of a
database management system, the choice remains open to the
implementation.

We have an open architecture, since it accepts several instances.
The architecture is also extensible, because we could change the
modules in an easy way, for example, we can start with an element
of data persistence like a file in text format, and then, we could
carry out some changes in the element of interface to be able to
interact with files in XML format. After, if we want to make
evolve this part of data persistence to a system of databases, it
would be necessary to make changes in the interface layer and to
program the element of data persistence. All these changes without
never touching the element of the business logic. In Table
\ref{tb:tbTwo}, we show the possible evolution of an instance of
our architecture.

\begin{table}
\begin{tabular}{|c||c|c|c|}\hline
Technologie & Version 1 & Version 2 & Version 3\\ \hline\hline
Language to show & HTML et JSP & HTML et JSP & HTML et JSP\\
\hline

Format to data exchange & Text & XML & Database SQL\\
\hline

Language to code the business logic & Java & Java & Java\\
\hline

The data persistence & Text & XML & SQL, Database SQL \\
\hline

\end{tabular}
~ \\

\caption{ The evolution of an instance of the
architecture}\label{tb:tbTwo}
\end{table}

We can also imagine a system which starts with an interface based
in textual mode to show the information, which will evolve with a
display based in HTML and JSP by using a Web server.

\section{An instance of the architecture} \label{sec:tictac}

Once that we established the architecture and that we know that it
is open and extensible, it is necessary that the instance that we
will implement will be open and extensible too. We showed three
options of instance of the architecture (see Table
\ref{tb:tbOne}), for our implementation, we chose the option 1 of
this table.


Before begins, it should be known that an element is open if it is
built with approved standards, or, if he is built with a
specification private but made public by the developers.
\footnote{http:\//\//www.webopedia.com\//TERM\//O\//open\_architecture.html}

It also should be held in consideration that an element is
extensible if it is easy to adapt this product to the changes in
the specification.\cite{meyer97}


With these considerations in hand, we show here the analysis which
brought us to our choice:

\begin{description}

\item [Language to show:] For the language to show and to enter
data we chose HTML and JSP, given that those languages are the
standard for the creation of Web pages, and we want to benefit
from all the characteristics that the Web offer for the
transactional applications.

\item [Format to data exchange:] To exchange the data we chose
XML. The language XML (extensible Markup Language) let us to
store, exchange and show data or information in a structured way.
The language also makes possible to handle, to transform and to
visualize the data with many tools of formatting, for example, the
information or data stored in a document XML can be displayed in a
Web navigator. Moreover, when XML documents are stored in a
database, they can be the object of requests and found like any
other data format.

\item [Language to code the business logic:] We chose Java. Java
is an interpreted language which uses the Java Virtual Machine
(JVM). There are several versions of this virtual machine for
several different platforms, which means that the programmers can
write for this language by using a platform, and make turn their
programs on others.

\item [The data persistence:] For the implementation of the data
persistence layer we chose to use a database. A database is an
entity in which it is possible to store data in a structured way
with the minimum of redundancy. These data must be able to be used
by programs written in various languages by using standard
technology ODBC (Open DataBase Connectivity), if we use Java as
language to establish the connection towards the database, we can
use JDBC (Java Database Connectivity). The standard JDBC is an
application program (API) to establish the connection between the
Java language and a large numbers databases, by using philosophy
"Write Once, Run Anywhere".
\end{description}

Even if it is not the only option for the implementation of our
system of timetable production, we are convinced that our choice
will enable us to make evolve the system by versions adding or
modifying the new characteristics required by the institutions.


\section{Conclusions}\label{sec:conc}

We presented an architecture for the timetable production in an
academic institution. An objective guided our approach~: to have
an open and extensible architecture. In the presentation of the
architecture we showed that it has the two qualities. Our
architecture is open, since it accepts several instances. The
architecture is also extensible, because we could change the
modules in an easy way.


We arrived to the objective to model the timetable production
process to make decisions allowing its automation. This can
facilitate the development and the establishment of a system on
two points of view~:

\begin{itemize}

\item The development of software interfaces between various
components of the system, this can result in developing modules or
with the modification of existing modules.

\item The gradual integration of new modules in the time, in order
to disturb the least possible the users of the system.
\end{itemize}

In the system that we propose, we try as much as possible to limit
of human intervention in order to reduce errors of handling. We
try to also to reduce the possible sources of errors by checking
the input.

Future work is to continue to conceive the complete system by
keeping the hot lines of extensibility and opening.

\bibliography{bibDiamant,bibProg}


\end{document}



%%%%%%%%%%%%%%%%%%%%%%%%%%%%%%%%%%%%%%%%%%%%%%%%%%%%%%%%%%%%%%%%%%%%%%%%%%%%%%%


\section{Glossaire}


\begin{description}

\item [Architecture logiciel:] "Abstractly, software architecture
involves the description of elements from which systems are built,
interactions among those elements, patterns that guide their
composition, and constraints on there patterns." \cite{shaw96}

\item [Architecture Ouverte:] "An architecture whose
specifications are public. This includes officially approved
standards as well as privately designed architectures whose
specifications are made public by the designers. The opposite of
open is closed or proprietary."
\footnote{http:\//\//www.webopedia.com\//TERM\//O\//open\_architecture.html}

\item [Portability] "Portability is the easy of transferring
software products to various hardware and software
environments."\cite{meyer97}

\item Architecture Modulaire Ouverte Extensible

\item [Architecture Extensible] "Extendibility is the easy of
adapting software products to change of specification."
\cite{meyer97}

\item [Reusability] "Reusability is the ability of software
elements to serve for the construction of many different
applications." \cite{meyer97}

\item [Compatibility] "Compatibility is the easy of combining
software elements with others." \cite{meyer97}

\item [Module]"A module is a work assignment for a programmer or
programmer team. Each module consists of a group of closely
related programs. The module structure is the decomposition of a
program into modules and the assumption that the team responsible
for each module is allowed to make about the other modules."
\cite{meyer97}

\end{description}
