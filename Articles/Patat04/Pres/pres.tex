%% BEGIN Pr�sentation de  PILE aux Profs
%\title Pr�sentation de  PILE
%\author : Ruben Gonzalez Rubio
\newcommand{\saphir}{SAPHIR}
\newcommand{\diamant}{DIAMANT}
\newcommand{\tictac}{Tic-Tac}
\newcommand{\xp}{eXtreme Programming}
\def\eXit{$\epsilon$\kern-.100em \lower.5ex\hbox{X}\kern-.125emit}



%%%0
\bslide
\bheading A timetable production system architecture for courses and 
exams
\eheading

\begin{center}

Ruben Gonzalez Rubio 

and 

Domingo Palao Mu�oz

D�partement de g�nie �lectrique \\
et de g�nie informatique \\
Universit� de Sherbrooke,\\
Sherbrooke, Qu�bec, J1K 2R1 \\
Canada \\
\pro{Ruben.Gonzalez-Rubio@USherbrooke.ca} 

\end{center}

\eslide

%\bnote

%\enote



%%%0
\bslide
\bheading Presentation plan
\eheading

\bi
\item Motivation
\item Design patterns
\item The Test Condition Pattern
\bi

\item Construction of a timetable for exams
\item The Pattern 

\item The code

\ei


\item Conclusions

\ei

\eslide

%\bnote

%\enote
%%%0
\bslide
\bheading Motivation
\eheading

\bi
 \item A
timetable construction software must offer a flexible way to handle 
constraints. 
\ei

\eslide

%%%1% 


%\enote
%%%0
\bslide
\bheading  Design Patterns
\eheading

 \begin{itemize} 
      \item ``Design patterns are recurring solutions to design 
problems you see over and over''. 
      \item ``Design patterns constitute a set of rules describing 
how to 
     accomplish certain tasks in the realm of software development''. 
	 \item ``Patterns identify and specify 
abstractions that 
     are above the level of single classes and instances, or of 
     components''. 
 \end{itemize} 
 

\eslide




%%%0
\bslide
\bheading The Problem
\eheading


The construction of a timetable for exams.




\eslide


%%%0
\bslide
\bheading Problem Input
\eheading



 
\begin{itemize} 
\item A set of exams. 
\item A set of students who are taking exams; 
each student is associated with the exams he/she must take.
\item To simplify the problem,  one room 
with a 
given capacity. 
\item An empty timetable. 
The timetable is a data structure containing a number 
of timeslots which are empty. 
The algorithm takes one exam at the time, 
then it assigns it to a timeslot. 
Here assign means that the exam will take place at the assigned 
timeslot. 
\end{itemize} 


\eslide


\bslide
\bheading Problem Constraints
\eheading

The timetable covers $n$ days spread in $m$ weeks, 
the duration for each exam is t hours ($t=3$), 
and there are two timeslots each day: 
morning and afternoon (9:00 to 12:00 and 14:00 to 17:00). 

The hard  constraints (conditions) to satisfy are:
\begin{itemize} 
\item  No student must have two exams at the same time. 
\item The number of students in a period is less that the total 
capacity. 
\end{itemize} 
 
Any produced timetable must respect these constraints. Each 
constraint 
could be seen as a condition to be respected. 

\eslide

\bslide
\bheading Problem output
\eheading

 

\begin{itemize} 
\item The timetable where each timeslot is associated to the exams 
taking place at this timeslot without conflicts.  
\item A set of \ita{exams that are impossible to assign to a 
timeslot}. If all exams were assigned this set is empty. 
\end{itemize} 
\eslide

%%%1






\bslide
\bheading The algoritm
\eheading

 
\begin{enumerate} 
\item While the set of exams is not empty do : 
\begin{enumerate} 
\item Take out the exam with the greatest potential number of 
conflicts from the 
set of exams. 
\item Assign the exam to the first timeslot where the conditions are 
respected (\ita{no conflict with exams in this period} $ \wedge$ 
\ita{number of students in this timeslot less than capacity}). If 
this is impossible the exam then it will be placed in the set of 
\ita{exams that are impossible to assign to a timeslot}. 
\end{enumerate}
\item If the set of \ita{exams impossible to assign to a timeslot} is 
empty then print Timetable is ok else print Timetable is not ok 
\end{enumerate} 

\eslide

%\bnote

%\enote
%%%0
\bslide
\bheading Adding Constraints
\eheading

{\it All Mathematics exams must be held during the morning}. 

In order 
to handle this new condition it must be possible to recognize whether 
an exam is a mathematics exam and whether a period is a morning 
period. 

The algorithm remains basically the same, but a new (composed) 
condition  must be added:  \ita{is this a mathematics exam?} $ 
\wedge$  \ita{is this a morning period?}. This new condition must be 
respected only if the exam is a mathematics one; otherwise, the 
algorithm uses the original conditions. 

\eslide

%\bnote

%\enote







%%%0
\bslide
\bheading The Pattern
\eheading

A source of inspiration for the pattern was the Command pattern, 
which encloses the request for a specific action inside an object and 
gives a know public  \pro{interface}.

\eslide

%\bnote

%\enote



%%%0
\bslide
\bheading Condition
\eheading

We define the interface \pro{Condition} as follows: 
 
\begin{verbatim} 
public inteface Condition { 
   public boolean doTest() 
} 
\end{verbatim} 


\eslide

%\bnote

%\enote


%%%0
\bslide
\bheading Testing Conditions
\eheading

\begin{verbatim} 
... 
public void buildTimeTable() { 
... 
   for(int i = 0; i < v.size(), i++) { 
      Condition c = (Condition) v.elementAt(i); 
      if (!c.doTest()) 
         break; 
   } //end for 
   ... 
} 
... 
\end{verbatim}

\eslide

%\bnote

%\enote
	

 
%%%0
\bslide
\bheading A Condition
\eheading

\begin{verbatim} 
public class Cond1 implements Condition { 
   private Timetable tt; 
   ... 
   public boolean doTest() { 
      Timeslot cP = tt.getCurrentTimeslot(); 
      Exam cE = tt.getCurrentExam(); 
      retour cP.noConflict(cE); 
   } 
   ... 
} 
\end{verbatim} 

\eslide


\bslide
\bheading Advantages and disadvantage
\eheading

The main advantage of the proposed Design pattern is flexibility.

The main disadvatage is efficiency compared with a typical 
implementation but the execution time depends on the algorithm.

\eslide

\bslide
\bheading Conclusion
\eheading

We present a Design pattern to test conditions in timetable 
construction problems. We give the motivations for the Design 
pattern, we show the implementation of the Design pattern and 
evaluate the advantages and disadvantages. 
 
As with many Design patterns, its description is very easy to 
understand and to reuse in other cases. 

\eslide


