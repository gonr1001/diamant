La production d'horaires dans une institution d'enseignement est
une t�che p�riodique, dont la fr�quence d�pend de l'institution~:
chaque ann�e, chaque semestre ou chaque trimestre. Une partie de
cette production est la construction d'un horaire pour un ensemble
de cours. Un horaire doit satisfaire des contraintes fortes et
essayer de respecter un maximum de contraintes faibles. Il est
reconnu que la construction d'horaires est un probl�me tr�s
complexe, souvent des logiciels aident � effectuer cette
construction. Dans cet article, nous proposons de voir le probl�me
de construction d'horaires d'une mani�re plus �largie,
c'est-�-dire, consid�rer la production d'horaires comme un
processus plus large qui va de la cueillette et saisie de donn�es
jusqu'� la livraison d'un horaire, incluant la construction.
Ainsi, la production d'un horaire peut �tre automatis�e en
utilisant les bases de donn�es, un serveur Web, et un logiciel de
construction d'horaires.


!!
 Dans cet article, nous pr�sentons un processus de production
d'horaires. Nous avons analys� le flot d'information dans la
production d'un horaire, ce qui nous a permis de mieux comprendre
les caract�ristiques de ces syst�mes. Nous proposons
l'architecture d'un syst�me de production d'horaires de cours et
des examens. Cette architecture a �t� b�tie au tour de deux
qualit�s fondamentales dans la cr�ation d'un logiciel~:
l'ouverture et l'extensibilit�. Nous parlons de plusieurs
instances de l'architecture, et dans l'article nous expliquons les
raisons qui nous ont port�s � choisir une instance en
particuli�re. Les options technologiques pour l'instance choisie
ont �t� analys�es en d�tail, et cette analyse est pr�sent�e dans
l'article. Nous voulons offrir une guide de base pour montrer une
architecture d'un syst�me de production d'horaires, en incluant le
logiciel de construction d'horaires, les bases de donn�es, les
pages Web et les rapports. !!
