The timetable production in an academic institution is a periodic
task, where the frequency depends on the institution~: each year,
each six-months or each quarter. A part of this production is the
timetable construction for a set of courses. A timetable must
satisfy the hard constraints and try to respect a maximum of soft
constraints. It is recognized that the timetable construction is a
very complex problem, often software helps to carry out this
construction.

In this article, we propose to analyse the timetable construction
problem in a larger view, i.e., to consider the timetable
production as a process which goes from the gathering and data
input until the timetable delivery, including the construction.
Thus, the timetable production can be automated by using a
database, a Web site and a software for timetable construction. We
analyzed the flow of information necessary to the timetable
production, which enabled us to better understand the
characteristics of these systems. We propose a generic
architecture of a system of timetable production for courses and
exams. This architecture was built around two fundamental
qualities of software development~: the opening and extensibility.
