The timetable production for courses and exams in an academic
institution is a periodic activity. The frequency depends on the
institution~: yearly, every six-months, or quarterly. A timetable
must satisfy all the hard constraints and as many soft constraints
as possible. The timetable construction is one step in the
timetable production. Since it is a very complex activity, the use
of software helps carrying it out.


In this article, we propose to analyze the timetable construction
from a higher point of view. We are considering all the other
steps of the timetable production from data gathering and data
input to the timetable delivery, including the construction. We
analyzed the information flow necessary for the timetable
production. That let us to understand the characteristics of that
kind of systems. The timetable production can be automated by
using a database, a Web site, and timetable construction software.
We propose a generic architecture of a timetable production system
for courses and exams. We've developed that architecture around
two fundamental qualities of software development: openness and
extensibility.
