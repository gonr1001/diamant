The production of schedules in an institution of teaching is a
periodic task, on which the frequency depends on the institution~:
each year, each six-months or each quarter. A part of this
production is the construction of a schedule for a set of courses.
A schedule must satisfy hard constraints and try to respect a
maximum of soft constraints. It is recognized that the
construction of schedules is a very complex problem, often
software helps to carry out this construction.

In this article, we propose to see the problem of construction of
schedules in a larger way, i.e., to consider the production of
schedules as a process which goes from the gathering and data
capture until the delivery of a schedule, including the
construction. Thus, the production of a schedule can be automated
by using a database, a Web server and a software of construction
of schedules. We analyzed the flow of information necessary to the
production of a schedule, which enabled us to better understand
the characteristics of these systems. We propose a generic
architecture of a system of production of schedules of courses and
exams. This architecture was built around two fundamental
qualities of software development~: the opening and extensibility.
