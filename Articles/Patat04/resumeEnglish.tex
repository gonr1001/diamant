The courses and exams timetable production in an academic
institution is a periodic activiy, where the frequency depends on
the institution, each year, each six-months or each quarter. A
part of this production is the timetable construction. A timetable
must satisfy the hard constraints and try to respect a maximum of
soft constraints. The timetable construction is a very complex
problem. Often software helps to carry out this construction.

In this article, we propose to analyse the timetable construction
problem in a larger view, i.e., we consider the timetable
production as a process that goes from the gathering and data
input until the timetable delivery, including the construction.
Thus, we can automate the timetable production by using a
database, a Web site and software for timetable construction. We
analyzed the flow of information necessary to the timetable
production, which enabled us to understand the characteristics of
these systems. We propose a generic architecture of a system of
timetable production for courses and exams. We constructed this
architecture around two fundamental qualities of software
development~: opening and extensibility.
