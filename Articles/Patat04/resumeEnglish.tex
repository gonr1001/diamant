The production of a timetable for courses and exams in an academic 
institution is a periodic activity, the frequency of which depends on 
the institution: yearly, semi-annually, or quarterly. A timetable 
must satisfy all the hard constraints and as many soft constraints as 
possible. Timetable construction is one step in timetable production. 
Since it is a very complex activity, the use of software helps 
carrying it out. In this article, we propose to analyze timetable 
construction from a broader point of view. We will consider the full 
range of  timetable production activities from data gathering and 
data input to timetable delivery, including construction. We analyzed 
the information flow required for timetable production. It led us to 
understand the characteristics of such systems. Timetable production 
can be
automated by using databases, a Web site, and timetable construction 
software. We propose a generic architecture of a timetable production 
system for courses and exams. We�ve developed that architecture 
around two fundamental qualities of software development: openness 
and extensibility.

