
Dans la maintenance de logiciel, un des aspects qui demande plus
de temps est la recherche des erreurs ou \emph{bugs} qui se
pr�sentent chez l'utilisateur.  C'est ce fait qui nous a incit�s �
d�velopper un outil destin� � enregistrer le comportement d'un
utilisateur dans une application, le but �tant de reproduire
int�gralement et de fa�on automatique le probl�me avec l'aide de
l'enregistrement effectu�. Le principal avantage est de pouvoir
r�p�ter fid�lement le comportement de l'utilisateur afin de
reproduire la s�quence conduisant � un probl�me logiciel ou
\emph{bug}. Ainsi, il sera possible de fournir le logiciel �
l'utilisateur en lui indiquant qu'en cas d'erreur, le programmeur
dispose d'outils pour trouver le probl�me plus rapidement.



%Dans cet article, nous proposons de voir le probl�me de
%construction d'horaires d'une mani�re plus �largie, c'est-�-dire,
%de consid�rer la production d'horaires comme un processus  qui va
%de la cueillette et saisie de donn�es jusqu'� la livraison d'un
%horaire, incluant la construction. Ainsi, la production d'un
%horaire peut �tre automatis�e en utilisant les bases de donn�es,
%un site Web et un logiciel de construction d'horaires. Nous avons
%analys� le flot d'information n�cessaire � la production d'un
%horaire, ce qui nous a permis de mieux comprendre les
%caract�ristiques de ces syst�mes. Nous proposons l'architecture
%g�n�rique d'un syst�me de production d'horaires de cours et
%d'examens. Cette architecture a �t� b�tie autour de deux qualit�s
%fondamentales pour le d�veloppement d'un logiciel~: l'ouverture et
%l'extensibilit�.
