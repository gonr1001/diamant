La production d'horaires de cours et d'examens dans une
institution d'enseignement est une t�che p�riodique, dont la
fr�quence d�pend de l'institution~: chaque ann�e, chaque semestre
ou chaque trimestre. Une partie de cette production consiste �
construire un horaire. L'horaire doit satisfaire des contraintes
fortes et essayer de respecter un maximum de contraintes faibles.
Il est reconnu que la construction d'horaires est un probl�me tr�s
complexe. Des logiciels aident souvent � effectuer cette
construction.

Dans cet article, nous proposons de voir le probl�me de
construction d'horaires d'une mani�re plus �largie, c'est-�-dire,
de consid�rer la production d'horaires comme un processus  qui va
de la cueillette et saisie de donn�es jusqu'� la livraison d'un
horaire, incluant la construction. Ainsi, la production d'un
horaire peut �tre automatis�e en utilisant les bases de donn�es,
un site Web et un logiciel de construction d'horaires. Nous avons
analys� le flot d'information n�cessaire � la production d'un
horaire, ce qui nous a permis de mieux comprendre les
caract�ristiques de ces syst�mes. Nous proposons l'architecture
g�n�rique d'un syst�me de production d'horaires de cours et
d'examens. Cette architecture a �t� b�tie autour de deux qualit�s
fondamentales pour le d�veloppement d'un logiciel~: l'ouverture et
l'extensibilit�.
