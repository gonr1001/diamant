
Dans la maintenance de logiciel, un des aspects qui demande plus
de temps est la recherche des erreurs ou \emph{bugs} qui se
pr�sentent chez l'utilisateur.  C'est ce fait qui nous a incit�s �
d�velopper l'outil jReplay destin� � enregistrer le comportement
d'un utilisateur dans une application, le but �tant de reproduire
de fa�on automatique le probl�me avec l'aide de l'enregistrement
effectu�. Le principal avantage est de pouvoir r�p�ter le
comportement de l'utilisateur afin de reproduire la s�quence
conduisant � un probl�me logiciel ou \emph{bug}. Ainsi, il sera
possible de fournir le logiciel � l'utilisateur en lui indiquant
qu'en cas d'erreur, le programmeur dispose d'outils pour trouver
le probl�me plus rapidement.

jReplay a �t� con�u pour des applications interactives Java
construits avec la structure MVC (Mod�le Vue Contr�leur) qui est
la plus utilis� actuellement. Il a �t� construit en utilisant des
outils comme Log4j et JUnit. Log4j est un outil qui permet faire
la journalisation d'une application et d'activer ou de d�sactiver
certains messages en fonction des besoins. Pour autre parte, JUnit
permet faire de test unitaires qui attestent la validit� de
jReplay.
