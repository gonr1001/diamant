
Dans la maintenance de logiciel, un des aspects qui demande plus
de temps est la recherche des erreurs ou \emph{bugs} qui se
pr�sentent chez l'utilisateur.  C'est ce fait qui nous a incit�s �
d�velopper l'outil jReplay destin� � enregistrer le comportement
d'un utilisateur dans une application, le but �tant de reproduire
de fa�on automatique le probl�me avec l'aide de l'enregistrement
effectu�. Le principal avantage est de pouvoir r�p�ter le
comportement de l'utilisateur afin de reproduire la s�quence
conduisant � un probl�me logiciel ou \emph{bug}. Ainsi, il sera
possible de fournir le logiciel � l'utilisateur en lui indiquant
qu'en cas d'erreur, le programmeur dispose d'outils pour trouver
le probl�me plus rapidement.

jReplay a �t� con�u pour des applications interactives Java
construites avec le \emph{pattern} MVC (Mod�le Vue Contr�leur). Il
a �t� construit en utilisant l'outil Log4j qui permet de faire la
journalisation d'une application et d'activer ou de d�sactiver
certains messages en fonction des besoins. jReplay demande
l'insertion de certaines instructions dans des endroits tr�s
pr�cises du code pour g�n�rer le Journal de trace. Avec ce dernier
il est possible de re-ex�cuter l'application.


%La plus grande difficult� r�side dans l'insertion des instructions
%de journalisation. Vous devrez les placer avec soin pour que les
%renseignements soient suffisamment pr�cis sans verser dans le
%redondant ou l'inutile.
