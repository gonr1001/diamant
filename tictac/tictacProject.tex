% -*- TeX -*- -*- FR -*-
\documentclass[francais,letterpaper]{uds-article}

%-----------------------------------------------------------------------------
%----- Identification des packages n�cessaires
%-----------------------------------------------------------------------------

\usepackage{babel}
\usepackage{color}
\usepackage[latin1]{inputenc}
%\usepackage{udstitle,dfd}
%\newcommand{\diamant}{Diamant}
\setlength{\oddsidemargin}{0.25in}
\setlength{\evensidemargin}{0.25in}
\setlength{\textwidth}{6.0in}
%\setlength{\parskip}{0.2in}
\newcounter{auxcounter}
\renewcommand{\baselinestretch}{1.5}
\setlength{\parskip}{1.5ex plus0.5ex minus0ex}

\newcommand{\ints}{\renewcommand{\baselinestretch}{1.0}\small \normalsize}
\newcommand{\intm}{\renewcommand{\baselinestretch}{1.5}\small \normalsize}
\newcommand{\intd}{\renewcommand{\baselinestretch}{2.0}\small \normalsize}

\newcommand{\bi}{\begin{itemize}}
\newcommand{\ei}{\end{itemize}}
\newcommand{\be}{\begin{enumerate}}
\newcommand{\ee}{\end{enumerate}}
\newcommand{\bd}{\begin{description}}
\newcommand{\ed}{\end{description}}



\newcommand{\bv}{\verb}

\newcommand{\bve}{\verb*}

\newcommand{\brun}{\noindent $\triangleright$}
\newcommand{\erun}{$\triangleleft$}

\newcommand{\ang}{\textsf}
\newcommand{\key}{\textsf}
\newcommand{\ita}{\textit}
\newcommand{\bld}{\textbf}
\newcommand{\dos}{\textsc}
\newcommand{\pro}{\texttt}

\newcommand{\diamant}{DIAMANT}
\newcommand{\dia}{DIAMANT 1.0}
\newcommand{\dx}{DIAMANT 1.5}
\newcommand{\saphir}{SAPHIR}
\newcommand{\sig}{SIG}
\newcommand{\tict}{TicTac}
\newcommand{\dbug}{DiamantBug}

%-----------------------------------------------------------------------------
%----- Page Titre
%-----------------------------------------------------------------------------

\Titre{DX\\
Planification \tict{} et \dbug\\
\dx{}}
\Logo{Images/logoDX.eps}
\Auteurs{Ruben Gonzalez-Rubio, \\
Domingo Palao, \\
et Alexander Jaramillo }
 \Date{\today}

%-----------------------------------------------------------------------------
%----- Identification des fichiers des pages pr�liminaires et bibliographique
%-----------------------------------------------------------------------------

\FichierResume{}
\FichierRemerciements{}
\FichierGlossaire{} % \FichierLexique est �quivalent
\FichiersBibliographie{udsplain}{Inputs/bibDiamant,Inputs/bib2}

%-----------------------------------------------------------------------------
%----- Le document
%-----------------------------------------------------------------------------

\includeonly{UserManualInputs/intro,UserManualInputs/menus,UserManualInputs/horaire}

\begin{document}
\begin{articleDX}

%\chapter{Introduction}

\dx{} est un logiciel servant � la construction d'horaires de
cours et d'examens. Le manuel de l'utilisateur pr�sente~:

\begin{itemize}
    %\item Une description des menus, des bo�tes de dialogue et des fen�tres;
     \item Un exemple d'utilisation de \dx{} pour la pr�paration d'un horaire de cours (voir section \ref{prepacours}), consistant � placer dans une grille horaire les activit�s comportant plusieurs natures, plusieurs groupes et plusieurs blocs.
    \item Un exemple d'utilisation de \dx{} pour la pr�paration d'un horaire d'examens (voir section \ref{prepaexam}), consistant � placer dans une grille horaire les activit�s comportant une seule nature, un seul groupe et un seul bloc. �tant donn� que les informations utilis�es � l'universit� de Sherbrooke pour construire un horaire d'examen comportent des groupes, plusieurs natures, ainsi que plusieurs blocs, la construction d'un horaire d'examen ne conserve que la nature 1 d'une activit�, fusionne tous les groupes d'une nature en un seul groupe et ne conserve que le bloc 1 du groupe.
   
    %\item Une description du processus de la construction d'horaires.

%\item Une description des menus;

%\item Une description des bo�tes de dialogue;

%\item Une description des fen�tres;

%\item Guide d'installation.
\end{itemize}

Il s'agit de la version de \diamant{} telle que disponible en
mars 2004.

%\chapter{Construction d'un horaire avec \dx{}}
\section{Introduction}\ \label{horaire}
Construire un horaire avec \dx{} suppose que l'utilisateur a d�j�
 sur son ordinateur les fichiers n�cessaires \corrkad{les fichiers n�cessaires sur son ordinateur ; changez l'ordre} (ceux du STI et
  ceux de la Facult�)~: \verb!cours.sig! (fichier de cours), \verb!choixet.sig!
(fichier d'�tudiants), \verb!disprof.sig! (fichier d'enseignants) et
\verb!locaux.txt! (fichier de locaux). Ces fichiers peuvent porter
n'importe quel nom, mais pour notre exemple, les noms sont bien
explicites. Il est conseill� de regrouper toutes les informations
concernant les horaires d'un trimestre dans un r�pertoire (par
exemple \verb!hiver04! ou \verb!h2004!). Ainsi, les fichiers
d'importation et les fichiers produits pendant la construction d'un
horaire seront ensemble. La construction d'un horaire (cours ou
examen) peut varier d'une facult� � une autre, car certaines
facult�s pour faire leurs horaires utilisent un seul site
\corrolivier{Suggestion : "certaines facult�s utilisent un seul site
pour faire leurs horaires"}, tandis que d'autres en utilisent
plusieurs (construction multi-site).

La construction d'un nouvel horaire avec \dx{} comporte cinq phases:
\begin{itemize}
    \item Phase 1: elle concerne le chargement de donn�es dans le
    logiciel; \corrolivier{Je pense que �a serait l�g�rement mieux d'�crire "Phase 1 : Chargement des donn�es dans le logiciel ;" plut�t que de r�p�ter "elle concerne" � chaque fois (style � r�p�ter pour les autres items)}
    \item Phase 2: elle concerne la modification et l'�puration des donn�es;
    \item Phase 3: elle concerne la construction � proprement
    parler; \corrolivier{En suivant le style de la phase 1 : "Phase 3 : Construction de l'horaire ;"}
    \item Phase 4: elle concerne l'utilisation d'outils permettant de raffiner l'horaire (construction manuelle de
    l'horaire);
    \item Phase 5: elle concerne l'exportation de l'horaire.
\end{itemize}

En principe un horaire peut �tre fait en suivant les phases 1, 3 et
5. Mais, il est fort possible qu'il soit n�cessaire
\corrolivier{Trop lourd, surtout que cette expression est r�p�t�e
dans la phrase suivante. Suggestion : "Mais il peut �tre n�cessaire
de"} de modifier et d'�purer les donn�es, phase 2 \corrolivier{Il
est pr�f�rable de mettre "phase 2" entre parenth�ses ici}. Il est
aussi possible qu'il soit n�cessaire de construire un horaire
manuellement. La phase 4 d�crit les op�rations n�cessaires, ainsi
que les les op�rations d'ouverture et de sauvegarde d'un horaire.
\corrolivier{Suggestion pour les deux derni�res phrases : Tout
d'abord supprimer le point avant la phrase "Il est aussi possible"
et le remplacer par "ou bien de construire un horaire manuellement
(phase 4, notamment pour les op�rations d'ouverture et de sauvegarde
d'un horaire).}

Nous vous pr�senterons dans ce chapitre toutes ces phases
appliqu�es � la construction d'un horaire de cours et d'un horaire d'examen.

\section{Construction d'un horaire de cours}\label{prepacours}

\subsection{Phase 1: chargement et initialisation des donn�es}\label{finprepa}

\begin{enumerate}
    \item Lancer \dx{}.
    \item Aller au menu \textbf{\emph{Fichier}} \Rar \textbf{\emph{Nouvel
    horaire}}\index{Fichier \Rar Nouvel horaire}
    \Rar \textbf{\emph{Horaire cycle}}\footnote{Le symbole \Rar est utilis� pour dire
    qu'il faut chercher un niveau hi�rarchique plus bas un sous-menu. \corrolivier{Un peu compliqu�. Je pense que "[le symb�le] est utilis� pour indiquer un sous-menu" serait suffisant. A voir avec les utilisateurs.}}
    , une bo�te de dialogue comme celle
de la Figure \ref{selectgrillecyc} doit appara�tre
\corrolivier{Suggestion de remplacement : "afin de choisir le
fichier"} et elle vous permettra de choisir le fichier (fichier avec
extension \emph{.xml}) contenant la d�finition de votre grille
horaire.

    \begin{figure}[h]
    % Requires \usepackage{graphicx}
    \begin{center}
        \includegraphics[width=2.5in]{UserManualInputs/images/selectXMLfile.eps}
        \caption{Selection de la grille horaire de cours}\label{selectgrillecyc}
    \end{center}
    \end{figure}

    \item En cliquant sur le bouton \emph{\textbf{Grille horaire cycle}}, la grille horaire s�lectionn�e est charg�e et pr�sent�e � l'�cran (voir figure \ref{grillecyc}).
\begin{figure}[h]
  % Requires \usepackage{graphicx}
  \begin{center}
    \includegraphics[width=4.5in]{UserManualInputs/images/grille.eps}
    \caption{Grille horaire de cours}\label{grillecyc}
  \end{center}
\end{figure}

    \item Aller au menu \textbf{\emph{Fichier}} \Rar \textbf{\emph{D�finir fichiers � importer}}. Une bo�te de dialogue comme celle de la Figure \ref{defautoimport}  doit appara�tre. Rep�rer l'endroit o� chacun des fichiers est localis�, puis cliquer sur le bouton \textbf{\emph{OK}}. Une nouvelle fen�tre se pr�sentera afin de vous permettre d'enregistrer la configuration des fichiers que vous venez de faire.

Il est recommand� d'enregistrer cette configuration en utilisant un nom de fichier unique et repr�sentatif.

Exemple: choisissez le nom de fichier \verb!E02cours! pour �t� 2002 horaire de cours et le fichier cr�� sera \verb!E02cours.dim!. L'extension \verb!.dim! est rajout�e automatiquement.

\begin{figure}[h]
  % Requires \usepackage{graphicx}
  \begin{center}
    \includegraphics[width=2.5in]{UserManualInputs/images/autoimport.eps}
    \caption{D�finition des fichiers d'importation}\label{defautoimport}
  \end{center}
\end{figure}

    \item Aller au menu \textbf{\emph{Fichier}} \Rar \textbf{\emph{Importer automatiquement}}.
    Une boite de dialogue appara�tra et vous permettra de choisir le fichier \verb!.dim! de �configuration de
    fichiers� \corrolivier{Dans le fichier dvi, ce type de guillement est mal reproduit. A v�rifier.}
    pr�c�demment cr�� � partir de la fonction  \textbf{\emph{D�finir fichiers � importer}}
    (dans l'exemple pr�c�dent il s'agit du fichier \verb!E02cours.dim!).
    Cliquer � pr�sent sur le bouton \textbf{\emph{Importation de fichiers}},
    toutes vos donn�es (cours, �tudiants, enseignants et locaux) seront charg�es dans le logiciel et
    pr�tes � �tre modifi�es.

\item Lancer \textbf{\emph{Optimisation}} \Rar \textbf{\emph{Affectation initiale}} avant de passer � la phase 2. Elle ex�cutera les op�rations suivantes:

\begin{enumerate}
    \item Place de fa�on al�atoire dans les groupes d'activit� \corrolivier{S'il y a plusieurs activit�s dans un groupe, il faut ajouter un "s" � "activit�"}, les �tudiants non pr�-affect�s � des groupes, tout en �quilibrant les
    groupes. \corrolivier{Trop de r�p�tition du mot "groupe". A all�ger.}
    \item Place les �v�nements pr�-affect�s (plac�s ou fig�s) dans la grille horaire.
    \item Calcule les conflits g�n�r�s.
\end{enumerate}

Cette affectation initiale permet donc d'initialiser le logiciel
de fa�on � pouvoir faire des modifications sur les donn�es et
observer imm�diatement les r�percussions sur l'horaire.

\item Aller au menu \textbf{\emph{Rapports}} \Rar \textbf{\emph{Rapports...}} afin de v�rifier le rapport d'importation et pouvoir y d�celer les erreurs se trouvant dans les fichiers d'activit�s, d'�tudiants, d'enseignants et de locaux.

La pr�paration de l'horaire �tant achev�e, nous pouvons � pr�sent passer � la phase de modification et d'�puration de donn�es (phase 2).

\end{enumerate}

\subsection{Phase 2: modification et �purations des donn�es}

Le but de cette �tape est de permettre de construire l'horaire uniquement � partir de donn�es propres. Cette modification et/ou �puration peut se faire sur les activit�s, les �v�nements, les groupes d'�tudiants, les enseignants ou les locaux.

\begin{itemize}
    \item Modifier une activit� en cliquant sur le menu \textbf{\emph{Affectation}} \Rar \textbf{\emph{Activit�s}} pour voir appara�tre la \emph{liste des activit�s} (voir figure \ref{listactc}) ou alors cliquer sur le menu \textbf{\emph{Affectation}} \Rar \textbf{\emph{�v�nements}} pour voir appara�tre la \emph{liste des �v�nements} (voir figure \ref{listeeventc}).\\

� partir de la fen�tre \emph{liste des activit�s}, vous pouvez s�lectionner une ou plusieurs activit�s et les faire passer de la colonne \emph{inclue(s)} � \emph{non inclue(s)} (non inclue(s) veut dire que les activit�s ne seront pas utilis�es dans la construction de l'horaire). \\

� partir de la fen�tre \emph{liste des �v�nements}, vous pouvez
s�lectionner un ou plusieurs �v�nements et les faire passer de la
colonne \emph{fig�s} (\emph{fig�s} veut dire que les �v�nements sont
plac�s et fig�s dans la grille horaire) \corrolivier{Rappel du
probl�me que j'avais d�j� �voqu� lors d'une r�union : Je pense qu'il
faudrait renommer cette colonne "Fig� et plac�s" au lieu de
simplement "Fig�s" car il y a une ambiguit� avec le fonctionnement
du bouton "Figer" de la bo�te de dialogue d'�v�nement disponible en
double-cliquant sur un des �v�nements de la liste. Si cette bo�te de
dialogue ne doit pas �tre utilis�e ici pour �diter les �v�nements,
il faudrait alors en supprimer l'acc�s.} � \emph{plac�s}
(\emph{plac�s} veut dire que les �v�nements sont plac�s dans la
grille horaire) ou de la colonne \emph{plac�s} � \emph{non plac�s}
(\emph{non plac�s} veut dire que les �v�nements ne sont pas encore
plac�s dans la grille horaire).

    \begin{figure}[h]
      % Requires \usepackage{graphicx}
      \begin{center}
        \includegraphics[width=2.5in]{UserManualInputs/images/listeact.eps}
        \caption{Liste des activit�s}\label{listactc}
      \end{center}
    \end{figure}

    \begin{figure}[h]
      % Requires \usepackage{graphicx}
      \begin{center}
        \includegraphics[width=2.5in]{UserManualInputs/images/listeevent.eps}
        \caption{Liste des �v�nements}\label{listeeventc}
      \end{center}
    \end{figure}

� partir de l'une ou l'autre des fen�tres (figure \ref{listactc}
ou figure \ref{listeeventc}), double-cliquer sur l'activit� ou
l'�v�nement � modifier pour faire appara�tre le dialogue
d'\emph{affectation d'�v�nements} (voir figure \ref{eventc}). Il
vous est possible de modifier, � partir de cette fen�tre
d'\emph{affectation d'�v�nements}, le jour et l'heure de d�but de
l'�v�nement, l'enseignant, le local, de placer ou de figer
l'activit�. Pour la modification de la dur�e d'un �v�nement, voir la section \ref{modification}.

    \begin{figure}[h]
      % Requires \usepackage{graphicx}
      \begin{center}
        \includegraphics[width=2.5in]{UserManualInputs/images/event.eps}
        \caption{Affectation d'�v�nement \corrolivier{Capture � mettre � jour}}\label{eventc}
      \end{center}
    \end{figure}

Cliquer sur \textbf{\emph{Appliquer}} pour valider les changements et cliquer sur \textbf{\emph{Fermer}} pour fermer la fen�tre sans appliquer les changements.\\

    \item Ajouter ou supprimer un groupe ou un �v�nement en cliquant sur le menu
    \textbf{\emph{Modification}} \Rar{} \textbf{\emph{Activit�s}} (voir section \ref{modification} pour plus de d�tails).

\item Modifier les groupes d'�tudiants en cliquant sur le menu \textbf{\emph{Affectation}} \Rar \textbf{\emph{Groupes}} pour voir appara�tre la fen�tre de la figure \ref{groupec}. S�lectionner dans la colonne de droite (�tudiants assign�s) le groupe (Groupe 01, Groupe 10, ...) que l'on souhaite modifier. S�lectionner ensuite un ou plusieurs �tudiants que l'on souhaite d�placer. Utilisez enfin les fl�ches pour d�placer les �tudiants de la gauche vers la droite ou vice-versa.

    \begin{figure}[h]
      % Requires \usepackage{graphicx}
      \begin{center}
        \includegraphics[width=2.5in]{UserManualInputs/images/groupe.eps}
        \caption{Affectation des groupes}\label{groupec}
      \end{center}
    \end{figure}

Choisir dans la liste de s�lection \textbf{\emph{Par matricule/Par programme/Par nom}} de la zone \emph{trier}, pour choisir le type de tri que l'on souhaite faire.

Cliquer sur \textbf{\emph{Appliquer}} pour valider les changements et cliquer sur \textbf{\emph{Fermer}} pour fermer la fen�tre sans appliquer les changements.\\

\item \textcolor[rgb]{0.98,0.00,0.00}{Modifier la disponibilit� d'un enseignant en cliquant sur le menu \textbf{\emph{Affectation}} \Rar  \textbf{\emph{Enseignants}} pour voir appara�tre la fen�tre de la figure \ref{enseignantc}. En s�lectionnant une zone correspondant au jour et � l'heure que vous souhaitez modifier. Une zone s�lectionn�e appara�t en fonc� et indique que l'enseignant y est disponible.}

    \begin{figure}[h]
      % Requires \usepackage{graphicx}
      \begin{center}
        \includegraphics[width=2.5in]{UserManualInputs/images/enseignant.eps}
        \caption{Disponibilit� des enseignants}\label{enseignantc}
      \end{center}
    \end{figure}

Cliquer sur \textbf{\emph{Appliquer}} pour accepter et effectuer les changements et cliquer sur \textbf{\emph{Fermer}} pour fermer la fen�tre sans appliquer les changements.\\

    \item Modifier la disponibilit� d'un local en cliquant sur le menu
    \textbf{\emph{Affectation}} \Rar  \textbf{\emph{Locaux}}
    pour voir appara�tre la fen�tre de la figure  \ref{localc}.
    En s�lectionnant une zone correspondant au jour et � l'heure que vous souhaitez modifier. Une zone s�lectionn�e appara�t en fonc� et indique que le local y est
    disponible. \corrolivier{Les deux derni�res phrases sont � remplacer. Suggestion : "Une s�lection de la zone correspondant au jour et � l'heure que vous souhaitez modifier permet de passer de l'�tat disponible (couleur fonc�e) � l'�tat indisponible (couleur claire)."}

    \begin{figure}[h]
      % Requires \usepackage{graphicx}
      \begin{center}
        \includegraphics[width=2.5in]{UserManualInputs/images/local.eps}
        \caption{Disponibilit� des locaux}\label{localc}
      \end{center}
    \end{figure}

Cliquer sur \textbf{\emph{Appliquer}} pour accepter et
effectuer les changements et cliquer sur \textbf{\emph{Fermer}} pour fermer la fen�tre sans appliquer les changements.\\

 \end{itemize}

\subsection{Phase 3: construction de l'horaire}

Cette �tape consiste � lancer le menu \textbf{\emph{Optimisation}} \Rar \textbf{\emph{Construire l'horaire}}, afin de laisser le logiciel placer automatiquement dans la grille horaire les �v�nements (ceux qui n'ont pas encore �t� plac�s dans la grille horaire) respectant toutes les contraintes sp�cifi�es et ne cr�ant aucun nouveau conflit (conflits d'enseignants, conflits d'�tudiants et conflits de locaux).

La modification des contraintes � respecter par \dx{} lors de la construction automatique de l'horaire se fait en lan�ant \textbf{\emph{Pr�f�rences}} \Rar \textbf{\emph{Options Conflits}} pour faire appara�tre la fen�tre d'option de conflits (voir figure \ref{confc}).

Cette fen�tre permet de modifier plusieurs param�tres, � savoir:

\begin{figure}[h]
      % Requires \usepackage{graphicx}
      \begin{center}
        \includegraphics[width=3.5in]{UserManualInputs/images/optionconflict.eps}
        \caption{Contraintes � respecter}\label{confc}
      \end{center}
    \end{figure}

\begin{itemize}
    \item \emph{Max Conflits �tu entre 2 Eve}: ce param�tre permet \corrolivier{Suggestion : remplacer simplement par "Permet"} de fixer le nombre maximal admissible de conflits d'�tudiants entre deux �v�nements, dans une p�riode.
    \item \emph{Max Conflits Ens entre 2 Eve}: ce param�tre permet de fixer le nombre maximal admissible de conflits d'enseignants entre deux �v�nements, dans une p�riode.
    \item \emph{Max Conflits Loc entre 2 Eve}: ce param�tre permet de fixer le nombre maximal admissible de conflits de locaux entre deux �v�nements, dans une p�riode.
    \item \emph{Placer dans p�riodes (0=normale, 1=normale et basse, 2=normale, basse et nulle)}: \corrolivier{le d�tail pour la valeur 2 n'est pas pr�sente dans la bo�te de dialogue.} ce param�tre permet de sp�cifier les types de p�riodes (normale, basse, nulle) de la grille horaire dans lesquels un �v�nement peut �tre plac�.
    \item \emph{Nombre Max de cours dans une p�riode}: ce param�tre permet de sp�cifier le nombre maximal d'�v�nements pouvant �tre plac�s dans une m�me p�riode de la grille horaire.
    \item \emph{P�riode d'�cart}: ce param�tre permet de sp�cifier l'�cart (en nombre de p�riodes) � respecter entre deux �v�nements potentiellement en conflit (conflits d'�tudiants).
\end{itemize}

Il est possible de modifier les contraintes et lancer ensuite la
construction de l'horaire afin que le logiciel  utilise les
nouvelles contraintes pour placer les �v�nements qu'il n'a pas pu
pr�c�demment placer \corrolivier{Beaucoup trop lourd. Suggestion :
"Il est possible de modifier les contraintes puis de relancer la
construction de l'horaire afin de voir le r�sultat des derni�res
modifications, notamment pour v�rifier si elles permettent de placer
plus d'�venements."}; tout ceci de fa�on it�rative jusqu'� obtention
d'un r�sultat paraissant satisfaisant aux yeux de l'utilisateur.
\corrolivier{Cette phrase peut �tre supprim�e afin d'all�ger le
texte.}

Il est �galement possible de revenir � l'�tat initial de l'horaire
(l'�tat de l'horaire juste apr�s l'affectation initiale
\corrolivier{Suggestion de remplacement : "c'est-�-dire apr�s
l'affectation initiale"}). Pour cela, il suffit d'ouvrir la fen�tre
\textbf{\emph{Liste des �v�nements}} � partir du menu
\textbf{\emph{Affectation}} \Rar \textbf{\emph{�v�nements}}, ensuite
de s�lectionner tous les �v�nements de la colonne des
\textbf{\emph{Plac�s}}, puis de les faire passer de la colonne des
\textbf{\emph{Plac�s}} � celle des \textbf{\emph{Non plac�s}} en
cliquant sur la fl�che allant de la gauche vers la droite et enfin
cliquer sur le bouton \textbf{\emph{Appliquer}} pour effectuer les
changements.

Si, malgr� les it�rations, l'horaire construit n'est pas
satisfaisant, il existe un certain nombre d'outils permettant de
raffiner manuellement l'horaire.


\subsection{Phase 4: raffinement de l'horaire}
Le raffinement de l'horaire propose � travers certains
\corrolivier{Remplacer "� travers certains" par "plusieurs"} outils
(affectation manuelle, formation de groupes, modification, rapports,
ouverture et sauvegarde d'un horaire), les voies et moyens
d'am�lioration de l'horaire construit. \corrolivier{� remplacer par
"permettant d'am�liorer l'horaire construit."}

\subsubsection{\textcolor[rgb]{0.98,0.00,0.00}{Importation selective}}
L'importation selective permet d'ins�rer de nouvelles donn�es dans
un horaire d�j� en cours de construction. On peut ins�rer de
nouveaux enseignants, de nouveaux locaux, de nouveaux �tudiants ou
de nouvelles activit�s.

La m�thode utilis�e pour l'insertion de nouvelles donn�es est la
suivante:

\begin{description}
    \item[Insertion de nouveaux enseignants: ] l'insertion de
    nouveaux enseignants rajoute � la liste des enseignants
    existant d�j� dans \diamant{}, les nouveaux enseignants
    contenu dans le fichier � ins�rer, sans modifier ceux se trouvant d�j� dans \diamant{}.
    \item[Insertion de nouveaux locaux: ] l'insertion de
    nouveaux locaux rajoute � la liste des locaux
    existant d�j� dans \diamant{}, les nouveaux locaux
    contenu dans le fichier � ins�rer, sans modifier ceux se trouvant d�j� dans \diamant{}.
    \item[Insertion de nouvelles activit�s:] l'insertion de
    nouvelles activit�s rajoute � la liste des activit�s
    existant d�j� dans \diamant{}, les nouvelles activit�s
    contenu dans le fichier � ins�rer, sans modifier ceux se trouvant d�j� dans \diamant{}.
    \item[Insertion de nouveaux �tudiants: ] l'insertion de
    nouveaux �tudiants rajoute � la liste des �tudiants
    existant d�j� dans \diamant{}, les nouveaux �tudiants
    contenu dans le fichier � ins�rer, supprime de cette liste les �tudiants n'existant plus dans le
    fichier et rajoute aux �tudiants leurs nouveaux choix de
    cours.
\end{description}

\subsubsection{\textcolor[rgb]{0.98,0.00,0.00}{Affectation manuelle}}

L'affectation manuelle permet � l'utilisateur de placer
individuellement les �v�nements dans la grille horaire. Elle se fait
� partir de la fen�tre d'\textbf{\emph{Affection manuelle}} (voir
figure \ref{manc}) obtenue � partir du menu
\textbf{\emph{Affectation}} \Rar \textbf{\emph{Affectation
manuelle}} .

    \begin{figure}[h]
      % Requires \usepackage{graphicx}
      \begin{center}
        \includegraphics[width=3.5in]{UserManualInputs/images/manualaffect.eps}
        \caption{Affectation manuelle}\label{manc}
      \end{center}
    \end{figure}

La fen�tre d'affectation manuelle offre deux possibilit�s d'utilisation:

\begin{enumerate}
    \item \textbf{Modification d'un �v�nement :} s�lectionner un �v�nement et cliquer sur le bouton \textbf{\emph{Modifier}} pour faire appara�tre la bo�te de dialogue d'\emph{affectation d'�v�nements} (voir figure \ref{evente}). Il vous est possible de modifier, � partir de cette fen�tre d'\emph{affectation d'�v�nements} \corrolivier{Inutile, on sait de quelle fen�tre il s'agit puisqu'elle est cit�e juste avant.}, le jour et l'heure de d�but de l'�v�nement, l'enseignant, le local, de placer ou figer l'�v�nement.
    \item \textbf{Recherche de potentiels conflits :} pour rechercher
    les potentiels conflits g�n�r�s par un �v�nement, double-cliquer
    sur l'�v�nement en question \corrolivier{Suggestion : "double-cliquer sur ce dernier"} pour voir appara�tre une nouvelle
    grille horaire (voir figure \ref{man1c}) vous informant sur les conflits que peut g�n�rer l'�v�nement \corrolivier{Suggestion : "qu'il peut g�n�rer"} dans chaque p�riode de la grille horaire. \\

Une repr�sentation par les couleurs \corrolivier{Suggestion : "par
couleurs"}, de la grille horaire d'affectation manuelle,
\corrolivier{A mettre plut�t entre parenth�ses au lieu d'utiliser
des virgules, voire � supprimer car ce n'est pas utile ici} a �t�
adopt�e afin de vous permettre, d'un seul coup d'oeil, de rep�rer
\corrolivier{A remplacer plut�t par "afin de permettre de r�p�rer
imm�diatement"}les p�riodes de conflits potentiels (couleur rose),
les p�riodes dans lesquelles l'�v�nement peut �tre plac� sans
g�n�rer de conflits (couleur par d�faut de la p�riode: gris), et la
ou les p�riodes dans lesquelles l'�v�nement est plac� (s'il �tait
pr�c�demment plac� dans la grille - couleur verte). Les p�riodes
poss�dant d�j� des conflits avant le lancement de l'am�lioration
manuelle sont aussi repr�sent�es par la couleur rose; il faut dans
ce cas, apr�s le lancement de l'affectation manuelle, regarder pour
chaque �v�nement de la p�riode les conflits potentiels avec le
nouvel �v�nement.

\subsubsection{Exemple: interpr�tation d'une p�riode de couleur rose}

Les donn�es ci-dessous repr�sentent une p�riode de la grille horaire d'affectation manuelle de l'�v�nement GEI220.1.01.1.
\begin{verbatim}
ADM111.1.02.1 0 1 0
GEI100.1.01.1 5 0 0
GEI200.1.01.1 5 0 1
     13 2 1
\end{verbatim}

Elles sont interpr�t�es de la mani�re suivante:
\begin{itemize}
    \item La ligne \verb!ADM111.1.02.1! \textcolor[rgb]{1.00,0.00,0.00}{0 1 0} repr�sente les conflits potentiels  entre les �v�nements \verb!ADM111.1.02.1! et \verb!GEI220.1.01.1!. Elle indique 0 conflit d'�tudiants, 1 conflit d'enseignants et 0 conflit de locaux.
    \item La ligne \verb!GEI100.1.01.1! \textcolor[rgb]{1.00,0.00,0.00}{5 0 0} repr�sente les conflits potentiels  entre les �v�nements \verb!GEI100.1.01.1! et \verb!GEI220.1.01.1!. Elle indique 5 conflits d'�tudiants, 0 conflit d'enseignants et 0 conflit de locaux.
    \item La ligne \verb!GEI200.1.01.1! \textcolor[rgb]{1.00,0.00,0.00}{5 0 1} repr�sente les conflits potentiels  entre les �v�nements \verb!GEI200.1.01.1! et \verb!GEI220.1.01.1!. Elle indique 5 conflits d'�tudiants, 0 conflit d'enseignants et 1 conflit de locaux.
    \item La ligne \textcolor[rgb]{1.00,0.00,0.00}{13 2 1} repr�sente la somme des conflits potentiels  de la p�riode (10 conflicts d'�tudiants, 1 conflit d'enseignants, 1 conflit de locaux) aux conflits r�els existant dans la p�riode (3 conflits d'�tudiants, 1 conflit d'enseignants et 0 conflit de locaux).
\end{itemize}

    \begin{figure}[h]
      % Requires \usepackage{graphicx}
      \begin{center}
        \includegraphics[width=4.0in]{UserManualInputs/images/manualaffectres.eps}
        \caption{Grille horaire d'affectation manuelle}\label{man1c}
      \end{center}
    \end{figure}

\end{enumerate}

\subsubsection{Modification des activit�s}\label{modification}

La fonctionnalit� de modification des activit�s permet d'ajouter
ou de supprimer une nature � une activit�, d'ajouter ou de
supprimer un groupe � une nature d'activit�, d'ajouter ou de
supprimer un bloc � un groupe et enfin de modifier la dur�e d'un
�v�nement ou d'affecter un �v�nement comme pr�sent� dans la figure
\ref{eventc}. Il faut cependant noter qu'elle ne permet pas
d'ajouter une activit�.

La modification d'une activit� se fait en cascade \corrkad{de fa�on
hi�rarchique}, c'est � dire que pour faire des modifications sur un
niveau, il faudrait \corrolivier{Remplacer plut�t par
"faut"}traverser les niveaux pr�c�dents (par exemple, pour faire des
modifications sur un groupe, il faut d'abord s�lectionner l'activit�
concern�e, puis la nature et pour finir le groupe concern�).

\begin{enumerate}

 \item \subsubsection{Ajout/suppression d'une nature � une activit�}

Ajouter et/ou supprimer une nature � une activit� se fait en lan�ant
le menu \textbf{\emph{Modification}} \Rar \textbf{\emph{Activit�}},
pour faire appara�tre la fen�tre d'activit�s de la figure
\ref{modifactc}. \corrolivier{Suggestion : "[...] se fait � partir
de la fen�tre d'activit�s (figure \ref{modifactc}, menu
\textbf{\emph{Modification}} \Rar \textbf{\emph{Activit�}}"}

    \begin{figure}[h]
      % Requires \usepackage{graphicx}
      \begin{center}
        \includegraphics[width=2.0in]{UserManualInputs/images/modifact.eps}
        \caption{Modification d'activit�s}\label{modifactc}
      \end{center}
    \end{figure}

Double-cliquer sur une activit� pour faire appara�tre la fen�tre de
natures de la figure \ref{modiftypec}. Toutes les natures de
l'activit� s�lectionn�e apparaissent dans cette fen�tre.

    \begin{figure}[h]
      % Requires \usepackage{graphicx}
      \begin{center}
        \includegraphics[width=2.0in]{UserManualInputs/images/modiftype.eps}
        \caption{Modification de natures}\label{modiftypec}
      \end{center}
    \end{figure}

\begin{itemize}
    \item On ne peut ajouter qu'une nature 2 � une activit� poss�dant pr�alablement une nature 1. Pour le faire \corrolivier{"Pour cela"} cliquer simplement sur le bouton \textbf{\emph{Ajouter}}.
    \item On ne peut supprimer qu'une nature 2 � une activit�. Pour le faire \corrolivier{idem} cliquer simplement sur le bouton \textbf{\emph{Supprimer}}.
\end{itemize}

    \item \subsubsection{Ajout/suppression d'un groupe � une nature d'activit�}

� partir de la fen�tre de modification de natures (voir figure
\ref{modiftypec}), double-cliquer sur une nature d'activit� pour
faire appara�tre la fen�tre de groupes de la figure
\ref{modifgroupec}. Tous les groupes de la nature s�lectionn�e
apparaissent dans cette fen�tre.

 \begin{figure}[h]
      % Requires \usepackage{graphicx}
      \begin{center}
        \includegraphics[width=2.0in]{UserManualInputs/images/modifgroupe.eps}
        \caption{Modification de groupes}\label{modifgroupec}
      \end{center}
    \end{figure}

\begin{itemize}
    \item Pour ajouter un groupe cliquer sur le bouton \textbf{\emph{Ajouter}} pour voir appara�tre la fen�tre de la figure \ref{modifgroupeselecc}
\begin{figure}[h]
      % Requires \usepackage{graphicx}
      \begin{center}
        \includegraphics[width=2.0in]{UserManualInputs/images/modifgroupeselect.eps}
        \caption{Choix du groupe � ajouter}\label{modifgroupeselecc}
      \end{center}
    \end{figure}
    Entrer le nom du groupe � ajouter et cliquer sur \textbf{\emph{Ok}} (Exemple: pour cr�er un groupe 3, entrer dans la zone \emph{Num�ro de groupe} la valeur 03).
    \item Pour supprimer un groupe, s�lectionner le groupe � supprimer et cliquer simplement sur le bouton \textbf{\emph{Supprimer}}.
\end{itemize}

    \item \subsubsection{Ajout/suppression d'un bloc � un groupe d'une nature d'activit�}

� partir de la fen�tre de modification de groupe (voir figure
\ref{modifgroupec}), double-cliquer sur un groupe pour faire
appara�tre la fen�tre de modification de blocs de la figure
\ref{modifblocc}. Tous les blocs du groupe s�lectionn� apparaissent
dans cette fen�tre.

    \begin{figure}[h]
        % Requires \usepackage{graphicx}
        \begin{center}
            \includegraphics[width=2.0in]{UserManualInputs/images/modifunit.eps}
            \caption{Choix du groupe � ajouter}\label{modifblocc}
        \end{center}
    \end{figure}

\begin{itemize}
    \item On ne peut \corrolivier{C'est "on ne peut" ou "on peut" ?} ajouter un bloc � un groupe en cliquant simplement sur le bouton \textbf{\emph{Ajouter}}. Le nouveau bloc est cr�� � la suite du dernier bloc de la liste.
    \item Pour supprimer un bloc, cliquer simplement sur le bouton \textbf{\emph{Supprimer}} et le dernier bloc de la liste sera supprim�. Si la liste ne poss�de qu'un seul bloc, ce dernier ne pourra pas �tre supprim�.
\end{itemize}

� partir de la fen�tre de modification de blocs, double-cliquer sur
un bloc pour faire appara�tre la fen�tre d'\emph{affectation
d'�v�nements} de la figure \ref{eventc}. Il vous est possible de
modifier, � partir de cette fen�tre d'\emph{affectation
d'�v�nements} \corrolivier{Inutile de r�p�ter le nom de la fen�tre
ici}, non seulement le jour et l'heure de d�but de l'�v�nement,
l'enseignant, le local, de placer ou de figer l'activit�, mais
�galement la dur�e d'un �v�nement \corrolivier{La phrase n'est pas
tr�s claire. Suggestion : "Il est possible, � partir de cette
fen�tre, de modifier le jour et l'heure de d�but de l'�v�nement ou
sa dur�e, mais aussi de modifier l'enseignant et le local, ou encore
de placer ou de figer l'activit�."}.

\end{enumerate}

\subsubsection{\textcolor[rgb]{0.98,0.00,0.00}{Formation de groupes}}

\textcolor[rgb]{0.98,0.00,0.00}{La formation de groupe commun�ment
appel� \og brassage d'�tudiants dans les groupes \fg{} peut se faire
de trois fa�ons distinctes:}

\begin{itemize}
    \item En cliquant sur \textbf{\emph{Optimisation}} \Rar \textbf{\emph{Formation de groupes}} \Rar \textbf{\emph{Balanc�}},
    le logiciel brasse les �tudiants des activit�s d�j� plac�es dans la grille horaire, afin de les placer dans les groupes en minimisant les conflits et en ne tol�rant qu'un �cart d'un �tudiant entre les diff�rents groupes d'une m�me activit� \corrolivier{Note par rapport au nom des menus : je vois mal le rapport entre le nom de ces trois sous-menus et plus particuli�rement celui-ci celui-ci, et leur fonction}.
    \item En cliquant sur \textbf{\emph{Optimisation}} \Rar \textbf{\emph{Formation de groupes}}
    \Rar \textbf{\emph{Interm�diaire}}, le logiciel brasse les �tudiants des activit�s d�j�
    plac�es dans la grille horaire, afin de les placer dans les groupes en minimisant une fois de plus les conflits,
    mais en tol�rant cette fois-ci un �cart de 15
    (valeur � param�trer si besoin se fait sentir)
    �tudiants au plus entre les diff�rents groupes d'une m�me activit�.
    \item En cliquant sur \textbf{\emph{Optimisation}} \Rar \textbf{\emph{Formation de groupes}}
    \Rar \textbf{\emph{Optimis�}}, le logiciel brasse les �tudiants des activit�s d�j� plac�es dans la grille horaire, de afin de les placer dans les groupes en minimisant les conflits et en autorisant tous les �carts entre les diff�rents groupes d'une m�me activit�.
\end{itemize}

\subsubsection{Rapports}

La fonctionnalit� rapport \corrolivier{Suggestion : mettre en
�vidence "rapport" dans le texte} permet de visualiser le rapport
complet des �v�nements plac�s dans la grille horaire, de visualiser
le rapport de conflits g�n�r�s par les �v�nements plac�s dans la
grille horaire et de visualiser les erreurs survenues lors de
l'importation de donn�es \corrolivier{Il faut �viter la r�p�tition
du verbe "visualiser". Suggestion : Effectuer une �num�ration
(items)}.

\begin{enumerate}

    \item \subsubsection{Rapport complet}

Le rapport complet permet de visualiser les �v�nements plac�s dans la grille horaire (voir figure \ref{rapfullc}).

\begin{figure}[h]
% Requires \usepackage{graphicx}
    \begin{center}
        \includegraphics[width=4.0in]{UserManualInputs/images/rap_full.eps}
        \caption{Rapport complet}\label{rapfullc}
    \end{center}
\end{figure}

Les informations affich�es sont tri�es en partant de la colonne de
gauche vers la droite. Les abr�viations utilis�es au niveau de
l'en-t�te du rapport se d�finissent comme suit:

\begin{itemize}
    \item \emph{JNum}: num�ro de la journ�e (jour 1, jour 2, ...),
    \item \emph{Jour}: nom de la journ�e (lundi, mardi, ...),
    \item \emph{Acti}: nom de l'activit�,
    \item \emph{Nat}: nature de l'activit�,
    \item \emph{Grp}: groupe de l'activit�,
    \item \emph{Blc}: bloc de l'activit�,
    \item \emph{Nomb �}: nombre d'�tudiants,
    \item \emph{Dur�e}: dur�e de l'�v�nement,
    \item \emph{D�but}: heure de d�but de l'�v�nement,
    \item \emph{Fin}: heure de fin de l'�v�nement,
    \item \emph{Enseignant}: nom de l'enseignant,
    \item \emph{Local}: nom ou num�ro du local.
\end{itemize}

Les informations � afficher pour chaque �v�nement peuvent �tre d�finies � partir de la fen�tre d'\textbf{\emph{Option de rapport}}, en cliquant sur le bouton \textbf{\emph{Options}} � partir de la fen�tre de \textbf{\emph{Rapport complet}} (voir figure \ref{rapfullc}).

\subsubsection{Option de rapport complet}

Cette fen�tre d'\textbf{\emph{Option de rapport}}
\corrolivier{Normalement, "option" doit �tre au pluriel ici. A
modifier aussi dans le menu de l'application} (voir figure
\ref{rapfulloptionsc}) permet de choisir les informations � afficher
dans le rapport complet. L'ordre d'apparition des champs dans la
colonne \emph{Champs choisis} (du haut vers le bas) d�finit l'ordre
dans lequel ces champs appara�tront dans le rapport complet (de la
gauche vers la droite).

Utiliser les boutons avec les fl�ches pour d�placer un champ d'une colonne vers une autre et les boutons avec les signes $+$ ou $-$ pour d�placer un champ du haut vers le bas et vice-versa.

\begin{figure}[h]
% Requires \usepackage{graphicx}
    \begin{center}
        \includegraphics[width=2.0in]{UserManualInputs/images/rap_full_options.eps}
        \caption{Options du rapport complet}\label{rapfulloptionsc}
    \end{center}
\end{figure}

    \item \subsubsection{Rapport de conflits}

Le rapport de conflits permet de visualiser les conflits g�n�r�s par les �v�nements plac�s dans la grille horaire (voir figure \ref{rapconfc}).

\begin{figure}[h]
% Requires \usepackage{graphicx}
    \begin{center}
        \includegraphics[width=4.0in]{UserManualInputs/images/rap_conflicts.eps}
        \caption{Rapport de conflits}\label{rapconfc}
    \end{center}
\end{figure}

Les informations affich�es sont tri�es en partant de la colonne de
gauche vers la droite. Les abr�viations utilis�es au niveau de
l'ent�te du rapport se d�finissent comme suit:

\begin{itemize}
    \item \emph{JNum}: num�ro de la journ�e (jour 1, jour 2, ...),
    \item \emph{Jour}: nom de la journ�e (lundi, mardi, ...),
    \item \emph{P�rJour}: p�riode de la journ�e (AM, PM),
    \item \emph{�v�nement 1 }: premier �v�nement en conflit,
    \item \emph{�v�nement 2}: �v�nement avec lequel l'�v�nement 1 est en conflit ou description du conflit si l'�v�nement 1 pr�sente un conflit interne (disponibilit� de l'enseignant, disponibilit� du local et/ou capacit� du local inf�rieur au nombre d'�tudiants).
    \item \emph{D�but}: heure de d�but de l'�v�nement,
    \item \emph{Type Conf}: type de conflits (�tudiant, enseignant ou local),
    \item \emph{Numb C}: nombre de conflits,
    \item \emph{Conflits}: d�tail du conflit (liste des �tudiants en conflit, nom ou num�ro du local ou de l'enseignant en conflit).
\end{itemize}

Les informations � afficher pour chaque conflit peuvent �tre d�finies � partir de la fen�tre d'\textbf{\emph{Option de conflits}}, en cliquant sur le bouton \textbf{\emph{Options}} � partir de la fen�tre de \textbf{\emph{Rapport de conflits}} (voir figure \ref{rapconfc}).

\subsubsection{Option de rapport de conflits}

Cette fen�tre d'\textbf{\emph{Option de conflits}} (voir figure \ref{rapconfoptionsc}) permet de choisir les information � afficher dans le rapport de conflits. L'ordre d'apparition des champs dans la colonne \emph{Champs choisis} (du haut vers le bas), d�finit l'ordre dans lequel ces champs appara�tront dans le rapport complet (de la gauche vers la droite).

Utiliser les boutons avec les fl�ches pour d�placer un champ d'une colonne vers une autre et les boutons avec les signes $+$ ou $-$ pour d�placer un champ du haut vers le bas et vice-versa.

\begin{figure}[h]
% Requires \usepackage{graphicx}
    \begin{center}
        \includegraphics[width=2.0in]{UserManualInputs/images/rap_conflicts_options.eps}
        \caption{Options du rapport de conflits}\label{rapconfoptionsc}
    \end{center}
\end{figure}

\end{enumerate}

\subsubsection{Sauvegarde et ouverture d'un horaire}

\subsubsection{Sauvegarde d'un horaire}

Pour sauvegarder un horaire, lancer le menu \textbf{\emph{Fichier}} \Rar \textbf{\emph{Enregistrer ...}}
ou \textbf{\emph{Fichier}} \Rar \textbf{\emph{Enregistrer sous ...}} pour voir appara�tre la fen�tre de la figure \ref{savettc}.
 Parcourir le disque pour trouver o� enregistrer l'horaire. Entrer le nom � attribuer � l'horaire (avec ou sans extension \verb!.dia!) et cliquer sur le bouton \textbf{\emph{Enregistrer}}.

\begin{figure}[h]
% Requires \usepackage{graphicx}
    \begin{center}
        \includegraphics[width=2.0in]{UserManualInputs/images/savett.eps}
        \caption{Sauvegarde d'un horaire}\label{savettc}
    \end{center}
\end{figure}

\subsubsection{Ouverture d'un horaire}

Pour ouvrir un horaire, lancer le menu \textbf{\emph{Fichier}} \Rar \textbf{\emph{Ouvrir horaire ...}} pour voir appara�tre la fen�tre de la figure \ref{openttc}. Parcourir le disque pour trouver l'emplacement du fichier d'horaire (avec l'extension \verb!.dia!) � ouvrir. S�lectionner le fichier et cliquer sur le bouton \textbf{\emph{Ouvrir}} ou double-cliquer simplement sur le fichier pour l'ouvrir.

L'op�ration d'ouverture d'un horaire ex�cutera automatiquement l'affectation initiale une fois les donn�es charg�es.

\begin{figure}[h]
% Requires \usepackage{graphicx}
    \begin{center}
        \includegraphics[width=2.0in]{UserManualInputs/images/opentt.eps}
        \caption{Ouverture d'un horaire}\label{openttc}
    \end{center}
\end{figure}

\subsection{Phase 5: Exportation}

Pour exporter les donn�es, aller au menu \textbf{\emph{Fichier}},
puis s�lectionner \textbf{\emph{Exporter}}; deux fichiers seront
cr��s dans le r�pertoire racine de votre document avec les noms
suivants:
\begin{itemize}
    \item \verb!S814.HORAIRE! (il contient l'horaire);
    \item \verb!S813.ASSGRO! (il contient les �tudiants, leurs choix d'activit�s et leurs groupes).
\end{itemize}

\section{Construction d'un horaire d'examen}\label{prepaexam}

\subsection{Phase 1: pr�paration de l'horaire}\label{finprepae}

\begin{enumerate}
    \item Lancer \dx{}.
    \item Aller au menu \textbf{\emph{Fichier}} \Rar \textbf{\emph{Nouvel horaire}} \Rar \textbf{\emph{Horaire Examen}}.
    Une bo�te de dialogue comme celle
de la Figure \ref{selectgrilleexam} doit appara�tre et pour
permettre de choisir le fichier (fichier avec extension \emph{.xml})
contenant la d�finition de la grille horaire.

    \begin{figure}[h]
    % Requires \usepackage{graphicx}
    \begin{center}
        \includegraphics[width=2.5in]{UserManualInputs/images/selectXMLfileExam.eps}
        \caption{S�lection de la grille horaire d'examen}\label{selectgrilleexam}
    \end{center}
    \end{figure}

    \item En cliquant sur le bouton \emph{\textbf{Grille horaire examen}}, la grille horaire s�lectionn�e est charg�e et pr�sent�e � l'�cran (voir figure \ref{grilleexam}).
\begin{figure}[h]
  % Requires \usepackage{graphicx}
  \begin{center}
    \includegraphics[width=4.5in]{UserManualInputs/images/grilleexam.eps}
    \caption{Grille horaire d'examen}\label{grilleexam}
  \end{center}
\end{figure}

    \item Aller au menu \textbf{\emph{Fichier}} \Rar \textbf{\emph{D�finir fichiers � importer}}. Une bo�te de dialogue comme celle de la Figure \ref{defautoimportex}  doit appara�tre. Rep�rer l'endroit o� chacun des fichiers est localis�, puis cliquer sur le bouton \textbf{\emph{OK}}. Une nouvelle fen�tre se pr�sentera afin de vous permettre d'enregistrer la configuration des fichiers que vous venez de faire.

\begin{figure}[h]
  % Requires \usepackage{graphicx}
  \begin{center}
    \includegraphics[width=2.5in]{UserManualInputs/images/autoimport.eps}
    \caption{D�finition des fichiers d'importation}\label{defautoimportex}
  \end{center}
\end{figure}

Il est recommand� d'enregistrer cette configuration en utilisant un nom de fichier unique et repr�sentatif.

Exemple: choisissez le nom de fichier \verb!E02exam! pour �t� 2002
horaire d'examen et le fichier cr�� sera \verb!E02exam.dim!.
L'extension \verb!.dim! est rajout�e automatiquement.

    \item Aller au menu \textbf{\emph{Fichier}} \Rar \textbf{\emph{Importer automatiquement}}.
    Une boite de dialogue appara�tra et vous permettra de choisir le fichier \verb!.dim! de �configuration de fichiers�
    pr�c�demment cr�� � partir de la fonction  \textbf{\emph{D�finir fichiers � importer}}
    (dans l'exemple pr�c�dent il s'agit du fichier \verb!H04exam.dim!).
    Cliquer � pr�sent sur le bouton \textbf{\emph{Importation de
    fichiers}}.
    Toutes vos donn�es (cours, �tudiants, enseignants et locaux) seront charg�es dans le logiciel et
    pr�tes � �tre modifi�es.

\item Lancer \textbf{\emph{Optimisation}} \Rar \textbf{\emph{Affectation initiale}} avant de passer � la phase 2. Elle ex�cutera les op�rations suivantes:

\begin{enumerate}
    \item Suppression des activit�s de nature 2.
    Seules les activit�s de nature 1 seront conserv�es pour l'horaire.
    Tous les �v�nements seront initialis�s afin qu'ils soient non plac�s et non fig�s.
    \item Suppression des groupes aux activit�s en poss�dant plusieurs \corrolivier{Je ne comprend pas, la phrase doit �tre reformul�e}, un seul groupe sera conserv� pour l'horaire (le premier groupe ou \emph{groupe A}).
    \item Suppression des �v�nements aux activit�s en poss�dant plusieurs \corrolivier{Idem}, un seul �v�nement sera conserv� pour l'horaire.
    \item Modification de la disponibilit� des enseignants afin de les rendre tous disponibles.
    \item Modification de la disponibilit� des locaux afin de les rendre tous disponibles.
\end{enumerate}

Exemple: Pour construire l'horaire \corrolivier{Suggestion de
remplacement : "Construction d'un horaire"} d'examen d'une activit�
(GEI200) poss�dant 2 natures (GEI200.1 et GEI200.2), chaque nature
poss�dant 2 groupes (GEI200.1.A, GEI200.1.B et GEI200.2.A,
GEI200.2.B), chaque groupe poss�dant 2 �v�nements (GEI200.1.A.1,
GEI200.1.A.2, GEI200.1.B.1, GEI200.1.B.2 et GEI200.2.A.1,
GEI200.2.A.2, GEI200.2.B.1, GEI200.2.B.2). La suppression de nature
2, de groupes et d'�v�nements par l'affectation initiale permettrait
d'obtenir \corrolivier{Suggestion : "La suppression des activit�s de
nature 2, ainsi que des groupes et des �v�nements qui en d�pendent "
(� v�rifier car je n'ai pas compris ce qui est expliqu� plus haut)
"permet d'obtenir"}, pour l'activit� GEI200, un seul �v�nement
devant servir � l'horaire d'examen, en l'occurrence GEI200.1.A.1.

Cette affectation initiale permet donc d'�purer les donn�es et
d'initialiser le logiciel afin de pouvoir faire
\corrolivier{Suggestion : "afin de pouvoir effectuer"} des
modifications sur les donn�es et d'observer imm�diatement les
r�percussions sur l'horaire.

La pr�paration de l'horaire �tant achev�e, nous pouvons � pr�sent
passer � la phase de modification et d'�puration de donn�es (phase
2). Il est cependant n�cessaire de noter que cette phase peut avoir
lieu avant ou apr�s la phase de construction � proprement parler
(phase 3), mais nous recommandons de la faire avant la phase 3
\corrolivier{"avant celle-ci} afin de travailler une bonne fois pour
toute sur des donn�es propres (�pur�es) \corrolivier{Il faut �viter
"une bonne fois pour toutes" ici. Suggestion : "Afin de travailler
directement sur des donn�es propres"}.
\end{enumerate}

\subsection{Phase 2: modification et �purations des donn�es}

Le but de cette �tape est de permettre de construire l'horaire uniquement � partir de donn�es propres. Cette modification et/ou �puration peut se faire sur les activit�s, les �v�nements, les groupes d'�tudiants, les enseignants ou les locaux.

\begin{itemize}

    \item Modifier une activit� en cliquant sur le menu \textbf{\emph{Affectation}} \Rar \textbf{\emph{Activit�s}} pour voir appara�tre la \emph{liste des activit�s} (voir figure \ref{listacte}) ou alors cliquer sur le menu \textbf{\emph{Affectation}} \Rar \textbf{\emph{�v�nements}} pour voir appara�tre la \emph{liste des �v�nements} (voir figure \ref{listeevente}).\\


   � partir de la fen�tre \emph{liste des activit�s}, vous pouvez s�lectionner une ou plusieurs activit�s et les faire passer de la colonne \emph{inclue(s)} � \emph{non inclue(s)} (non inclue(s) veut dire que les activit�s ne seront pas utilis�es dans la construction de l'horaire). \\

� partir de la fen�tre \emph{liste des �v�nements}, vous pouvez s�lectionner un ou plusieurs �v�nements et les faire passer de la colonne \emph{fig�s} (\emph{fig�s} veut dire que les �v�nements sont plac�s et fig�s dans la grille horaire) � \emph{plac�s} (\emph{plac�s} veut dire que les �v�nements sont plac�s dans la grille horaire) ou de la colonne \emph{plac�s} � \emph{non plac�s} (\emph{non plac�s} veut dire que les �v�nements ne sont pas encore plac�es dans la grille horaire).

    \begin{figure}[h]
      % Requires \usepackage{graphicx}
      \begin{center}
        \includegraphics[width=2.5in]{UserManualInputs/images/listeact.eps}
        \caption{Liste des activit�s}\label{listacte}
      \end{center}
    \end{figure}

    \begin{figure}[h]
      % Requires \usepackage{graphicx}
      \begin{center}
        \includegraphics[width=2.5in]{UserManualInputs/images/listeevent.eps}
        \caption{Liste des �v�nements}\label{listeevente}
      \end{center}
    \end{figure}


� partir de l'une ou l'autre des fen�tres, double-cliquer sur
l'activit� ou l'�v�nement � modifier pour faire appara�tre la bo�te
de dialogue d'\emph{affectation d'�v�nements} (voir figure
\ref{evente}). Il vous est possible de modifier, � partir de cette
fen�tre d'\emph{affectation d'�v�nements} \corrolivier{Inutile de
r�p�ter le nom de la fen�tre ici}, le jour et l'heure de d�but de
l'�v�nement, l'enseignant, le local, de placer ou figer l'�v�nement.
Pour la modification de la dur�e d'un �v�nement, voir la section
\ref{modificatione}.

    \begin{figure}[h]
      % Requires \usepackage{graphicx}
      \begin{center}
        \includegraphics[width=2.5in]{UserManualInputs/images/event.eps}
        \caption{Affectation d'�v�nement}\label{evente}
      \end{center}
    \end{figure}

Cliquer sur \textbf{\emph{Appliquer}} pour valider les changements et cliquer sur \textbf{\emph{Fermer}} pour fermer la fen�tre sans appliquer les changements.\\

\item \textcolor[rgb]{0.98,0.00,0.00}{Modifier la disponibilit� d'un enseignant en cliquant sur le menu \textbf{\emph{Affectation}} \Rar \textbf{\emph{Enseignants}} pour voir appara�tre la fen�tre de la figure \ref{enseignante}. En s�lectionnant une zone correspondant au jour et � l'heure que vous souhaitez modifier. Une zone s�lectionn�e appara�t en fonc� et indique que l'enseignant y est disponible.}

    \begin{figure}[h]
      % Requires \usepackage{graphicx}
      \begin{center}
        \includegraphics[width=2.5in]{UserManualInputs/images/enseignant.eps}
        \caption{Disponibilit� des enseignants}\label{enseignante}
      \end{center}
    \end{figure}

Cliquer sur \textbf{\emph{Appliquer}} pour accepter et effectuer les changements et cliquer sur \textbf{\emph{Fermer}} pour fermer la fen�tre sans appliquer les changements.\\

    \item Modifier la disponibilit� d'un local en cliquant
    sur le menu \textbf{\emph{Affectation}} \Rar \textbf{\emph{Locaux}}
    pour voir appara�tre la fen�tre de la figure  \ref{locale}.
    En s�lectionnant une zone correspondant au jour et � l'heure que vous souhaitez modifier. Une zone s�lectionn�e appara�t en fonc� et indique que le local y est
    disponible. \corrolivier{M�me remarque ici que pour la disponibilit� des locaux d�taill�e dans une section pr�c�dente.}

    \begin{figure}[h]
      % Requires \usepackage{graphicx}
      \begin{center}
        \includegraphics[width=2.5in]{UserManualInputs/images/local.eps}
        \caption{Disponibilit� des locaux}\label{locale}
      \end{center}
    \end{figure}

Cliquer sur \textbf{\emph{Appliquer}} pour accepter et effectuer les changements et cliquer sur \textbf{\emph{Fermer}} pour fermer la fen�tre sans appliquer les changements.\\

 \end{itemize}

\subsection{Phase 3: construction de l'horaire}

Cette �tape consiste � lancer le menu \textbf{\emph{Optimisation}} \Rar \textbf{\emph{Construire l'horaire}}, afin de laisser le logiciel placer automatiquement dans la grille horaire les �v�nements (ceux qui n'ont pas encore �t� plac�s dans la grille horaire) respectant toutes les contraintes sp�cifi�es et ne cr�ant aucun nouveau conflit (conflits d'enseignants, conflits d'�tudiants et conflits de locaux).

La modification des contraintes � respecter par \dx{} lors de la construction automatique de l'horaire se fait en lan�ant \textbf{\emph{Pr�f�rences}} \Rar \textbf{\emph{Options Conflits}} pour faire appara�tre la fen�tre d'option de conflits (voir figure \ref{confc}). Cette fen�tre permet de modifier plusieurs param�tres, � savoir:

\begin{figure}[h]
      % Requires \usepackage{graphicx}
      \begin{center}
        \includegraphics[width=3.5in]{UserManualInputs/images/optionconflict.eps}
        \caption{Contraintes � respecter}\label{conf}
      \end{center}
    \end{figure}

\begin{itemize}
    \item \emph{Max Conflits �tu entre 2 Eve}: ce param�tre permet \corrolivier{M�mes remarques que pour le paragraphe identique situ� dans une section plus haut} de fixer le nombre maximal admissible de conflits d'�tudiants entre deux �v�nements, dans une p�riode.
    \item \emph{Max Conflits Ens entre 2 Eve}: ce param�tre permet de fixer le nombre maximal admissible de conflits d'enseignants entre deux �v�nements, dans une p�riode.
    \item \emph{Max Conflits Loc entre 2 Eve}: ce param�tre permet de fixer le nombre maximal admissible de conflits de locaux entre deux �v�nements, dans une p�riode.
    \item \emph{Placer dans p�riodes (0=normale, 1=normale et basse, 2=normale, basse et nulle)}: ce param�tre permet de sp�cifier les types de p�riodes (normale, basse, nulle) de la grille horaire dans lesquels un �v�nement peut �tre plac�.
    \item \emph{Nombre Max de cours dans une p�riode}: ce param�tre permet de sp�cifier le nombre maximal d'�v�nements pouvant �tre plac�s dans une m�me p�riode de la grille horaire.
    \item \emph{P�riode d'�cart}: ce param�tre permet de sp�cifier l'�cart (en nombre de p�riodes) � respecter entre deux �v�nements potentiellement en conflit (conflits d'�tudiants).
\end{itemize}

Il est possible de modifier les contraintes et lancer ensuite la
construction de l'horaire afin que le logiciel  utilise les
nouvelles contraintes pour placer les �v�nements qu'il n'a pas pu
pr�c�demment placer; tout ceci de fa�on it�rative jusqu'� obtention
d'un r�sultat paraissant satisfaisant aux yeux de l'utilisateur.
\corrolivier{M�mes remarques que pour le paragraphe identique situ�
dans une section plus haut}

Il est �galement possible de revenir � l'�tat initial de l'horaire
(l'�tat de l'horaire juste apr�s l'affectation initiale). Pour
cela, il suffit  d'ouvrir la fen�tre \textbf{\emph{Liste des
�v�nements}} � partir du sous-menu \textbf{\emph{�v�nements}} du
menu \textbf{\emph{Affectation}}, ensuite de s�lectionner tous les
�v�nements de la colonnes des \textbf{\emph{Plac�s}}, puis de les
faire passer de la colonne des \textbf{\emph{Plac�s}} � celle des
\textbf{\emph{Non plac�s}} en cliquant sur la fl�che allant de la
gauche vers la droite et enfin cliquer sur le bouton Appliquer
pour effectuer les changements.

Si, malgr� les it�rations, l'horaire construit n'est pas
satisfaisant, il existe un certain nombre d'outils permettant de
raffiner manuellement l'horaire.

\subsection{Phase 4: raffinement de l'horaire}
\corrolivier{M�mes remarques que pour la section identique situ�
dans plus haut. Il faut v�rifier aussi que ces sections en double on
r�ellement une utilit� et s'il n'est pas possible d'organiser le
manuel de fa�on � avoir un seul exemplaire de chaque explication de
fonctionnalit�} Le raffinement de l'horaire propose � travers
certains outils (affectation manuelle, formation de groupes,
modification, rapports, ouverture et sauvegarde d'un horaire), les
voies et moyens d'am�lioration de l'horaire construit.


\subsubsection{\textcolor[rgb]{0.98,0.00,0.00}{Importation selective}}


\subsubsection{\textcolor[rgb]{0.98,0.00,0.00}{Affectation manuelle}}


\subsubsection{Affectation manuelle}

L'affectation manuelle permet � l'utilisateur de placer
individuellement les �v�nements dans la grille horaire. Elle se fait
� partir de la fen�tre d'\textbf{\emph{Affection manuelle}} (voir
figure \ref{man}) obtenue � partir du menu
\textbf{\emph{Affectation}} \Rar \textbf{\emph{Affectation
manuelle}}.

    \begin{figure}[h]
      % Requires \usepackage{graphicx}
      \begin{center}
        \includegraphics[width=3.5in]{UserManualInputs/images/manualaffect.eps}
        \caption{Affectation manuelle}\label{man}
      \end{center}
    \end{figure}

La fen�tre d'affectation manuelle offre deux possibilit�s d'utilisation:

\begin{enumerate}
    \item \textbf{Modification d'un �v�nement:} s�lectionner un �v�nement et cliquer sur le bouton \textbf{\emph{Modifier}} pour faire appara�tre le dialogue d'\emph{affectation d'�v�nements} (voir figure \ref{evente}). Il vous est possible de modifier, � partir de cette fen�tre d'\emph{affectation d'�v�nements}, le jour et l'heure de d�but de l'�v�nement, l'enseignant, le local, de placer ou figer l'�v�nement.
    \item \textbf{Recherche de conflits potentiels:} pour rechercher
    les conflits potentiels  g�n�r�s par un �v�nement,
    double-cliquer sur l'�v�nement en question pour voir
    appara�tre une nouvelle grille horaire (voir figure \ref{man1})
    vous informant sur les conflits que peut g�n�rer l'�v�nement dans chaque p�riode de la grille horaire.

Une repr�sentation par les couleurs, de la grille horaire
d'affectation manuelle, a �t� adopt�e afin de vous permettre, d'un
seul coup d'oeil, de rep�rer les p�riodes de conflits potentiels
(couleur rose), les p�riodes dans lesquelles l'�v�nement peut �tre
plac� sans g�n�rer de conflits (couleur par d�faut de la p�riode),
et la ou les p�riodes dans lesquelles l'�v�nement est plac� (s'il
�tait pr�c�demment plac� dans la grille - couleur verte). Les
p�riodes poss�dant d�j� des conflits avant le lancement de
l'am�lioration manuelle sont aussi repr�sent�es par la couleur
rose; il faut dans ce cas, apr�s le lancement de l'affectation
manuelle, regarder pour chaque �v�nement de la p�riode les
conflits potentiels avec le nouvel �v�nement.

\subsubsection{Exemple: interpr�tation d'une p�riode de couleur rose}

Les donn�es ci-dessous repr�sentent une p�riode la grille horaire d'affectation manuelle de l'�v�nement GEI220.1.01.1.
\begin{verbatim}
ADM111.1.01.1 0 0 0
GEI100.1.01.1 5 0 0
GEI200.1.01.1 5 0 1
     13 0 1
\end{verbatim}

Elles sont interpr�t�es de la mani�re suivante:
\begin{itemize}
    \item La ligne \verb!ADM111.1.01.1! \textcolor[rgb]{1.00,0.00,0.00}{0 0 0} repr�sente les conflits potentiels entre les �v�nements \verb!ADM111.1.01.1! et \verb!GEI220.1.01.1!. Elle indique 0 conflit d'�tudiants, 0 conflit d'enseignants et 0 conflit de locaux.
    \item La ligne \verb!GEI100.1.01.1! \textcolor[rgb]{1.00,0.00,0.00}{5 0 0} repr�sente les conflits potentiels entre les �v�nements \verb!GEI100.1.01.1! et \verb!GEI220.1.01.1!. Elle indique 5 conflits d'�tudiants, 0 conflit d'enseignants et 0 conflit de locaux.
    \item La ligne \verb!GEI200.1.01.1! \textcolor[rgb]{1.00,0.00,0.00}{5 0 1} repr�sente les conflits potentiels entre les �v�nements \verb!GEI200.1.01.1! et \verb!GEI220.1.01.1!. Elle indique 5 conflits d'�tudiants, 0 conflit d'enseignants et 1 conflit de locaux.
    \item La ligne \textcolor[rgb]{1.00,0.00,0.00}{13 0 1} repr�sente la somme des conflits potentiels de la p�riode (10 conflicts d'�tudiants, 0 conflit d'enseignants, 1 conflit de locaux) aux conflits r�els existant dans la p�riode (3 conflits d'�tudiants, 0 conflit d'enseignants et 0 conflit de locaux).
\end{itemize}

    \begin{figure}[h]
      % Requires \usepackage{graphicx}
      \begin{center}
        \includegraphics[width=4.0in]{UserManualInputs/images/manualaffectres.eps}
        \caption{Grille horaire d'affectation manuelle}\label{man1}
      \end{center}
    \end{figure}

\end{enumerate}


\subsubsection{Modification des activit�s}\label{modificatione}

La fonctionnalit� de modification des activit�s permet d'ajouter
ou de supprimer une nature � une activit�, d'ajouter ou de
supprimer un groupe � une nature d'activit�, d'ajouter ou de
supprimer un bloc � un groupe et enfin modifier la dur�e d'un
�v�nement ou d'affecter un �v�nement comme pr�sent� dans la figure
\ref{evente}. Il faut cependant noter qu'elle ne permet pas
d'ajouter une activit�.

On pourrait par contre se poser la question � savoir pourquoi, pour
des horaires d'examen, les termes tels que activit�, nature, groupe
et blocs resurgissent � la place du terme examen par exemple. La
r�ponse r�side dans l'approche utilis�e; car, que ce soit pour les
cours ou pour les examens, la construction d'un horaire se fait �
travers des �v�nements. Pour les horaires de cours, un �v�nement
repr�sente un cours - quadruplet (activit�, nature, groupe, bloc) -
qui doit avoir lieu, tandis que pour les horaires d'examen, un
�v�nement repr�sente un examen qui doit avoir lieu. Dans un souci
d'homog�n�it�, nous avons donc d�duit qu'un examen d'une activit�
pourrait �tre vu comme �v�nement r�sultant d'une activit� ne
poss�dant qu'une seule nature (la nature 1), cette nature ne
poss�dant qu'un seul groupe et ce groupe ne poss�dant qu'un seul
bloc.

Cette approche offre donc la possibilit� de pouvoir attribuer � une
activit�, un examen de nature 2 (laboratoire) ou encore de diviser
un examen devant se tenir dans un local en deux groupes afin qu'ils
se d�roulent dans deux locaux distincts.
 La modification d'une activit� se fait en cascade \corrolivier{Voir la remarque de Kader sur le m�me terme, employ� plus haut.}, c'est � dire que pour faire des
modifications sur un niveau, il faudrait traverser les niveaux
pr�c�dents (par exemple, pour faire des modifications sur un groupe,
il faut d'abord s�lectionner l'activit� concern�e, ensuite
s�lectionner la nature et enfin s�lectionner le groupe concern�).
\begin{enumerate}
    \item \subsubsection{Ajout/suppression d'une nature � une activit�}

Ajouter et/ou supprimer une nature � une activit� se fait en lan�ant le menu \textbf{\emph{Modification}} \Rar \textbf{\emph{Activit�}}, pour faire appara�tre la fen�tre d'activit�s de la figure \ref{modifacte}.

    \begin{figure}[h]
      % Requires \usepackage{graphicx}
      \begin{center}
        \includegraphics[width=2.0in]{UserManualInputs/images/modifact.eps}
        \caption{Modification d'activit�s}\label{modifacte}
      \end{center}
    \end{figure}

Double cliquer sur une activit� pour faire appara�tre la fen�tre de natures de la figure \ref{modiftypee}. Toutes les natures de l'activit� s�lectionn�e apparaissent dans cette fen�tre.

    \begin{figure}[h]
      % Requires \usepackage{graphicx}
      \begin{center}
        \includegraphics[width=2.0in]{UserManualInputs/images/modiftype.eps}
        \caption{Modification de natures}\label{modiftypee}
      \end{center}
    \end{figure}

\begin{itemize}
    \item On ne peut ajouter qu'une nature 2 � une activit� poss�dant pr�alablement une nature 1. Pour le faire cliquer simplement sur le bouton \textbf{\emph{Ajouter}}.
    \item On ne peut supprimer qu'une nature 2 � une activit�. Pour le faire cliquer simplement sur le bouton \textbf{\emph{Supprimer}}.
\end{itemize}

    \item \subsubsection{Ajout/suppression d'un groupe � une nature d'activit�}

� partir de la fen�tre de modification de natures (voir figure \ref{modiftypee}), double cliquer sur une nature d'activit� pour faire appara�tre la fen�tre de groupes de la figure \ref{modifgroupee}. Tous les groupes de la nature s�lectionn�e apparaissent dans cette fen�tre.

 \begin{figure}[h]
      % Requires \usepackage{graphicx}
      \begin{center}
        \includegraphics[width=2.0in]{UserManualInputs/images/modifgroupe.eps}
        \caption{Modification de groupes}\label{modifgroupee}
      \end{center}
    \end{figure}

\begin{itemize}
    \item Pour ajouter un groupe cliquer sur le bouton \textbf{\emph{Ajouter}} pour voir appara�tre la fen�tre de la figure \ref{modifgroupeselece}
\begin{figure}[h]
      % Requires \usepackage{graphicx}
      \begin{center}
        \includegraphics[width=2.0in]{UserManualInputs/images/modifgroupeselect.eps}
        \caption{Choix du groupe � ajouter}\label{modifgroupeselece}
      \end{center}
    \end{figure}
    Entrer le nom du groupe � ajouter et cliquer sur \textbf{\emph{Ok}} (Exemple: pour cr�er un groupe 3, entrer dans la zone \emph{Num�ro de groupe} la valeur 03).
    \item Pour supprimer un groupe, s�lectionner le groupe � supprimer et cliquer simplement sur le bouton \textbf{\emph{Supprimer}}.
\end{itemize}

    \item \subsubsection{Ajout/suppression d'un bloc � un groupe d'une nature d'activit�}

� partir de la fen�tre de modification de groupe (voir figure \ref{modifgroupee}), double cliquer sur un groupe pour faire appara�tre la fen�tre de blocs de la figure \ref{modifbloce}. Tous les blocs du groupe s�lectionn� apparaissent dans cette fen�tre.

    \begin{figure}[h]
        % Requires \usepackage{graphicx}
        \begin{center}
            \includegraphics[width=2.0in]{UserManualInputs/images/modifunit.eps}
            \caption{Choix du groupe � ajouter}\label{modifbloce}
        \end{center}
    \end{figure}

\begin{itemize}
    \item On ne peut ajouter un bloc � un groupe en cliquant simplement sur le bouton \textbf{\emph{Ajouter}}. Le nouveau bloc est cr�� � la suite du dernier bloc de la liste.
    \item Pour supprimer un bloc, cliquer simplement sur le bouton \textbf{\emph{Supprimer}} et le dernier bloc de la liste sera supprim�. Si la liste ne poss�de qu'un seul bloc, ce dernier ne pourra pas �tre supprim�.
\end{itemize}

� partir de la fen�tre de modification de blocs, double cliquer sur un bloc pour faire appara�tre la fen�tre d'\emph{affectation d'�v�nements} de la figure \ref{evente}. Il vous est possible de modifier, � partir de cette fen�tre d'\emph{affectation d'�v�nements}, non seulement le jour et l'heure de d�but de l'�v�nement, l'enseignant, le local, de placer ou de figer l'activit�, mais �galement la dur�e d'un �v�nement.

\end{enumerate}

\subsubsection{Rapports}

La fonctionnalit� rapport permet de visualiser le rapport complet des �v�nements plac�s dans la grille horaire, de visualiser le rapport de conflits g�n�r�s par les �v�nements plac�s dans la grille horaire et de visualiser les erreurs survenues lors de l'importation de donn�es.

\begin{enumerate}

    \item \subsubsection{Rapport complet}

Le rapport complet permet de visualiser les �v�nements plac�s dans la grille horaire (voir figure \ref{rapfulle}).

\begin{figure}[h]
% Requires \usepackage{graphicx}
    \begin{center}
        \includegraphics[width=4.0in]{UserManualInputs/images/rap_full.eps}
        \caption{Rapport complet}\label{rapfulle}
    \end{center}
\end{figure}

Les informations affich�es sont tri�es en partant de la colonne de
gauche vers la droite. Les abr�viations utilis�es au niveau de
l'en-t�te du rapport se d�finissent comme suit:

\begin{itemize}
    \item \emph{JNum}: num�ro de la journ�e (jour 1, jour 2, ...),
    \item \emph{Jour}: nom de la journ�e (lundi, mardi, ...),
    \item \emph{Acti}: nom de l'activit�,
    \item \emph{Nat}: nature de l'activit�,
    \item \emph{Grp}: groupe de l'activit�,
    \item \emph{Blc}: bloc de l'activit�,
    \item \emph{Nomb �}: nombre d'�tudiants,
    \item \emph{Dur�e}: dur�e de l'�v�nement,
    \item \emph{D�but}: heure de d�but de l'�v�nement,
    \item \emph{Fin}: heure de fin de l'�v�nement,
    \item \emph{Enseignant}: nom de l'enseignant,
    \item \emph{Local}: nom ou num�ro du local.
\end{itemize}

Les informations � afficher pour chaque �v�nement peuvent �tre d�finies � partir de la fen�tre d'\textbf{\emph{Option de rapport}}, en cliquant sur le bouton \textbf{\emph{Options}} � partir de la fen�tre de \textbf{\emph{Rapport complet}} (voir figure \ref{rapfulle}).

\subsubsection{Option de rapport complet}

Cette fen�tre d'\textbf{\emph{Option de rapport}} (voir figure \ref{rapfulloptionse}) permet de choisir les information � afficher dans le rapport complet. L'ordre d'apparition des champs dans la colonne \emph{Champs choisis} (du haut vers le bas), d�finit l'ordre dans lequel ces champs appara�tront dans le rapport complet (de la gauche vers la droite).

Utiliser les boutons avec les fl�ches pour d�placer un champ d'une colonne vers une autre et les boutons avec les signes $+$ ou $-$ pour d�placer un champ du haut vers le bas et vice-versa.

\begin{figure}[h]
% Requires \usepackage{graphicx}
    \begin{center}
        \includegraphics[width=2.0in]{UserManualInputs/images/rap_full_options.eps}
        \caption{Options du rapport complet}\label{rapfulloptionse}
    \end{center}
\end{figure}


    \item \subsubsection{Rapport de conflits}

Le rapport de conflits permet de visualiser les conflits g�n�r�s par les �v�nements plac�s dans la grille horaire (voir figure \ref{rapconfe}).

\begin{figure}[h]
% Requires \usepackage{graphicx}
    \begin{center}
        \includegraphics[width=4.0in]{UserManualInputs/images/rap_conflicts.eps}
        \caption{Rapport de conflits}\label{rapconfe}
    \end{center}
\end{figure}

Les informations affich�es sont tri�es en partant de la colonne de
gauche vers la droite. Les abr�viations utilis�es au niveau de
l'en-t�te du rapport se d�finissent comme suit:

\begin{itemize}
    \item \emph{JNum}: num�ro de la journ�e (jour 1, jour 2, ...),
    \item \emph{Jour}: nom de la journ�e (lundi, mardi, ...),
    \item \emph{P�rJour}: p�riode de la journ�e (AM, PM),
    \item \emph{�v�nement 1 }: premier �v�nement en conflit,
    \item \emph{�v�nement 2}: �v�nement avec lequel l'�v�nement 1
    est en conflit ou description du conflit si l'�v�nement 1 p
    r�sente un conflit interne (disponibilit� de l'enseignant, disponibilit� du local et/ou capacit� du local inf�rieure au nombre d'�tudiants).
    \item \emph{D�but}: heure de d�but de l'�v�nement,
    \item \emph{Type Conf}: type de conflits (�tudiant, enseignant ou local),
    \item \emph{Numb C}: nombre de conflits,
    \item \emph{Conflits}: d�tail du conflit (liste des �tudiants en conflit, nom du local ou de l'enseignant en conflit).
\end{itemize}

Les informations � afficher pour chaque conflit peuvent �tre d�finies � partir de la fen�tre d'\textbf{\emph{Option de conflits}}, en cliquant sur le bouton \textbf{\emph{Options}} � partir de la fen�tre de \textbf{\emph{Rapport de conflits}} (voir figure \ref{rapconfe}).

\subsubsection{Option de rapport de conflits}

Cette fen�tre d'\textbf{\emph{Option de conflits}} (voir figure \ref{rapconfoptionse}) permet de choisir les information � afficher dans le rapport de conflits. L'ordre d'apparition des champs dans la colonne \emph{Champs choisis} (du haut vers le bas), d�finit l'ordre dans lequel ces champs appara�tront dans le rapport complet (de la gauche vers la droite).

Utiliser les boutons avec les fl�ches pour d�placer un champ d'une colonne vers une autre et les boutons avec les signes $+$ ou $-$ pour d�placer un champ du haut vers le bas et vice-versa.

\begin{figure}[h]
% Requires \usepackage{graphicx}
    \begin{center}
        \includegraphics[width=2.0in]{UserManualInputs/images/rap_conflicts_options.eps}
        \caption{Options du rapport de conflits}\label{rapconfoptionse}
    \end{center}
\end{figure}

\end{enumerate}
\subsubsection{Sauvegarde et ouverture d'un horaire}

\subsubsection{Sauvegarde d'un horaire}

Pour sauvegarder un horaire, lancer le menu
\textbf{\emph{Fichier}} \Rar \textbf{\emph{Enregistrer ...}} ou
\textbf{\emph{Fichier}} \Rar \textbf{\emph{Enregistrer sous ...}}
pour voir appara�tre la fen�tre de la figure \ref{savettc}.
Parcourir le disque pour trouver o� enregistrer l'horaire. Entrer
le nom � attribuer � l'horaire (avec ou sans extension
\verb!.dia!) et cliquer sur le bouton \textbf{\emph{Enregistrer}}.

\begin{figure}[h]
% Requires \usepackage{graphicx}
    \begin{center}
        \includegraphics[width=2.0in]{UserManualInputs/images/savett.eps}
        \caption{Sauvegarde d'un horaire}\label{savettc}
    \end{center}
\end{figure}

\subsubsection{Ouverture d'un horaire}

Pour ouvrir un horaire, lancer le menu \textbf{\emph{Fichier}} \Rar \textbf{\emph{Ouvrir horaire ...}} pour voir appara�tre la fen�tre de la figure \ref{openttc}. Parcourir le disque pour trouver l'emplacement du fichier d'horaire (avec l'extension \verb!.dia!) � ouvrir. S�lectionner le fichier et cliquer sur le bouton \textbf{\emph{Ouvrir}} ou double-cliquer simplement sur le fichier pour l'ouvrir.

L'op�ration d'ouverture d'un horaire ex�cutera automatiquement l'affectation initiale une fois les donn�es charg�es.

\begin{figure}[h]
% Requires \usepackage{graphicx}
    \begin{center}
        \includegraphics[width=2.0in]{UserManualInputs/images/opentt.eps}
        \caption{Ouverture d'un horaire}\label{openttc}
    \end{center}
\end{figure}

\subsection{Phase 5: Exportation}

Pour exporter les donn�es, aller au menu \textbf{\emph{Fichier}}, puis s�lectionner \textbf{\emph{Exporter}}; deux fichiers seront cr��s dans le repertoire racine de votre document avec les noms suivants:
\begin{itemize}
    \item \verb!S814.HORAIRE! (il contient l'horaire);
    \item \verb!S813.ASSGRO! (il contient les �tudiants, leurs examens).
\end{itemize}

%\chapter{Description des menus, des dialogues et des fen�tres}

\section{DESCRIPTION DES MENUS}


\begin{description}
    \item[Fichier :] permet de faire appel aux fonctions classiques relatives aux fichiers : Nouveau
projet, Nouveau, Ouvrir, Fermer, Enregistrer, Enregistrer sous, Importer manuellement,
D�finir le fichier Import Auto, Importer automatiquement, Exporter et Quitter.
    \item[Affichage :] permet de faire appel aux fonctions n�cessaire � l'affichage des fen�tres associ�es � un projet d'horaire.
    \item[Affectation :] permet d'affecter (fixer) certains choix relatifs � la construction d'un horaire, par exemple, un �tudiant dans un groupe donn� ou une activit� � une p�riode d�termin�e, etc. Contient aussi les menus qui permettent la construction automatique optimis�e d'un horaire.
    \item[Rapport :] permet de faire appel aux fonctions qui donnent acc�s aux rapports et de les sauvegarder comme des fichiers � texte �.
    \item[Pr�f�rences :] permet le changement de certains param�tres du programme.
    \item[Fen�tre :] Permet d'ouvrir une fen�tre de listes de conflits pr�c�demment ouverte.
    \item[Aide :] permet d'obtenir de l'information sur le logiciel ainsi que de l'aide.
\end{description}    
 

%  
% 
% 
% 
% 

\section{Les livrables de \dbug}
\begin{itemize}
\item Standard de codage

\item Liste d'outils de d�veloppement, guide d'installation des
outils. Avec des pages t�moin de chaque outil en �tat de marche.

\item Guide de la gestion de la configuration

\item Guide de l'administrateur du site Web et de la base de
donn�es

\item Description de l'architecture (en utilisant Struts)

\item Sch�ma de l'information de la base de donn�es

\item Mod�les objet et relationnel, ansi que l'interface entre les
deux.


\end{itemize}
\section{Les livrables de \tict}
\begin{itemize}
\item Standard de codage

\item Liste d'outils de d�veloppement, guide d'installation des
outils. Avec des pages t�moin de chaque outil en �tat de marche.

\item Guide de la gestion de la configuration

\item Guide de l'administrateur du site Web et de la base de
donn�es

\item Description de l'architecture (en utilisant Struts)

\item Sch�ma de l'information de la base de donn�es

\item Mod�les objet et relationnel, ansi que l'interface entre les
deux.

\end{itemize}

%\part{Une partie}
%\include{revue}
%\include{theorie}

%\part{Derni�re partie}
%\chapter{Conclusion}

Bla bla bla bla bla.

Bla bla bla bla bla.

%\chapter*{Glossaire}

Bla bla bla bla bla.

Bla bla bla bla bla.

%\chapter*{Index}

A
AAA
ABC.

B
Bla bla bla bla bla.

%\include{formules}
\end{articleDX}
\end{document}
