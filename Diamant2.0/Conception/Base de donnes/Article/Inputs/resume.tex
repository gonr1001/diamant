\chapter*{R�sum�}

C'est vrai que une partie importante dans le temps de
d�veloppement d'un logiciel est consacr� � la validation et
nettoyage des donn�es, ce processus peut devenir tr�s complexe et
tr�s co�teux.

Le logiciel \diamant{} utilise certains fichiers qui � l'heure
actuel sont en enregistr�s en format texte, la gestion et
validation de ce type de fichiers est complexe.

Ce document pr�sente comment des donn�es utilis�es par le logiciel
\diamant{} qui r�sident dans des fichiers text sont charg�es dans
une base de donn�es relationnelle, dans cette base de donn�es nous
r�alisons certains validations d'int�grit� des donn�es. Une fois
que les donn�es sont valides, nous pouvons g�n�rer des fichiers en
format XML, qui est devenu un standard pour �changer des donn�es.

Avec ce processus, nous pouvons garantir que \diamant{} peut se
concentrer dans les t�ches de cr�er des horaires, sans distraire
son attention dans les activit�s de validation des donn�es.

Ce papier pr�sente les d�tails de formats de fichiers, fichiers
originaux et fichiers produits; avec des indications sur
l'utilisation de champs et les classes concernant cette partie de
l'application ainsi que la correspondance entre champs des
enregistrements fichiers et donn�es membres, tout cela pour le cas
pr�cise de \diamant{}.

Mais aussi nous avons comme objectif montrer un mod�le de donn�es
et un processus de validation de donn�es assez ouvert et
extensible pour l'utiliser dans autres cas.
