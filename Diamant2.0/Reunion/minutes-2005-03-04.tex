\documentclass[11pt]{article}

\usepackage[frenchb]{babel}
\usepackage[ansinew]{inputenc}
\usepackage[T1]{fontenc}

\usepackage{fancyhdr}

\pagestyle{fancy}


%% Definition du nom de chaque membre de l'equipe
\newcommand{\nomGroupe}{eXit}
\newcommand{\ruben}{Ruben Gonz�lez-Rubio}
\newcommand{\bernard}{Bernard Beaulieu}
\newcommand{\pascal}{Pascal Durocher}
\newcommand{\rosa}{Rosa Lourdes Garc�a}
\newcommand{\domingo}{Domingo Palao Mu�oz}
\newcommand{\yannick}{Yannick Syam}
\newcommand{\andres}{Andres Nunez}

%% Definition du lieu et du moment de la reunion
\newcommand{\lieureunion}{local C1-3072}
\newcommand{\datereunion}{4 f�vrier 2005}
\newcommand{\heurereunion}{14h30}

\newcommand{\checkpresence}{$\hspace{0.5ex}\bigodot{}$}
\newcommand{\noncheckpresence}{$\bigcirc{}$}

\fancyhf{} \cfoot{\thepage} \rhead{\large{\textit{Minutes de la r�union
hebdomadaire. Diamant2.0}}}

\begin{document}

\section*{Pr�sence}

\begin{tabular}{c}
%%        \hline

    \begin{tabular}{lccc}
        \textbf{Nom} & \textbf{P} & \textbf{R} & \textbf{H} \\
        \hline
        \ruben     & \checkpresence & \noncheckpresence & \noncheckpresence \\
        \domingo   & \checkpresence & \noncheckpresence & \noncheckpresence \\
        \hline
    \end{tabular}
    \vspace{0.1cm}
    \\
    \hfill

    \renewcommand{\arraystretch}{0.5}
    \begin{tabular}{l}
        \tiny P = \textbf{P}r�sent \\
        \tiny R = Arriv� en \textbf{r}etard \\
        \tiny H = Parti \textbf{h}�tivement \\
    \end{tabular}

    \\
%%        \hline
\end{tabular}

\section*{Minutes}

\subsection*{Discussion}

Domingo a montr� la presentation
DiamantExtreme{$\backslash$}Diamant2.0{$\backslash$}Conception{$\backslash$}CreationGrille--2.ppt.

Nous avons parl� de l'importance de garder un standard dans les termes �
utiliser, surtout � l'interne.

L'interface de la grille est correcte on commence avec la d�finition des
sc�narios.



\end{document}
